Een van de complexere taken is printen. Als het om \'e\'en printer gaat en \'e\'en werkstation dan valt het wel mee, maar als je in je netwerk meerdere verschillende printers hebt van verschillende merken en je hebt verschillende besturingssystemen die moeten kunnen printen dan wordt het al snel complex. Om deze complexiteit op te lossen kan je gebruik maken van een printserver.

De printserver is een stuk software dat een print opdracht van een client aanneemt, deze in de rij plaatst en als de printer beschikbaar is de opdracht naar de printer stuurt. Dit lijkt een simpel proces, maar er zijn verschillende printertalen en verschillende document standaarden. Op de \'e\'en of andere manier moeten die verschillende documenten om gezet worden naar de taal die de betreffende printer snapt.

De verschillende document formaten die een printer snapt zijn:
\begin{itemize}
\item PostScript (EPS)
\item PDF
\item PCL
\end{itemize}

PostScript\texttrademark{} is ooit bedacht door Adobe Systems tussen 1982 en 1984. Het is jarenlang de standaard geweest als beschrijving van een document zodat een printer het kon afdrukken. PDF heeft het sinds 1992 langzaam aan verdrongen. Ook PDF is bedacht door Adobe Systems.

PCL is een taal bedacht door de mensen bij Hewlett-Packard, maar wordt allang niet meer alleen op HP printers gebruikt.
 
