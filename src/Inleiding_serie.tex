Deze Linux cursus beoogt aan te sluiten bij het Linux Essentials examen van de LPI (Linux Professional Institute) en dient als voorbereiding op het MBO ICT Systems and Devices Expert examen. Voor het leren gebruiken van de grafische interface en de command line maken we gebruik van CentOS en om kennis te maken met het gebruik van Linux als server installeren we Debian. De keuze om CentOS als werkstation te installeren en Debian als server is volledig willekeurig. Het doel is dat de studenten kennis maken met de rpm en dpkg package managers en leren dat het ene Linux systeem het andere niet is.

Alle Linux systemen zullen ge\"installeerd worden als virtuele machines op Virtual Box (\url{https://www.virtualbox.org/}). Door gebruik te maken van virtuele machines zijn we niet afhankelijk van de onderliggende hardware. De keuze voor VirtualBox heeft te maken met het feit dat dit product gratis te gebruiken is en beschikbaar is voor zowel Windows, Mac OS X als Linux.

Voor de CentOS machine is 15G vrije schijfruimte nodig en voor het Debian systeem 8G, wat een totaal aan 23G vrije schijfruimte vereist. Voor elke machine hebben we 2G RAM nodig, dus een totaal van 4G RAM moet vrij beschikbaar zijn.
