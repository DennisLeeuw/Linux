Om gebruikers toe te voegen aan het systeem zijn er twee tools. De eerste is \texttt{useradd}\index{useradd}\index{commandos!useradd} en de andere is \texttt{adduser}\index{adduser}\index{commando!adduser}. Het programma \texttt{useradd} is een low-level tool, de standaard manier om gebruikers aan te maken is via \texttt{adduser} en dat is dan ook wat we gaan gebruiken.

\begin{lstlisting}[language=bash]
$ sudo adduser eengebruiker
\end{lstlisting}
Om gebruikers te kunnen aanmaken hebben we root-rechten nodig, dus we gebruiken het \texttt{sudo} commando.

Het \texttt{adduser} commando maakt bijna alles automatisch aan. Het enige dat je hoeft te doen is te vertellen wat het wachtwoord van de gebruiker is en welke informatie er in het GECOS-veld terecht moet komen.

Je mag aan \texttt{adduser} ook meegeven dat zaken anders moeten zijn. Bijvoorbeeld dat een gebruiker een andere shell gebruikt dan bash. Dat kan door de optie \texttt{--shell /bin/chsh} mee te geven. Lees de man-page van \texttt{adduser} eens door met wat er nog meer mogelijk is.

Als op het systeem UPG gebruikt wordt dan zal je zien dat er ook gelijk een groep voor de gebruiker aangemaakt is. Gebruik \texttt{id} om te zien wat de ID is van de nieuwe gebruiker.

