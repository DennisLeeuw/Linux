\documentclass[a4paper,12pt,twoside,openright,titlepage]{book}

%Additional packages
\usepackage[utf8]{inputenc}
\usepackage[T1]{fontenc}
\usepackage[dutch,english]{babel}
\usepackage{syntonly}
\usepackage[official]{eurosym}

% Handle images
%\usepackage[graphicx]
\usepackage{graphicx}
\graphicspath{ {./images/}{./styles/} }
\usepackage{float}
\usepackage{wrapfig}

% Handle URLs
\usepackage{xurl}
\usepackage{hyperref}
\hypersetup{colorlinks=true, linkcolor=blue, citecolor=blue, filecolor=blue, urlcolor=blue, pdftitle=, pdfauthor=, pdfsubject=, pdfkeywords=}

% Tables and listings
\usepackage{tabularx}
\usepackage{scrextend}
\addtokomafont{labelinglabel}{\sffamily}
\usepackage{listings}
\usepackage{adjustbox}

% Turn on indexing
\usepackage{imakeidx}
\makeindex[intoc]

% Define colors
\usepackage{color}
\definecolor{ashgrey}{rgb}{0.7, 0.75, 0.71}



% Listing style
\input{styles/lstset}

% Uncomment for production
% \syntaxonly

% Style
\pagestyle{headings}

% Turn on indexing
\makeindex[intoc]

% Define document
\author{D. Leeuw}
\title{Linux introductie}
%\subtitle{Linux voor MBO niveau 4 en het LPI Linux Essentials examen}
%\subject{Een Praktische Gids}
\date{\today\\v.1.1.0}

\begin{document}
\selectlanguage{dutch}

\maketitle

\copyright\ 2020-2022 Dennis Leeuw\\

\input{styles/licentie}

%%%%%%%%%%%%%%%%%%%
%%% Introductie %%%
%%%%%%%%%%%%%%%%%%%

\frontmatter
\chapter{Over dit Document}
\input{src/OverDitDocument}
\begin{flushleft}
\begin{table}[h!]
\centering
\begin{tabularx}{\textwidth}{ |c|c|c|X| }
\hline
	Versienummer &
	Auteurs &
	Verspreiding &
	Wijzigingen\\
\hline
	1.1.0 &
	Dennis Leeuw &
	&
	Installatie opdracht voor CentOS verplaatst naar opdrachten document.\\
\hline
	1.0.0 &
	Dennis Leeuw &
	&
	Veel taalfouten verwijderd en herschrijvingen om het een beter leesbaar document te maken.\\
\hline
	0.9.0 &
	Dennis Leeuw &
	&
	Eerste release na splitsing GUI document in een Intro en een GUI document\\
\hline
\end{tabularx}
\caption{Document wijzigingen}
\label{table:1}
\end{table}
\end{flushleft}



%%%%%%%%%%%%%%%%%
%%% De inhoud %%%
%%%%%%%%%%%%%%%%%
\tableofcontents

\mainmatter
\chapter{Inleiding}
Deze Linux cursus beoogt aan te sluiten bij het Linux Essentials examen van de LPI (Linux Professional Institute) en dient als voorbereiding op het MBO ICT Systems and Devices Expert examen. Voor het leren gebruiken van de grafische interface en de command line maken we gebruik van CentOS en om kennis te maken met het gebruik van Linux als server/command line installeren we Debian. De keuze om CentOS als werkstation te installeren en Debian als server is volledig willekeurig. Het doel is dat de studenten kennis maken met de zowel rpm/dnf en de apt package managers en leren dat het ene Linux systeem het andere niet is.

Alle Linux systemen zullen ge\"installeerd worden als virtuele machines. Door gebruik te maken van virtuele machines zijn we niet afhankelijk van de onderliggende hardware. De keuze van de virtuele omgeving is aan de gebruiker. Ons advies zou zijn om gebruik te maken van VirtualBox en VMware Workstation, beide zijn gratis en kunnen op de meest gangbare operating systems gebruikt worden.

Voor de CentOS machine is 15G vrije schijfruimte nodig en voor het Debian systeem 5G, wat een totaal aan 20G vrije schijfruimte vereist. Voor elke machine hebben we 2G RAM nodig, dus een totaal van 4G RAM moet vrij beschikbaar zijn.



\chapter{Wat is Linux?}
Om de vraag te kunnen beantwoorden wat Linux is moeten we een stukje terug in de tijd en wel naar het eind van de jaren '60 uit de vorige eeuw. Een groepje programmeurs werkend aan een project genaamd Multics\index{Multics} bij Bell Labs van AT\&T zaten op een dood spoor of beter Bell Labs vond het een dood spoor en cancelde het project, dus schreef een van die programmeurs in ongeveer een maand tijd in 1969 een simpeler alternatief. De ontwikkelaar was Ken Thompson\index{Thompson, Ken} die samen met Dennis Ritchie\index{Ritche, Dennis} een team leidde.\par

Een ander lid van het team Brian Kernighan\index{Kernighan, Brian} schijnt met de naam Unics gekomen te zijn als woordgrap op Multics. Hoe de naam ooit \index{Unix}Unix is geworden is niet bekend.\par

De gedachte achter Unix is ooit in 1979 mooi beschreven door Dennis Ritchie:
\selectlanguage{english}
\textquote{What we wanted to preserve was not just a good environment in which to do programming, but a system around which a fellowship could form. We knew from experience that the essence of communal computing, as supplied by remote-access, time-shared machines, is not just to type programs into a terminal instead of a keypunch, but to encourage close communication.}
\selectlanguage{dutch}
Het ging dus om een systeem waarop samengewerkt kon worden en dan vooral geprogrammeerd.\par

Uit de wens vanuit de organisatie om tekst te kunnen verwerken werd het systeem uitgebreid met tekstverwerkingsfuncties
en een eerste tekst editor en tekst formatter. Hieruit blijkt meteen de filosofie achter Unix. Maak kleine tools die
\'e\'en ding goed doen en gebruik ze gezamenlijk om complexe dingen te doen. Er is dus een editor en een formatter. De
eerste tekst formatter heette roff\index{roff} en deze werd als snel opgevolgd door troff\index{troff}, die je nog steeds terug vindt op de
systemen. De unix manual-pages\index{manual-pages}, waarover later meer, worden opgemaakt met behulp troff.\par

De editor is inmiddels verschillende keren verbeterd, wat een ander voordeel is van dit \textquote{small is beautiful} principe. Je kan kleine stukjes aanpassen terwijl wat goed is behouden blijft.

De eerste versies van Unix waren geschreven in assembly. Assembly is een programmeertaal die volledig toegespitst is op de onderliggende hardware. Iets dat in assembly geschreven is werkt dan ook alleen op \'e\'en type machine. In 1974 kwam Unix versie 4 uit die bijna volledig herschreven was in de
taal \index{C}C. C is niet machine afhankelijk, er is een compiler nodig om de taal om te zetten naar de machine afhankelijke machinetaal. Door het gebruik van C werd het mogelijk Unix ook op andere systemen te draaien dan het PDP systeem
waar het oorspronkelijk voor geschreven was. Unix kon de wereld gaan veroveren.\par

Bell Labs mocht door juridische afspraken met de Amerikaanse overheid geen commerci\"ele zaken ondernemen buiten de telefonie, daardoor werd de
vraag naar Unix beantwoord door alleen geld te rekenen voor de media (tapes) en het verzenden van het systeem, en niet voor het
product zelf.\par

Omdat het systeem goedkoop verkrijgbaar was was het een ideaal systeem voor universiteiten. Studenten konden er relatief goedkoop op leren programmeren. Door studenten en andere
gebruikers werden er in de loop van de tijd meer en meer programma's geschreven en verbeteringen gemaakt. Deze verbeterde stukken software werden met elkaar gedeeld. Van verkoop was nog geen sprake. Het voelde als hobby en leer projecten. Soms waren de wijzigingen zo groot dat er een variant van het oorspronkelijk Unix werd gedeeld (gedistribueerd). Deze collecties van programma's werden dan ook distributies genoemd. E\'en van de meest bekende van deze distributies is die van University of California die ontwikkeld werd door de Berkeley Computer Systems Research Group, de Berkeley Software Distribution\index{Berkeley Software Distribution} of afgekort \index{BSD}BSD.

Een klein bedrijfje in de US was een van de eerste bedrijven die Unix naar de microcomputer, ofwel de PC, bracht. Tot dan draaide het eigenlijk alleen op grote servers. Hun Unix versie heette Xenix\index{Xenix} en het bedrijf stond bekend als Microsoft\index{Microsoft}. Grote bedrijven als IBM (AIX\index{AIX}) en HP (HPUX\index{HPUX}) hebben hun eigen Unix versie voor grote machines.

De vele verschillende systemen die ontstonden waren voor de eindgebruikers een ramp. Wat op het ene systeem aanwezig was
was op een andere niet aanwezig en omgekeerd. Een Unix-applicatie op de ene distributie werkte werkte niet zomaar ook
op een andere. Gebruikers wilden standaardisatie. Software moest uitwisselbaar zijn. De standaard die ontstond in 1988
is de \index{POSIX}POSIX standaard van de IEEE. Het beschrijft welke software minimaal aanwezig moet zijn en ook aan welke functionalitiet die software minimaal moet voldoen. Extra functionaliteit is dus geen probleem. Als een systeem aan de eisen van POSIX-standaard voldoet heet die dan ook POSIX-compliant.

In 1983 bracht Richard Stallman\index{Stallman, Richard} een nieuw project in de wereld genaamd het \index{GNU}GNU Project, waarbij GNU staat
voor GNU's not Unix! Dit om aan te geven dat het gebaseerd is op Unix maar geen Unix componenten bevat. GNU is een "from
scratch" geschreven systeem dat volledig open source is. From-scratch betekent dat alle code opnieuw geschreven is en Open Source wil zeggen dat van het hele systeem alle broncode openbaar
beschikbaar is, hierdoor kan het gebruikt worden als studiemateriaal. Het heeft tevens het voordeel dat iedereen fouten
uit de software kan halen en aan de software kan bijdragen om het beter te maken. Dit is dan ook altijd de strijd van
Richard Stallman geweest en zijn daarvoor opgerichte organisatie The \index{Free Software Foundation}Free Software
Foundation, software moest vrij zijn en voor iedereen toegankelijk. Om dit te bewerkstelligen stelde Stallman het
\index{copyleft}copyleft in, in plaats van het copyright, en maakte een licentie genaamd de \index{GPL}GPL, \index{GNU Public License}GNU Public
License, die ervoor moest zorgen dat de gemaakte software niet zomaar weer gesloten kon worden.\par

De software werd door vele verschillende distributies in die tijd gebruikt om software van Unix geheel of gedeeltelijk
te vervangen. Missend onderdeel was echter jarenlang een kernel. Het stukje software dat ligt tussen de
gebruikersinterface en de hardware.\par

In 1991 schreef een Finse student met de naam Linus Torvalds\index{Torvalds, Linus} een berichtje in een news-group dat hij bezig was met zo'n
kernel. Niets groots zei hij, maar wel onder de GPL-licentie van de Free Software Foundation. Velen werden aangetrokken
door dit project en gingen Linus helpen zijn kernel verder uit te breiden. Die kernel ging naar de maker heten en werd
\index{Linux}Linux genoemd.\par

Linux is zoals gezegd een kernel; een abstractie laag tussen de hardware en de gebruikers interface. Voor een compleet systeem is er veel meer nodig. De overige software op wat wij nu een LInux systeem noemen is dan ook veelal afkomstig van het
GNU-project. Fanatiekelingen willen dan ook graag dat je spreekt van GNU/Linux, maar de hele wereld
spreekt meestal gewoon over Linux als we een systeem bedoelen met een Linux kernel.\par

De Linux kernel kan natuurlijk gecombineerd worden met vele software pakketten en dat gebeurt ook. En net als bij
Unix werden de systemen die ontstonden distributies genoemd. E\'en van de eerste distributies was Slackware, later
kwamen SuSE, Debian, Red Hat, CentOS, Ubuntu en nog vele vele anderen.\par

Het meest gebruikte Unix desktop systeem is ongetwijfeld Darwin het open source besturingssysteem van Apple dat de basis
vormt voor Mac OS X en op de telefoon is dat Android het systeem van Google dat een Linux kernel en andere open source
tools onderwater gebruikt.\par

En zo komen we na een lang verhaal aan het einde van deze simpele geschiedenis over Unix en Linux. Waarin hopelijk duidelijk is geworden waarom Linux Linux wordt genoemd, duidelijk is waarom Linux open source is en waarom delen zo'n belangrijk onderdeel is van de cultuur rond Unix en Linux systemen.



\chapter{Waarom Open Source?}
Wat Linux en het GNU project bijzonder maken is het feit dat alle code open source\index{open source} is. Je mag er mee doen en laten wat
je wilt, je mag het aanpassen, je kan het inzien en je mag het natuurlijk gewoon gebruiken. Voor het merendeel is de software nog gratis ook. Dat is
natuurlijk bijzonder in een wereld die draait om commercie. Daarom willen hier iets dieper duiken in de wereld van open source.

De eerst versies van Unix zoals geschreven door Ken Thompson, Dennis Ritchie en de overige leden
van het team bij Bell Labs was geschreven is de assembly language. Assembly language\index{Assembly} is een taal die heel dicht ligt bij
wat computers snappen en daarmee altijd hardware afhankelijk is. In de tijd dat Unix werd ontwikkeld geloofde men dat je
assembly language nodig had om een besturingssysteem snel genoeg te laten zijn. Dennis Ritchie nam de taal B, ontwikkeld
door Ken Thompson, maakte verbeteringen en kwam met \index{C}C in
1972. In 1973 kwam Unix versie 4 uit die voor een groot deel herschreven was in C en daarmee aantoonde dat een hogere
programmeertaal gebruikt kon worden om besturingssystemen in te schrijven die snelgenoeg waren, maar belangrijker nog omdat er een hogere
programmeertaal werd gebruikt was Unix opeens overdraagbaar naar andere hardware en dat had voor de ontwikkeling van software grote voordelen. Software kon opeens geschreven worden op het ene systeem en gebruikt worden op een totaal ander systeem.

Voor het programmeren in C heb je een C-compiler\index{Compiler}, een linker\index{Linker} en een C-Library\index{Library}\index{C!Library} nodig. Library is het Engelse woord voor
bibliotheek. Een C-bibliotheek bevat een aantal standaardfuncties die je kan gebruiken in een programma.
Een heel simpel programma als Hello World\index{Hello World} ziet er in C zo uit:

\lstinputlisting[language=C]{c/helloworld.c}

De printf functie die de woorden \textquote{Hello World!} op het scherm laat zien is zo'n standaard functie uit de C-Library.
 De C-bibliotheek bevat een aantal standaardfuncties die je kan gebruiken in een programma, zo hoeft niet
elke programmeur de printf functie te programmeren. Functies in een library zijn dus kleine stukjes code die je met elkaar delen kan zodat iedereen in zijn programma deze functies gebruiken kan mits de library op het systeem aanwezig is.

De compiler is verantwoordelijk voor het omzetten van de C-code in machinetaal\index{Machinetaal}. Machinetaal is binair, daar computers werken met 1 en 0 en niets anders.
Het proces om C-code om te zetten in binaire code heet dan ook compileren. De compiler vertaalt alles wat er geschreven is in C in 1-en en 0-en zodat de computer ermee kan werken. Maar omdat je ook functies gebruikt uit de C-library moet ook daar nog iets mee gebeuren, daarvoor zorgt de linker. De linker zorgt ervoor dat de functie uit de C-library gelinkt wordt aan het programma dat je geschreven hebt. Een compiler is niet alleen hardware afhankelijk, maar ook operating systeem afhankelijk. Een compiler voor Mac OS X maakt van C machinetaal voor Mac OS X en een C-compiler voor Windows maakt machinetaal voor Windows.

Er zijn twee manieren waarop de linker ervoor kan zorgen dat bijvoorbeeld de printf-functie gelinked kan worden met je programma. Het
kan statisch en dynamisch. Statisch betekent dat een kopie van de functie toegevoegd wordt aan je programma. Bij dynamisch linken betekent het dat er in je programma een verwijzing
komt te staan naar de binaire printf-functie in die specifieke binaire C-library. Je C-programma wordt zo afhankelijk van deze specifieke
versie van de C-library die aanwezig is op je systeem.

Het voordeel van statisch linken is dat het programma onafhankelijk is van de C-library, het nadeel is dat het binaire-programma vele malen groter wordt omdat dat alle code uit de C-library (of andere bibliotheken) ook aan het programma wordt toegevoegd. Bij dynamisch linken is dit juist omgekeerd. Het programma blijft kleiner, laadt daardoor sneller van disk en neemt minder geheugen ruimte in, maar dat gaat tenkoste van de portabiliteit, kortom het kan alleen nog gebruikt worden op systemen die exact dezelfde versies van de libraries ge\"instaleerd heeft.
Dat laatste is geen probleem zolang je het programma gebruikt op
systemen die dezelfde libraries hebben als jij, zoals het geval is bij mensen die dezelfde distributie gebruiken als
jij. Ook als er een kleine wijziging gemaakt wordt in de printf-functie die geen invloed heeft op de binaire syntax van de printf-functie
dan kan je programma gelijk gebruik maken van deze verbetering doordat je alleen de C-library update. Je hoeft je
programma dan niet opnieuw te compileren. Er zijn dus vele voordelen aan dynamisch linken en daarmee is het dan ook de meest gebruikte methode op Linux systemen.

Dat dynamisch linken klinkt allemaal vreselijk complex en dat is het ook. Er zijn momenten waarop er zoveel wijzgingen zijn in bijvoorbeeld een nieuwe C-library dat jij je programma opnieuw moet compileren om het te laten werken met die nieuwe C-libary, maar kleine wijzigingen in de C-library kunnen ervoor zorgen dat dat niet hoeft en dan blijft je programma gewoon werken. Dat verschil heeft te maken met de API\index{API} (Application Programming Interface\index{Application Programming Interface}) en ABI\index{ABI} (Application Binary Interface\index{Application Binary Interface}). De API van een functie is de syntax van de functie, als deze
verandert dan moet je je programma opnieuw compileren en als het tegenzit moet je zelfs je code aanpassen. Als echter de API blijft zoals hij is en er zijn alleen kleine wijzigen zodat de ABI niet veranderd, dan heeft je programma geen last van de wijziging.

Programma's die werken op je telefoon, je Windows systeem of op Mac OS X worden je vaak aangeboden als binary, kortom ze zijn al door iemand gecompiled. Deze applicaties kan je direct gebruiken, maar alleen op het systeem waarvoor ze gecompiled zijn. Je kan een Windows .exe niet gebruiken op een Mac OS X systeem.

Als je (ook) de beschikking hebt over de broncode dan kan je die code ook compilen op je eigen computer en zorgen dat
die ook werkt op jou systeem. Je bent dan niet meer afhankelijk van een leverancier die jou systeem moet ondersteunen.
Soms moet je wel wat aanpassingen maken om het geheel goed te laten werken. De grafische interface van Windows is heel
anders dan die van Mac OS X, en daar zit dan ook een heel andere library onder. Dus als je een Windows applicatie op
een Mac wil compileren zul je wel wat programmeer werk moeten doen. Maar als een applicatie is geschreven op een
Debian systeem dan kan deze meestal zonder enige wijziging gecompileerd worden op een CentOS systeem.

En daar zit de kracht van open source. Met het delen van de broncode wordt de reikwijdte van die software groter. Zoals gezegd moet je soms wel wijzigen maken om het te laten werken en dus is het bijna een eis dat je de software ook mag aanpassen en die eisen zijn in open source licenties vastgelegd.



\chapter{Gebruiksrechten en licenties}
Het auteursrecht\index{Auteursrecht} stamt uit 1710 en was oorspronkelijk bedoeld om de drukker te beschermen. Later is dit over gegaan op
de auteur. Omdat ook een programmeur een schrijver is geldt er voor broncode\index{Broncode} dezelfde rechten als op boeken. Maar wat
houdt auteursrecht nu eigenlijk in.

{\selectlanguage{dutch}
Volgens de wet:}

{\selectlanguage{dutch}
{\textquotedbl}Het auteursrecht is het uitsluitend recht van de maker van een werk van letterkunde,
wetenschap of kunst, of van diens rechtverkrijgenden, om dit openbaar te maken en te verveelvoudigen, behoudens de
beperkingen, bij de wet gesteld.{\textquotedbl} (Artikel 1 Auteurswet) }

{\selectlanguage{dutch}
Dit zegt dus iets over het openbaar maken en het kopi\"eren (verveelvoudigen) van het gemaakte werk. De auteur mag dit
doen en niemand anders (uitsluitend recht van de maker). Als een programmeur software schrijft en dit aan iemand anders
geeft, dan mag deze de software gebruiken maar niet kopi\"eren en verder verspreiden.}

{\selectlanguage{dutch}
Voor de auteur zijn de regels dus in de wet vastgelegd, welke rechten een gebruiker van de software heeft dat mag de
auteur zelf bepalen. Deze rechten worden vastgelegd in een gebruikerslicentie\index{Gebruikerslicentie}. Elke auteur kan zijn eigen licentie
maken, wat veel bedrijven dan ook doen. De meest bekende is waarschijnlijk Microsoft's \index{EULA}EULA; End-Users
License Agreement\index{End-Users License Agreement}. De EULA zegt in het kort dat Microsoft niet verantwoordelijk is voor de gemaakt software en
eventuele fouten die het mocht bevatten, dat je het op 1 machine mag gebruiken, je mag geen kopie\"en mag maken en het mag
maar door 1 persoon tegelijkertijd gebruikt worden.}

{\selectlanguage{dutch}
Omdat Unix ontstaan is door software met elkaar te delen zijn er andere licenties ontstaan. De
meestvoorkomende zullen we in dit hoofdstuk behandelen.}


\section{Gebruikslicenties}
\subsection{MIT}
\input{src/MIT}
\subsection{BSD}
\input{src/BSD}
\subsection{GPL}
\input{src/GPL}
\section{Creative Commons}
Naast software wordt er natuurlijk ook documentatie geschreven voor de software, worden er logo's gemaakt voor software en worden er websites
ontworpen, soms worden er ook filmpjes gemaakt. Kortom we willen vaak meer creatieve uitingen vangen in een
licentie dan alleen de software. In het begin werden de software licenties ook voor deze zaken gebruikt, maar dat bleek
niet altijd toereikend. De oplossing is uiteindelijk gevonden in een set van voorwaarden die bekend staan als de
Creative Commons.\par

Een Creative Commons\index{Creative Commons} (CC\index{Licentie!CC}) licentie zegt dat je werk hergebruikt mag worden. Met wat toevoegingen aan de licentie kan je zelf bepalen wat er wel niet met je werk
gedaan mag worden. De meest eenvoudige vorm is de CC BY. Deze zegt dat iedereen van alles met je werk mag doen, maar
dat ze daarbij altijd aan naamsvermelding moeten dien. Een beetje zoals de MIT licentie.\par

Zaken die je toe kan voegen zijn:

\begin{labeling}{ND}
\item [ND] staat voor dat er Geen Afgeleide werken gemaakt mogen worden. Een logo mag dus wel gebruikt worden maar mag niet gewijzigd worden. ND is een afkorting voor No Derivatives.
\item [NC] wat staat voor Niet Commercieel (Non Commercial). Het product of een afgeleide ervan mag niet commercieel gebruikt worden.
\item [SA] staat voor dezelfde licentie (Share Alike). Het product of een afgeleide daarvan mag dan niet van licentie veranderen. Dit lijkt erg op de GPL licentie.
\end{labeling}

Dit boek is uitgebracht onder de CC BY NC SA. Dat wil dus zeggen dat als er een gewijzigd werk gemaakt wordt van dit
document dan moeten daar de namen van de auteurs van dit document bij vermeld worden, mag het niet commercieel
uitgegeven worden en moet het onder dezelfde licentie verspreid worden. Je mag dus wel geld vragen voor het feit dat je een
boek bijvoorbeeld gedrukt hebt.


\section{Open standaarden}
Een verwarring die weleens wil ontstaan is het verschil tussen open source en open standaarden en toch is daar een
wezenlijk verschil.\par

Open standaarden beschrijven hoe bijvoorbeeld data uitgewisseld kan worden. Protocollen als SMTP, POP3, IMAP, Telnet en
FTP zijn allemaal beschreven in documenten die vrij op Internet toegankelijk zijn. Iedereen, dus ook de open source
wereld, kan deze standaarden implementeren en er dus voor zorgen dat verschillende systemen, open source en
commercieel, met elkaar kunnen communiceren. Gesloten protocollen die door een bedrijf zijn bedacht kunnen alleen door
dat bedrijf gebruikt worden, hoewel door luisteren op het netwerk er natuurlijk ook gekeken kan worden hoe het protocol
werkt.

Ook voor het uitwisselen van data is het van belang dat er open standaarden zijn. Het feit dat je een document dat je in Microsoft Word maakt alleen in Word kan lezen is natuurlijk een enorme beperking van je vrijheid. Je document zou in elke willekeurige tekstverwerker te openen en wijzigen moeten zijn, het is tenslotte jouw tekst. Toch is er pas sinds 2005 een open standaard voor office documenten. In 2005 werd door Organization for the Advancement of Structured Information Standards (OASIS) de OpenDocument standaard goedgekeurd. Deze standaard beschrijft hoe office documenten eruit moeten zien en omdat het een open standaard is kan iedereen de standaard implementeren, daarmee zou het mogelijk moeten zijn dat elke tekstverwerker een OpenDocument tekstdocument moet kunnen lezen en schrijven.

\section{Open data}
In navolging van open source en open standaarden kwam er ook steeds meer de vraag op hoe dat zit met data. Kan data ook open zijn?

Is onderzoeksdata van een universiteit van de universiteit, van de onderzoeker of van de overheid (of instantie) die het onderzoek betaald heeft? En als die data gedeeld wordt met de wereld wat mag je er dan mee doen. Mag je de data van een onderzoek wijzigen?

Al dit soort zaken zijn het domein van de open data. Het is dus belangrijk dat bij data ook een keuze gemaakt wordt wat er wel en niet mee mag gebeuren.

\section{Open Source Business Model}
Een veel gehoord tegenargument bij (tegen) het gebruik van een open source licentie bij de ontwikkeling
van nieuwe software is dat er geen geld te verdienen zou zijn op deze manier en dat is gedeeltelijk waar. Het \'e\'en keer
ontwikkelen van software, het daarna eindeloos kopi\"eren en er geld voor vragen, dat werkt niet meer. Maar je mag nog
altijd geld vragen voor de distributie, je mag (installatie) hulp aanbieden (helpdesk), cursussen aanbieden, of adviesuren verkopen. Er blijven dus voldoende middelen over om geld te verdienen.


\chapter{Linux Distributies}
Een van de allereerste distributies was het Softlanding Linux System (SLS) door Peter MacDonald in 1992, deze Linux distributie bevatte, als eerste, een grafische interface. Het stond bekent om zijn buggy character en er ontstonden dan ook al snel opvolgers zoals Yggdrasil en Slackware van Patrick Volkerding. Slackware kwam in 1993 uit en is de oudste nog steeds bestaande distributie. Ook Debian is een afgeleide van SLS.

Een Linux distributie is een collectie van software samen met de Linux-kernel. Veel van de software is afkomstig van het GNU-project. Makers van een distributie maken hun eigen keuzes welke software zij belangrijk vinden. Daarom zijn er ook veel verschillende distributies omdat er zoveel mogelijk is met open source in iedereen iets anders belangrijk vindt. We kunnen je in dit document niet kennis laten maken met alle bestaande distributies, maar we kunnen wel een paar van de belangrijkste distributies voor je beschrijven.

De software die meegeleverd wordt met een distributie is allemaal voor gecompileerde software, je krijgt dus binairies net als bij Windows en Mac OS. Je hoeft de software niet meer zelf te compileren. Om die voorgecompileerde software te kunnen installeren is er een package manager nodig. Een stukje software dat de binaries allemaal op de juiste plek op de harddisk zet en er eventueel voor zorgt dat benodigde extra software, zoals libraries, ook ge\"installeerd worden. Verschillende distributies gebruiken verschillende package managers.

\section{Debian}
Debian\index{Debian} is een op SLS gebaseerde distributie. Debian is een distributie die ooit is opgezet door Ian Murdoch, in 1993, en de distributie is vernoemd naar hem en zijn vrouw Debra; Deb-Ian ofwel Debian. Debian wordt ontwikkeld zoals ook open source software wordt ontwikkeld, door vrijwilligers op een volledig open en transparante manier. Ze hebben zelfs hun eigen voorwaarden waaraan iedereen die mee ontwikkeld moet voldoen. Debian is de distributie met de meest beschikbare software pakketten.

De package manager van Debian is apt\index{apt}.

Debian vormt zelf weer de basis voor heel veel andere distributies. Ubuntu\index{Ubuntu} en Linux Mint\index{Linux Mint} zijn gebaseerd op Debian. Ubuntu en Linux Mint zijn vooral populair als desktop omgevingen, terwijl we Debian vaker terug vinden is in de serverruimte.


\section{Red Hat}
Red Hat\index{Red Hat} is de grootste commerci\"ele leverancier van Linux en tegenwoordig onderdeel van IBM. Red Hat wordt in het
bedrijfsleven veel gebruikt. Omdat Red Hat commercieel is moet je betalen voor gebruik. Gelukkig voor ons is er een
community based version genaamd CentOS\index{CentOS} die we gratis kunnen gebruiken. CentOS is dan ook de versie die we in deze serie gebruiken als het om de dekstop omgeving gaat.

Red Hat gebruikt de Red Hat Packet Manager \index{rpm}rpm welke door veel distributies ook gebruikt wordt.

Ook Red Hat kent verschillende afgeleide systemen zoals CentOS en Scientific Linux (gemaakt door Fermilab, CERN, DESY en ETH Zuric).

\section{OpenSuSE}
\input{src/SuSE}

%%%%%%%%%%%%%%%%%%%%%
%%% Index and End %%%
%%%%%%%%%%%%%%%%%%%%%
\backmatter
\printindex
\end{document}

%%% Last line %%%
