C\index{C} is een procedurele programmeertaal. Hij is ontworpen bijna gelijk met het ontstaan van Unix. De ontwerpers van C zijn Dennis Ritchie en Brian Kernighan, de taal stamt al uit 1969 en zijn voorlopers waren daadwerkelijk de talen 'A' en 'B'.

Een voorbeeld van een programma geschreven in C is Hello World uit het C programmeer boek. Dit is het eerste programma dat je als voorbeeld kreeg. Dit heeft veel navolging gehad in andere talen. Wij zullen bij elke taal die we hier noemen een voorbeeld geven van Hello World.
\lstinputlisting[language=C]{c/helloworld.c}

Voor C heb je een compiler nodig om de broncode om te zetten in iets dat de computer begrijpt:
\begin{lstlisting}[language=bash]
$ gcc hello.c
$ ./a.out
\end{lstlisting}

