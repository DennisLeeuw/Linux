\index{4th-bit} De rechten r, w en x zijn elke keer blokjes van 3-bits. Het totaal is dus 3x3 is 9 bits lang. Dat is een raar getal in de computerwereld en dan ook niet helemaal correct, eigenlijk is het blok 12-bits lang en bestaat het uit 4x3 bits. De triplet 777 is dus eigenlijk een quadlet 7777. De eerste 7 wordt gebruikt om extra zaken in te coderen. De rechten op een bestand zonder extra functionaliteit is dus eigenlijke 0777, maar daar laten we de 0 meestal weg.

Met de extra rechten kunnen we de volgend functionaliteit weergeven:
\begin{itemize}
\item SUID bit
\item SGID bit
\item Sticky bit
\end{itemize}

