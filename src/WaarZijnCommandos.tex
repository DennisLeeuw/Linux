Toen we eerder de p typten en daarna twee keer op de tab-toets toen zagen we dat we 177 commando's hadden die met een p
begonnen. 177 commando's alleen al met een p dat is veel. Een gemiddeld Linux systeem bevat heel veel commando's en dat
is omdat er in de Unix-wereld twee filosofie\"en zijn die bij elkaar aansluiten de eerste is het \ KISS-principle. KISS
staat voor Keep It Simple, Stupid en is oorspronkelijk afkomstig uit de US Navy. En de tweede is Small is Beautiful en
die is afkomstig uit de Economie\footnote{Small Is Beautiful: A Study of Economics As If People Mattered door E. F.
Schumacher}

Op Linux systemen kom je vele kleine commando's tegen die \'e\'en ding goed doen. Dit heeft een aantal voordelen. Omdat
ze maar \ \'e\'en ding doen is de code simpel en is dus makkelijker te controleren op fouten. Omdat niet iedereen in
elk programma weer dezelfde code hoeft te herhalen maar gebruik kan maken van iemand anders zijn programma is de totale
hoeveelheid code klein, een compleet Linux systeem met webserver kan zonder grafische interface ge\"installeerd worden
op een 8G USB-stick. De laatste reden is dat wij als gebruikers vaak zonder te programmeren al heel complexe dingen met
Linux kunnen doen omdat er al zoveel tools al beschikbaar zijn.

Het nadeel van heel veel kleine programma's is dat je het overzicht snel kwijt kan raken. Een van de simpele programma's
is `ls', dat is een afkorting voor list. Veel commando's in Linux zijn afkortingen. De oorspronkelijke ontwikkeling van
Unix werd gedaan op systemen met toetsenborden die niet zo ergonomisch zijn als de onze. Ze hadden toetsen die je met
enige kracht moest indrukken en na een dag programmeren hadden de programmeurs Ken Thompson en Dennis Ritchie en hun
team vaak zere knokkels door overbelasting. Door commando's zo kort mogelijke namen te geven verminderden ze het aantal
toetsaanslagen. Vandaar de korte commando namen.

Type
\begin{lstlisting}[language=bash]
$ ls
\end{lstlisting}
en je zal een aantal blauwe directories op je scherm zien verschijnen. Als je nu
\begin{lstlisting}[language=bash]
$ ls /usr/bin/
\end{lstlisting}
typt dan verschijnen er allemaal groene commando's op je scherm, of beter ze scrollen van je scherm in twee kolommen,
omdat het er te veel zijn. Het past niet op je scherm. Als we de hele lijst willen zien dan zullen we gebruik moeten
maken van een programma de uitvoer van `ls' opdeelt in pagina's die zoveel regels bevatten dat ze het scherm vullen.
Een programma dat dat doet heet `more'. De kunst is nu om de uitvoer van `ls' te koppelen aan `more' en daarvoor is er
de pipe (|) of de pijp. Het pipe-character koppelt twee commando's aan elkaar:
\begin{lstlisting}[language=bash]
$ ls /usr/bin/ | more
\end{lstlisting}
met de spatie-balk kan je nu pagina voor pagina bekijken en met de letter q verlaat je `more'. Nu is more wel heel
simpel en kan het alleen dat wat je nu gezien hebt. Makkelijker zou het zijn als je omhoog en omlaag door de commando's
kan gaan, en misschien zelfs wel zou kunnen zoeken in zo'n lange lijst. Dat kan ook, daarvoor hebben we de opvolger van
`more' die meer kan en `less' heet want less is more.

\begin{lstlisting}[language=bash]
$ ls /usr/bin/ | less
\end{lstlisting}
Nu kan je met de spatie-balk door de pagina's gaan, met de pijltjes omhoog en omlaag per regel door de lijst gaan, met
PgUp en PgDn per pagina omhoog en omlaag gaan en met / kan je zoeken. Type maar eens als je in `less' zit
\begin{lstlisting}[language=bash]
/firefox
\end{lstlisting}
Zo kom je bij het commando `firefox' uit. Ook hier weer is de q-toets de manier om `less' te verlaten.

We hebben nu gezien dat heel veel commando's terug te vinden zijn in de /usr/bin/ directory. Maar dit is maar \'e\'en
plek waar commando's te vinden zijn. Commando's vind je terug in de (s)bin/ directories. We
gebruiken hier een meervoud omdat ze zich op verschillende plekken kunnen vinden. Je bent in de / directory al de bin
en sbin directories tegen gekomen. sbin is voor de systeembeheerder commando's en bin voor
commando's die ook door de normale gebruiker gebruikt kunnen worden.

Maar een Linux systeem zit iets complexer in elkaar. Je hebt waarschijnlijk ook al de /usr/
directory gezien. In die directory kom je een vergelijkbare structuur tegen als in /. Vaak wordt er gezegd dat
commando's in /bin/ en /sbin/ nodig zijn om het systeem op te starten. Alles ze niet direct nodig zijn voor het
opstarten dan vind je ze in /usr/bin/ en /usr/sbin/. Dat geldt ook voor de libraries in /lib/ en /usr/lib/.

Linux is een open source systeem, dus een systeembeheer kan ook bepalen dat bepaalde software
door hemzelf gecompileerd wordt omdat hij bijvoorbeeld een nieuwere versie beschikbaar wil maken dan door de
distributie geleverd wordt. Deze lokaal gecompileerde software wordt dan ge\"installeerd in /usr/local/. Dus
/usr/local/bin bevat commando's die afkomstig zijn van lokaal gecompileerde software en /usr/local/lib/ de libraries.

En als laatste is er nog de /opt/ directory. /opt/ is voor voorgecompileerde software die niet onderdeel is van de
distributie maar die bijvoorbeeld gecompileerd is door een commerci\"ele leverancier van software.

Om te bepalen welke directories gebruikt worden voor het zoeken naar een commando is er een variabele aanwezig in de
shell waarin we werken en die variabele heeft de logische naam PATH. Type maar eens:
\begin{lstlisting}[language=bash]
$ echo $PATH
\end{lstlisting}
Dit levert een output op die er ongeveer zo uit ziet: /usr/local/bin:/usr/bin:/bin. Voor een gebruiker met dit PATH
wordt er voor een commando eerst gezocht in /usr/local/bin, daarna in /usr/bin en als laatste in /bin/.

Als we willen weten waar een commando vandaan komt, dan kunnen we \texttt{which} gebruiken:
\begin{lstlisting}[language=bash]
$ which ls
\end{lstlisting}

Met het commando su kan je een andere gebruiker worden (switch user). Type eens:
\begin{lstlisting}[language=bash]
$ su -
\end{lstlisting}
geef het root wachtwoord en type
\begin{lstlisting}[language=bash]
# echo $PATH
# exit
\end{lstlisting}
Na het su commando moet je het password van de root gebruiker geven dat je ingesteld hebt tijdens
de installatie. Je zal nu zien dat het PATH van de root gebruiker ook de sbin/ directories bevat. De root gebruiker
heeft dus veel meer commando's tot zijn beschikking dan een normale gebruiker.
