Om bestanden naar een disk te kunnen schrijven moet er een soort van database bijgehouden worden met op welke track en sector bij het bestand horen, maar ook gegevens als van wie is het bestand en wie heeft er toegang tot het bestand. Al deze gegevens worden bijgehouden door het file system\index{file system}.

In de loop der tijd zijn er verschillende bestandssystemen bedacht voorbeelden zijn FAT, ExtFAT, FAT32, NTFS, ext3, ext4, HFS+ en nog vele anderen. Sommige besturingssystemen ondersteunen alleen hun eigen bestandssysteem, Linux ondersteunt er heel veel. Je kan in Linux vaak dus schijven van andere OSen lezen en schrijven.
