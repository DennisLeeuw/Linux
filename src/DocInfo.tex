The GNU-project heeft een eigen documentatie systeem ontworpen dat info\index{info} genoemd wordt. Het is een hypertext, dus met links, gerelateerd systeem. Dit is dus documentatie die je naast \texttt{man} tegen komt op systemen. Mocht je in de manual-pages niet vinden wat je zoekt, misschien kan je dan eens \texttt{info} proberen. Ook in info werken de pijltjes voor het scrollen en is de q-toets er weer om het \texttt{info} te verlaten. Een extra functie is de enter-toets die je kan gebruiken om een link te volgen. Een link herken je aan een onderstreepte tekst.

Het \texttt{info} systeem is niet op iedere Linux-distributie standaard ge\"installeerd, het kan dus gebeuren dat \texttt{info} op jouw systeem niet werkt.

