Naast software wordt er natuurlijk ook documentatie geschreven voor de software, worden er logo's gemaakt voor software en worden er websites
ontworpen, soms worden er ook filmpjes gemaakt. Kortom we willen vaak meer creatieve uitingen vangen in een
licentie dan alleen de software. In het begin werden de software licenties ook voor deze zaken gebruikt, maar dat bleek
niet altijd toereikend. De oplossing is uiteindelijk gevonden in een set van voorwaarden die bekend staan als de
Creative Commons.\par

Een Creative Commons\index{Creative Commons} (CC\index{Licentie!CC}) licentie zegt dat je werk hergebruikt mag worden. Met wat toevoegingen aan de licentie kan je zelf bepalen wat er wel niet met je werk
gedaan mag worden. De meest eenvoudige vorm is de CC BY. Deze zegt dat iedereen van alles met je werk mag doen, maar
dat ze daarbij altijd aan naamsvermelding moeten dien. Een beetje zoals de MIT licentie.\par

Zaken die je toe kan voegen zijn:

\begin{labeling}{ND}
\item [ND] staat voor dat er Geen Afgeleide werken gemaakt mogen worden. Een logo mag dus wel gebruikt worden maar mag niet gewijzigd worden. ND is een afkorting voor No Derivatives.
\item [NC] wat staat voor Niet Commercieel (Non Commercial). Het product of een afgeleide ervan mag niet commercieel gebruikt worden.
\item [SA] staat voor dezelfde licentie (Share Alike). Het product of een afgeleide daarvan mag dan niet van licentie veranderen. Dit lijkt erg op de GPL licentie.
\end{labeling}

Dit boek is uitgebracht onder de CC BY NC SA. Dat wil dus zeggen dat als er een gewijzigd werk gemaakt wordt van dit
document dan moeten daar de namen van de auteurs van dit document bij vermeld worden, mag het niet commercieel
uitgegeven worden en moet het onder dezelfde licentie verspreid worden. Je mag dus wel geld vragen voor het feit dat je een
boek bijvoorbeeld gedrukt hebt.

