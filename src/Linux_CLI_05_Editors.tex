\documentclass[a4paper,12pt,twoside,openright,titlepage]{article}

%Additional packages
\usepackage[utf8]{inputenc}
\usepackage[T1]{fontenc}
\usepackage[dutch,english]{babel}
\usepackage{syntonly}
\usepackage[official]{eurosym}

% Handle images
%\usepackage[graphicx]
\usepackage{graphicx}
\graphicspath{ {./images/}{./styles/} }
\usepackage{float}
\usepackage{wrapfig}

% Handle URLs
\usepackage{xurl}
\usepackage{hyperref}
\hypersetup{colorlinks=true, linkcolor=blue, citecolor=blue, filecolor=blue, urlcolor=blue, pdftitle=, pdfauthor=, pdfsubject=, pdfkeywords=}

% Tables and listings
\usepackage{tabularx}
\usepackage{scrextend}
\addtokomafont{labelinglabel}{\sffamily}
\usepackage{listings}
\usepackage{adjustbox}

% Turn on indexing
\usepackage{imakeidx}
\makeindex[intoc]

% Define colors
\usepackage{color}
\definecolor{ashgrey}{rgb}{0.7, 0.75, 0.71}



% Listing style
\input{styles/lstset}

% Uncomment for production
% \syntaxonly

% Style
\pagestyle{headings}

% Turn on indexing
\makeindex[intoc]

% Define document
\author{D. Leeuw}
\title{Linux: Editors en Vim}
\date{\today\\
1.0.0\\
\vfill
\raggedright
\copyright\ 2020-2025 Dennis Leeuw\\
\input{styles/licentie-titlepage}}


\begin{document}
\selectlanguage{dutch}

\maketitle

%%%%%%%%%%%%%%%%%%%
%%% Introductie %%%
%%%%%%%%%%%%%%%%%%%

\section{Over dit Document}
\subsection{Leerdoelen}
Na het bestuderen van dit document heeft de lezer kennis van:
\begin{itemize}
\item wat een editor is
\item de commando's: vi, nano, pico, vim
\item de basis functionaliteit van vim: tekst invoer, knippen, plakken
\end{itemize}

Dit document sluit aan op de volgende onderdelen van de LPI:
\begin{itemize}
\item LPI Linux Essentials 010-160 - 6.3.3 Turning Commands into a Script (weight: 4)
\end{itemize}


\subsection{Voorkennis}
Voor een goed begrip van dit document is het van belang dat de lezer kennis heeft van:
\begin{itemize}
\item de shell
\end{itemize}



%%%%%%%%%%%%%%%%%
%%% De inhoud %%%
%%%%%%%%%%%%%%%%%

% Requires: fhs
% Provides: vi, vim, nano, pico
\section{Het gebruik van een editor}
Op de commandline heb je geen menu's en vaak geen muis om door een applicatie te navigeren. Het maken van documenten is dan ook een stuk lastiger dan in een grafische interface. Toch zijn er oplossingen om op de commandline te werken met bestanden. Een tekstverwerker\index{tekstverwerker} op de commandline heet een editor\index{editor}. Er zijn verschillende editors bedacht en in gebruik. Een van de oudste voor Unix geschreven editors is \texttt{vi}.

Een van de grote namen achter Unix is Ken Thompson. De eerste drie commando's die hij schreef voor het jonge Unix systeem waren \texttt{as} (assembler), \texttt{ed}\index{ed} (editor) en \texttt{sh} (shell). Dennis M. Ritchie bracht verbeteringen aan op \texttt{ed} en vanaf 1969 tot 1976 bleef dit de editor op een Unix systeem. In 1976 kwamen Billy Joy en Chuck Haley met een nieuwe editor die \texttt{ex}\index{ex} werd genoemd. Voor \texttt{ex} schreef Billy Joy ook een soort interface om er makkelijker mee te kunnen werken en die wrapper om \texttt{ex} noemde hij \texttt{vi}\index{vi} (visual interface\index{visual interface}). Vanaf 1979 werd \texttt{ex} geintergreerd in \texttt{vi} en was er alleen nog \texttt{vi}. Later werd \texttt{vi} onderdeel van de Single Unix Specification en daarmee een editor die op bijna elk Unix systeem aanwezig is en dat is nogsteeds het geval. Op bijna alle beschikbare Unix systemen, van BSD tot Linux en Mac OS X is een vorm van \texttt{vi} aanwezig. Dat is dan ook het voordeel van het aanleren van het werken met \texttt{vi}, ja kan je kennis op verschillende platformen gebruiken.

\subsection{vi, pico, nano}
De eerste redelijk gebruiksvriendelijke editor op Unix was \texttt{vi}. De \texttt{vi} editor kent twee modi. De eerste modus is de \textquote{edit mode} en de tweede is de \textquote{command mode}. Standaard start \texttt{vi} op in de command mode waarin je commando's kunt geven om bestanden te laden of op te slaan en waarin je functies als knippen en plakken kan uitvoeren. De edit modus is die waarin je tekst invoert. Dit onderscheid maakt het voor beginnende gebruikers \texttt{vi} soms verwarrend.

Naast \texttt{vi} zijn er ook andere editors voor Unix-achtige systemen ontwikkeld. De meeste bekende zijn \texttt{pico}\index{pico} en \texttt{nano}\index{nano}. Pico was de oorspronkelijke editor. Nano is ontwikkled door het GNU-project en is een vervanging van pico omdat pico een licentie had die \textquote{problematisch} was. Dat probleem is inmiddels opgelost, maar nano biedt zoveel extra mogelijkheden dat velen de voorkeur geven aan nano.

Het grote voordeel van nano ten opzichte van vi is zijn gebruiksvriendelijke interface. Nano kent geen edit en command mode zoals vi. Nano gebruikt control codes om commando's te geven en is direct beschikbaar voor de invoer van tekst van de gebruiker.

\section{vim}
De naam \texttt{vim}\index{vim} staat voor Vi IMproved\index{Vi IMproved}. Of wel een verbeterde versie van \texttt{vi}. Bram Molenaar, een Nederlandse software ontwikkelaar, schrijft al sinds 1991 aan de code van \texttt{vim} en zijn verbeteringen zijn zo populair dat op alle Linux systemen alleen nog \texttt{vim} ge\"installeerd wordt. Je kan op een Linux systeem \texttt{vim} opstarten als \texttt{vi} waarmee je zoveel mogelijk de functionaliteiten krijgt als het oude \texttt{vi} en je kan \texttt{vim} opstarten als \texttt{vim} waarmee je alle nieuwe toevoegingen van Bram Molenaar en zijn medeontwikkelaars krijgt.

\texttt{vim} is niet een van de makkelijkste editors maar heeft als grote voordeel, zoals eerder gezegd, dat het beschikbaar is op elk willekeurig Unix systeem.

\subsection{vim opstarten}
Het kan zijn dat \texttt{vim} nog niet geinstalleerd is. Mocht dat het geval zijn, installeer dan \texttt{vim} via de packagemanager voor jouw systeem. Voor Debian systemen is dat:\index{vim!installeren}

\begin{lstlisting}[language=bash]
$ sudo apt-get install vim
\end{lstlisting}

De meeste gebruikelijke manier om \texttt{vim} op te starten is door aan \texttt{vim} meteen een bestandsnaam mee te geven:\index{vim!starten met bestandsnaam}
\begin{lstlisting}[language=bash]
$ vim bestand.txt
\end{lstlisting}
Als je klaar bent met het toevoegen van tekst kan je met \textbf{:wq} afsluiten\index{vim!opslaan en afsluiten}. Dit slaat het document op (w)\index{vim!opslaan} en sluit af (q). Wil je de editor verlaten zonder de gemaakte wijzigingen op te slaan, dan gebruikt je \textbf{:q!}\index{vim!afsluiten zonder opslaan}. Het doet een quit (q) zonder verdere vragen stellen.

Een andere manier om vim op te starten is door geen bestandsnaam mee te geven:\index{vim!starten zonder bestandsnaam}
\begin{lstlisting}[language=bash]
$ vim
\end{lstlisting}
de editor weet nu niet onder welke bestandsnaam een bestand opgeslagen moet worden. Bij de write (w) moet je nu dus de bestandsnaam meegeven: \textbf{:w bestand.txt}\index{vim!bestand schrijven} slaat het bestand dat je gemaakt hebt op als bestand.txt. Na deze opdracht kan je met \textbf{:q} vim afsluiten.

\subsection{Text invoer}
Om tekst toe te voegen of te wijzigigen in \texttt{vi} moet je vanuit de command mode\index{command mode} naar de edit mode\index{edit mode} gaan. Hiervoor zijn verschillende commando's beschikbaar. De meest gebruikte zijn \textbf{i} van insert\index{vim!insert} of \textbf{a} van add\index{vim!add}. Om de edit mode\index{edit mode!verlaten} te verlaten gebruik je de \textsc{ESC} toets.

Met het gebruik van het \textbf{i} commando voeg je tekst in v\'o\'or de plek van de cursor. Door gebruik te maken van \textbf{a} voeg je tekst in na de positie van de cursor.

\subsection{Text delete}
\input{src/vimDelete}
\subsection{Knippen en plakken}
Met het \textbf{x} commando kan je \'e\'en enkel character knippen.

De traditionele manier om een kopie van een stuk tekst te maken is het gebruik van het \textbf{y} commando\index{vim!copy}. Het \textbf{yl} commando kopieert een character, \textbf{yw} kopieert een woord en \textbf{yy} kopieert regels.

Het \textbf{p}\index{vim!paste} commando kan gebruikt worden om text te plakken. Het \textbf{p} commando is plakken achter de positie van de cursor en \textbf{P} is plakken voor de positie van de cursor.

Een speciale functie van vim en niet van vi is het gebruik van \textbf{v} om een visuele selectie\index{vim!visual selection} maken, met de pijltjes toetsen kan je nu bepalen hoe groot de selectie worden moet. Het commando \textbf{v} geeft je de mogelijkheid om de selectie op character niveau te maken. Met \textbf{V} maak je selecties per regel, hier gebruik je de omhoog en omlaag pijltjes toetsen om je selectie groter of kleiner te maken. Als je de selectie gemaakt hebt gebruik je \textbf{y} om te kopie\"eren of \textbf{d} om het stuk te verwijderen.

Met \textbf{u} kan je een wijziging ongedaan maken.


\subsection{Bewegen door een tekstbestand}
\input{src/vimBewegen}

%%%%%%%%%%%%%%%%%%%%%
%%% Index and End %%%
%%%%%%%%%%%%%%%%%%%%%
\printindex
\end{document}

%%% Last line %%%
