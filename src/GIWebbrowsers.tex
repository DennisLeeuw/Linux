Een populaire browser is Mozilla \index{Firefox}Firefox welke dan ook door veel distributies standaard meegeleverd wordt. Firefox is de doorontwikkelde browser van Netscape toen die in 1998 open source werd. Eerst hete de gehele suite Mozilla\index{Mozilla}. De library die voor alle HTML/CSS afhandeling zorgt is Gecko. Het Mozilla project heeft inmiddels vele software producten opgeleverd waarvan de belangrijkste de webbrowser Firefox en de e-mail client Thunderbird.

Het KDE-project heeft zijn eigen webbrowser: Konquerer. De KDE browser bestaat uit een engine en een interface. De engine is de library die alle benodigde functies voor het afhandelen van webpagina's heeft. Die engine is ooit begonnen als KHTML, maar hij heet nu \index{WebKit}WebKit en wordt inmiddels ook door Apple gebruikt voor zijn Safari browser.

Toen Google zijn eigen webbrowser ontwikkelde werd de basis hiervan vrij gegeven als open source
browser met de naam \index{Chromium}Chromium. Google gebruikt chromium als basis voor zijn Chrome browser en Microsoft gebruikt het als basis voor Edge. De open source versie is op de meeste Linux systemen te installeren via de standaard package managers zoals het eerder beschreven Software.
