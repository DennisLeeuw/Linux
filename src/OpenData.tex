In navolging van open source en open standaarden kwam er ook steeds meer de vraag op hoe dat zit met data. Kan data ook open zijn?

Is onderzoeksdata van een universiteit van de universiteit, van de onderzoeker of van de overheid (of instantie) die het onderzoek betaald heeft? En als die data gedeeld wordt met de wereld wat mag je er dan mee doen. Mag je de data van een onderzoek wijzigen?

Al dit soort zaken zijn het domein van de open data. Het is dus belangrijk dat bij data ook een keuze gemaakt wordt wat er wel en niet mee mag gebeuren.
