Om data op een computer te structureren is het handig om de data te verdelen over directories\index{Directories}. Directories zijn ook bekend als mappen\index{Mappen} en folders\index{Folders}. Wij zullen alleen nog spreken van directories omdat binnen de Unix-wereld de meest gebruikte term is.

Een directory maak je aan met het commando \texttt{mkdir}\index{mkdir}\index{Directories!mkdir}:
\begin{lstlisting}[language=bash]
$ mkdir LinuxCursus
\end{lstlisting}
Met \texttt{ls} kan je controleren of de directory ook daadwerkelijk aangemaakt is.

Je kan ook meerdere directories tegelijk aanmaken door ze als een lijst op te geven, gescheiden door spaties:
\begin{lstlisting}[language=bash]
$ mkdir Aap Noot Mies
\end{lstlisting}

Soms wil je ook een heel pad gelijk aanmaken met:
\begin{lstlisting}[language=bash]
$ mkdir Boom/Roos/Vis/Vuur
\end{lstlisting}
gaat dat niet lukken, want de Boom directory bestaat niet. Gelukkig kan je aan \texttt{mkdir} en optie meegeven die vertelt dat \texttt{mkdir} ook alle onderliggende directories moet aanmaken:
\begin{lstlisting}[language=bash]
$ mkdir -p Boom/Roos/Vis/Vuur
\end{lstlisting}
Gebruik \texttt{ls} om te controleren dat alle directories aanwezig zijn.

Tot slot wil je ook instaat zijn om directories weg te gooien. Met \texttt{rmdir}\index{rmdir}\index{Directories!rmdir} kan dir als de directories leeg zijn.
\begin{lstlisting}[language=bash]
$ rmdir Aap Noot Mies
\end{lstlisting}
gooit keurig alle aangemaakte directies weg. Controleer dit met \texttt{ls}. Maar doen we:
\begin{lstlisting}[language=bash]
$ rmdir Boom
\end{lstlisting}
dan krijgen we een error melding, want de \texttt{Boom} directory is niet leeg. We zullen dus eerst alle andere directories moeten weggooien, te beginnen met \texttt{Vuur}, dan \texttt{Vis}, dan \texttt{Roos} en tot slot kunnen we pas \texttt{Boom} weggooien. Gebruik \texttt{rmdir} om alle directories \textbf{behalve} Boom te verwijderen. De \texttt{Boom} directory moet dus blijven bestaan. 
