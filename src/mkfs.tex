Om een disk voor te bereiden op het ontvangen van data moet de disk eerst voorzien worden van de nodige structuren die uniek zijn voor het specifieke bestandssysteem. Het aanbrengen van deze structuren heet het formateren\index{formateren} van de disk. Voor elke bestandssysteem heb je dan ook een eigen format tool.

De format tool op een Linux systeem heet \texttt{mkfs}\index{mkfs}\index{commando!mkfs} (make filesystem). Voor de verschillende file systems is er een eigen tool, die begint met mkfs, dan een punt en dan de naam van het bestandssysteem zoals Linux het kent. Voorbeelden: \texttt{mkfs.exfat}, \texttt{mkfs.ext3}, \texttt{mkfs.fat}, \texttt{mkfs.msdos}, \texttt{mkfs.vfat}.

