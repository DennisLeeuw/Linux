Elk besturingssysteem dat meer dan \'e\'en gebruiker kent heeft een manier nodig om ervoor te zorgen dat twee gebruikers niet bij elkaars bestanden kunnen als ze dat niet willen. Ook moeten gebruikers niet bij de bestanden van de beheerder (root) kunnen komen. Er moet dus door het systeem bijgehouden worden wie welke rechten heeft.

Bij het ontwerp van Unix was de centrale gedachte dat alles binnen het systeem gerepresenteerd werd door het idee van een bestand: 'Everything is a file'. Dus niet alleen documenten zijn bestanden, maar ook aangesloten printers, harddisks, etc. Omdat alles een bestand is kan je met rechten op bestanden ook bepalen wie toegang heeft tot bepaalde stukken hardware.

