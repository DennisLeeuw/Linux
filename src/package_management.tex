De broncode wordt door software ontwikkelaars van Open Source software op het Internet gezet. De broncode kan je niet draaien op een machine, die moet eerst vertaald worden naar een binary. Bij Windows en Mac OS X zijn de applicaties die je op het Internet vindt al in binaire vorm. Dit is een groot verschil tussen Open Source software en software voor commerci\"ele besturingssystemen.

Het is bijna ondoenlijk om alle software zelf te compilen en daarom zijn er Linux distributies die dat voor je gedaan hebben. Elke distributie maakt daarin keuzes met wat er standaard al ge\"installeerd wordt en wat er later nog extra kan worden ge\"installeerd. De software die extra ge\"installeerd kan worden wordt aangeboden in repositories.

Naast de binaire applicatie die ge\"installeerd kan worden zit er bij software vaak ook een handleiding (manual-page), een configuratie bestand en een eventueel een opstart scripts als de software als achtergrond proces kan draaien. Ook moeten er misschien extra gebruikers aangemaakt worden, moet de software toegang hebben tot een database of logs en misschien moeten er nog wel veel meer administratieve handelingen gedaan worden. Je wil dit natuurlijk niet allemaal met de hand doen, dus veel van deze zaken zijn geautomatiseerd. Om dit te kunnen doen wordt de software via de repository geleverd als pakket (packet). Verschillende distributies hebben verschillende manieren om deze pakketten te maken.

De verschillende pakketvormen en repositories is het het onderwerp van dit hoofdstuk.

