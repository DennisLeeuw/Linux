We kunnen programeertalen grofweg opdelen in twee soorten:
\begin{itemize}
\item Procedural Programming
\item Object Oriented Programming (OOP)\index{OOP}
\end{itemize}

Een Procedural\index{Procedural Programming} programming language is gebaseerd op het gebruik van procedure aanroepen (procedure calls). Een Procedure is een routine of subroutine, misschien beter bekent als functies. Een functie of routine is een blok met commando's die bij elkaar horen en die vanuit het hoofdprogramma \'e\'en of meer keren aangeroepen kan worden.

Een Object Oriented\index{Object Oriented Programming} language is een taal die is gebaseerd op Objecten. Een object is programma-code met data. Waar een functie alleen de programma code bevat bevat een object ook de data. De code bestaat net als bij Procedural languages uit procedures, maar heten dan methods en de data kan aangesproken worden als attributes of fields (velden). Er is gebleken dat veel vertalingen van functies in de wereld die we willen automatiseren zich makkelijker laten vertalen in objecten.
