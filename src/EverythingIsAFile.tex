In Unix en Unix-like operating systems zoals Linux is het basis principe dat alles een bestand (file) is. Dit betekent dat alles binnen het systeem; documenten, directories, harddisks, printers, toetsenborden, maar ook processen weergegeven worden als bestanden. Het voordeel hiervan is dat dezelfde commando's en API's gebruikt kunnen worden voor verschillende onderdelen van het besturingssysteem.

Het filesysteem is een enkele boom van bestanden en directories, zonder onderscheid tussen disks en partities. Zelfs verplaatsbare media zoals USB-sticks en DVDs zijn onderdeel van deze boom als ze 'gemount' zijn. Ook processen (/proc) en de kernel (/sys) is voor een groot deel benaderbaar via het bestandssysteem.

Met \texttt{ls} kan je in bijvoorbeeld \texttt{/proc} kijken en zien welke processen er zijn. Je ziet er nummer staan en die nummers komen overeen met de process nummers die ook \texttt{ps} heeft. Doe maar eens:
\begin{lstlisting}[language=bash]
$ ps aux
\end{lstlisting}
en je ziet in de tweede kolom dezelfde nummers staan. \texttt{ps} is dan ook het commando om te zien welke processen er op je systeem actief zijn. We zullen ps in een volgend hoofdstuk nog uitgebreid behandelen.
