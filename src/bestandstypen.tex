\begin{center}
\begin{tabular}{ | c | c | c | }
\hline
Bestandstype & Beschrijving & mode veld \\
\hline
\hline
 normaal bestand & Documenten, etc. & - \\
\hline
 directory & Directories bevatten geen bestand, maar een overzicht van de bestandsnamen waaraan gekoppeld referenties naar iets wat inodes worden genoemd. De inodes bevatten de daadwerkelijke bestanden en meta-data (eigenaar, groep, permissies, time stamps, etc.). Door deze manier van werken kan een bestand (met meta-data) verschillende namen hebben (hard-link), mits binnen \'e\'en bestandssysteem (partitie of disk). Bij verschillende bestandsnamen kunnen dezelfde inodes vermeldt staan. & d \\
\hline
 Symbolic link & een link naar een bestand die over bestandssystemen heen kan gaan & l \\
\hline
Block device & Een apparaat waar van of waar naar toe data in een random manier gestuurd kan worden. Het hoeft dus niet in de juiste volgorde te zijn. Denk aan harddisks waar eerst sector 2014 en dan sector 5678 geschreven kan worden. & b \\
\hline
Character device & Een apparaat waar data in een stroom van characters naar of naar toe gestuurd kan worden & c \\
\hline
FIFO & Ook bekend als named pipes. Een pipe verbindt het ene proces met het andere proces zodat data van proces 1 naar proces 2 gestuurd kan worden. Dit kan maar \'e\'en kant op. & p \\
\hline
Socket & Verbindt net als FIFO's processen, maar dan op een manier dat er twee weg communicatie mogelijk is. & s \\
\hline
\end{tabular}
\end{center}

