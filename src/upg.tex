User Personal Group\index{User Personal Group} of UPG\index{UPG} is een gecombineerd gebruik van de user-ID en de group-ID voor het opslaan van data. Er is bij gebruik van UPG ook een groepsnaam met de gebruikersnaam.

In de voorgaande voorbeelden zag je het gebruik van UPG. De gebruiker dennis heeft een UID van 1000 en een GID van 1000 en beide zijn gekoppeld aan de naam dennis. Dus de groepsnaam is gelijk aan de gebruikers naam (de nummerieke ID's hoeven niet gelijk te zijn, maar zijn dat vaak wel). De gebruiker is dus niet lid van de groep \textbf{users} waar op sommige andere systemen elke gebruiker lid van is.

Door gebruik te maken van een groep \textbf{users} waar elke gebruiker lid van is maak je het delen van informatie met andere gebruikers makkelijker omdat iedereen een gedeelde groep heeft. Data die toegankelijk is voor de groep \textbf{users} is dus toegankelijk voor iedereen.

UPG is meer gericht op veiligheid. Omdat elke gebruiker alleen in zijn eigen groep zit kan er standaard geen data gedeeld worden. Data kan alleen gedeeld worden als er een groep aangemaakt wordt waaraan gebruikers worden toegevoegd. Die groep kan dan data met elkaar delen. Verder is alle data die je maakt alleen toegankelijk voor jezelf.
