Het Sticky-bit\index{Sticky-bit} kan gezet worden met de decimale waarde 1 of met \texttt{o+t}. Voorbeelden:
\begin{lstlisting}[language=bash]
$ mkdir sticky.d
$ touch sticky.txt
$ chmod o+t sticky.d
$ chmod 1777 sticky.txt
$ ls -dl sticky*
drwxr-xr-t 2 dennis dennis 4096 Jan 24 10:04 sticky.d
-rwxrwxrwt 1 dennis dennis    0 Jan 24 10:04 sticky.txt
\end{lstlisting}
Bij het rechtenblok van other is de x vervangen door een t.

Voor bestanden heeft het Sticky-bit in Linux geen betekenis.

Als het Sticky-bit gezet is op een directory dan betekent dat alleen de eigenaar, de eigenaar van de directory en root het bestand kunnen weggooien en hernoemen. Zelfs als iemand in de juiste groep zit en de groep heeft schrijfrechten dan nog heeft die persoon niet de rechten. Een voorbeeld van het gebruik van het Sticky-bit is het \texttt{/tmp} directory.

