Een ander handigheidje van de shell is dat hij een geschiedenis (history) opslaat van gebruikte commando's. Met de
pijltjes toetsen $\uparrow $ (pijltje-omhoog) en $\downarrow $ (pijltje-omlaag) kan je door de geschiedenis van je
commando's scrollen. Omhoog is terug in de tijd en omlaag is het omgekeerde. Met CTRL-c breek je af waar je gebleven
bent en met enter voer je het commando uit.

Met CTRL-r kan je zoeken in je history. Type maar eens CTRL-r en dan de p. Er zal vrijwel direct pwd verschijnen. Op
enter drukken is dan het enige dat nog nodig is om het commando uit te voeren. Als je toch besluit dat je het commando
niet wil uitvoeren dan druk je op CTRL-c om de zoekfunctie te verlaten.

Met !! herhaal je het laatste commando dat je gedaan hebt en met !<commando> herhaal je het
laatste <commando> dat je gedaan hebt. Type nu eens

\begin{lstlisting}[language=bash]
$ !!
\end{lstlisting}

dan zal het pwd commando opnieuw uitgevoerd worden.

\begin{lstlisting}[language=bash]
$ !hostname
\end{lstlisting}
zal de hostname -s uitvoeren want dat is het laatste hostname commando dat we gedaan hebben.
