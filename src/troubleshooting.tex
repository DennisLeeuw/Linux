De basis van problemen zoeken\index{troubleshooting}\index{probleem zoeken} op een Linux systeem zijn in de voorgaande hoofdstukken behandeld. Vooral de log bestanden zijn altijd een rijke bron van informatie. Soms is de error melding die je krijgt niet helemaal duidelijk, het is dan handig om de melding letterlijk te knippen en plakken in een Internet zoekmachine. Linux en bijna alle programma's die erop draaien zijn open source en het beheer en troubleshooten van software uit de open source wereld wordt op Internet gedaan, daarom zijn veel van de problemen dan ook op het Internet te vinden, vaak met de bij behorende oplossing. Er zijn vele sites met vele forums en soms worden vragen niet in het juiste forum gesteld. Het kan dus zijn dat de eerste hit van de zoekmachine niet de juiste is en dat het antwoord niet tot een juiste oplossing leidt. Zoek dan nog even verder. Het kan ook gebeuren dat een bepaalde oplossing voor een bepaalde versie werkt, maar niet voor een nieuwere versie. Let dus ook altijd goed op de datum dat bij een antwoord staat, nieuwer is vaak beter.

Tikfouten in configuratie bestanden zijn waarschijnlijk de meeste voorkomende beginnersfouten op Linux. Het gebruik van enkele-quotes of dubbele-quotes, het vergeten van een ; of het perongeluk vervangen door een : zijn ook bekende voorbeelden van fouten. Zeker het blind knippen en plakken van code of commando's kan leiden tot bijzondere effecten, zeker als er niet op het font gelet wordt.

Maar misschien wel de belangrijkste vaardigheid bij het zoeken van problemen is het goed lezen en de tijd nemen om te begrijpen wat er in een foutmelding staat. Linux en open source in zijn algemeenheid wordt geschreven door mensen. Het zijn mensen die wereldwijd samenwerken ieder met zijn eigen culturele en eigen taal achtergrond. Dat vertaalt zich ook in de manier waarop fouten gemeld worden aan de gebruiker. Een error-melding is vervelend en soms moeilijk te lezen, maar erger zou het zijn als er geen melding werd gegeven.

