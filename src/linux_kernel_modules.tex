De Linux kernel is modulair opgebouwd. Dit betekent dat de kernel na het opstarten met extra functionaliteit uitgebreid kan worden door extra modules te laden.

De modules die bij de kernel horen kunnen gevonden worden in de directories onder \texttt{/lib/modules/}\index{/lib/modules/}. De modules die behoren bij de draaiende kernel kunnen gevonden worden via:
\begin{lstlisting}[language=bash]
$ ls /lib/modules/`uname -r`/kernel/
\end{lstlisting}

Modules kunnen on-the-fly geladen worden. Een overzicht van de reeds geladen modules kan verkregen worden met \index{lsmod}:
\begin{lstlisting}[language=bash]
$ lsmod
\end{lstlisting}

Om te zorgen dat onze kernel kan omgaan met het FAT bestandssysteem gaan we de fat-modules toevoegen aan de draaiende kernel\index{modprobe}. Dit mag natuurlijk alleen root doen.
\begin{lstlisting}[language=bash]
$ sudo modprobe fat
\end{lstlisting}
Bij geen error-melding is alles goed gegaan. Dit kunnen we controleren
\begin{lstlisting}[language=bash]
$ lsmod | grep fat
\end{lstlisting}
Na deze handeling zouden we bijvoorbeeld een USB-stick met een FAT bestandssysteem kunnen lezen.

Het verwijderen van een module is zo simpel als aan \texttt{modprobe} vertellen met de \texttt{-r} optie dat het een module moet verwijderen
\begin{lstlisting}[language=bash]
$ sudo modprobe -r fat
\end{lstlisting}

