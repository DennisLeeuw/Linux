Het SGID-bit\index{SGID-bit} is het tweede bit uit de reeks en is dus decimaal: 2. Het behoort bij de groepsrechten en kan dus gezet worden met \texttt{g+s}. Voorbeelden
\begin{lstlisting}[language=bash]
$ mkdir SGID.d
$ touch SGID.txt
$ chmod g+s SGID.d
$ chmod 2777 SGID.txt
$ ls -ld SGID*
drwxr-sr-x 2 dennis dennis 4096 Jan 24 09:28 SGID.d
-rwxrwsrwx 1 dennis dennis    0 Jan 24 09:28 SGID.txt
\end{lstlisting}

Het SGID-bit op een bestand zet de effectieve groep waaronder een applicatie draait. Dus net als wat de SUID-bit doet voor de gebruiker doet het SGID-bit voor de groep,

Het SGID-bit op directories betekent dat de groep eigenaarschap wordt doorgegeven aan nieuwe bestanden of directories die aangemaakt worden in de directory. Dus als we een directory hebben met het SGID-bit en de groepseigenaar van de directory is \textbf{samen} en we maken in die directory een nieuwe bestand aan dan wordt de groepseigenaar weer \textbf{samen} ongeacht de groep waarin de gebruiker zit die het bestand aanmaakt. Natuurlijk moet een van de groepen van die gebruiker wel \textbf{samen} zijn. 

