We kunnen ook regels zoeken in een bestand die aan een bepaald patroon (regex) voldoen en dan het patroon vervangen door iets anders. Een van de meest gebruikte tools daarvoor heet \texttt{sed}\index{sed}.

\begin{lstlisting}[language=bash]
$ ls -1
$ ls | sed -e 's/aap/noot/'
\end{lstlisting}
overal waar aap stond staat nu noot. Op de disk is er niets gewijzigd, we hebben alleen de output van ls aangepast. We hebben de \textquote{taak} voor \texttt{sed} tussen ticks gezet omdat we zeker willen weten dat deze opdracht bij \texttt{sed} terecht komt en dat niet de shell zich ermee gaat bemoeien.

Om iets meer van \texttt{sed} te leren gaan we eerst een bestand aanmaken. Gebruik hiervoor \texttt{vi}, \texttt{vim} of een andere editor.
\lstinputlisting[language=bash]{data/stageverslag.txt}

In deze tekst zitten wat taalfouten. Die gaan we zoeken en vervangen. We beginnen met het gebruik van Me aan het begin van de eerste twee alinea's. Dat moet natuurlijk Mijn zijn.
\begin{lstlisting}[language=bash]
$ sed -e 's/^Me/Mijn/'
\end{lstlisting}
Op het scherm zien we de verbeterde tekst, op disk staat echter nog de oude tekst. We kunnen de verbeterde tekst natuurlijk met het groter dan teken wegschrijven naar disk met een nieuwe bestandsnaam. Makkelijker is het om de verbeterde tekst in het al bestaande bestand aan te passen:
\begin{lstlisting}[language=bash]
$ sed -ie 's/^Me/Mijn/'
\end{lstlisting}
We zien nu geen output, omdat de tekst op de disk gewijzigd is. Gebruik \texttt{cat} om te zien dat de tekst inderdaad gewijzigd is. Zoek in de man-page op wat de betekenis is van de -i en de -e opties.

Een ander 'me' die fout is is de 'me' voor 'me school' maar de 'me' voor 'me ontwikkelen' is bijvoorbeeld goed. We moeten dus een regular expression schrijven ide alleen matched op 'me school':
\begin{lstlisting}[language=bash]
$ sed -ie 's/me\ school.$/mijn\ school/'
\end{lstlisting}
Let op de $\backslash$ voor de spatie. Een spatie is in de shell een scheidingsteken en dat willen we nu niet. We bedoelen een echte spatie, dus moeten we hem escapen.
