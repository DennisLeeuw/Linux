De MIT (Massachusetts Institute of Technology) licentie geven we hier in zijn geheel weer omdat hij de basis vormt voor veel van de erop volgende licenties.

\bigskip

{\selectlanguage{english}
Copyright {\textless}YEAR{\textgreater} {\textless}COPYRIGHT HOLDER{\textgreater}}

{\selectlanguage{english}
Permission is hereby granted, free of charge, to any person obtaining a copy of this software and associated
documentation files (the {\textquotedbl}Software{\textquotedbl}), to deal in the Software without restriction,
including without limitation the rights to use, copy, modify, merge, publish, distribute, sublicense, and/or sell
copies of the Software, and to permit persons to whom the Software is furnished to do so, subject to the following
conditions:}

{\selectlanguage{english}
The above copyright notice and this permission notice shall be included in all copies or substantial portions of the
Software.}

{\selectlanguage{english}
THE SOFTWARE IS PROVIDED {\textquotedbl}AS IS{\textquotedbl}, WITHOUT WARRANTY OF ANY KIND, EXPRESS OR IMPLIED,
INCLUDING BUT NOT LIMITED TO THE WARRANTIES OF MERCHANTABILITY, FITNESS FOR A PARTICULAR PURPOSE AND NONINFRINGEMENT.
IN NO EVENT SHALL THE AUTHORS OR COPYRIGHT HOLDERS BE LIABLE FOR ANY CLAIM, DAMAGES OR OTHER LIABILITY, WHETHER IN AN
ACTION OF CONTRACT, TORT OR OTHERWISE, ARISING FROM, OUT OF OR IN CONNECTION WITH THE SOFTWARE OR THE USE OR OTHER
DEALINGS IN THE SOFTWARE.}

\bigskip

Eigenlijk zegt deze licentie dat je alles met de software mag doen, behalve het weghalen van de copyright en de
licentievoorwaarden. Ook is er een duidelijk statement dat de maker niet verantwoordelijk is voor wat de software doet. Deze laatste kom je
in bijna elke licentie tegen. Ook in de EULA van Microsoft staat dit voorbehoud.

