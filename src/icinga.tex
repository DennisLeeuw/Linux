Voordat we beginnen met de installatie moeten we eerst een paar variabelen een waarde geven. Wijzig deze variabelen naar eigen behoefte. De FQDN is de verwijzing naar de host waarop Icinga draait en wordt gebruikt als hostname in de virtual host configuratie van Apache.
\begin{lstlisting}[language=bash]
FQDN="icinga.example.com"
EMAIL="me@example.com"
ICINGAWEB2_DB_PASSWORD="changeme"
ICINGA_WEB_ADMIN_USER="admin"
ICINGA_WEB_ADMIN_PWD="changeme"
\end{lstlisting}


\section{Apache}

\subsection{Installatie}
\begin{lstlisting}[language=bash]
DEBIAN_FRONTEND=noninteractive apt-get -y install \
apache2
\end{lstlisting}

\subsection{Configuratie}
\begin{lstlisting}[language=bash]
cat << EOF > /etc/apache2/sites-available/icinga.conf
<VirtualHost *:80>
    ServerName ${FQDN}
    ServerAdmin ${EMAIL}

    DocumentRoot "/usr/share/icingaweb2/public"

    <Directory "/usr/share/icingaweb2/public">
        Options SymLinksIfOwnerMatch
        AllowOverride None

        <IfModule mod_authz_core.c>
            # Apache 2.4
            <RequireAll>
                Require all granted
            </RequireAll>
        </IfModule>

        SetEnv ICINGAWEB_CONFIGDIR "/etc/icingaweb2"

        EnableSendfile Off

        <IfModule mod_rewrite.c>
            RewriteEngine on
            # RewriteBase /icingaweb2/
            RewriteBase /
            RewriteCond %{REQUEST_FILENAME} -s [OR]
            RewriteCond %{REQUEST_FILENAME} -l [OR]
            RewriteCond %{REQUEST_FILENAME} -d
            RewriteRule ^.*$ - [NC,L]
            RewriteRule ^.*$ index.php [NC,L]
        </IfModule>

        <IfModule !mod_rewrite.c>
            DirectoryIndex error_norewrite.html
            ErrorDocument 404 /error_norewrite.html
        </IfModule>
    </Directory>

    ErrorLog  /var/log/apache2/icinga-error.log
    CustomLog /var/log/apache2/icinga-access.log combined
</VirtualHost>
EOF
\end{lstlisting}

\subsection{Activatie}
\begin{lstlisting}[language=bash]
a2ensite icinga.conf
apache2ctl configtest
systemctl reload apache2
\end{lstlisting}


\section{PHP-FPM}
PHP-FPM is een FastCGI Process Manager tussen Apache en PHP. PHP-FPM wordt vaak gebruikt op zwaar belaste web-servers. Meer informatie is te vinden op \url{https://www.php.net/manual/en/install.fpm.php}

\subsection{Installatie}
\begin{lstlisting}[language=bash]
DEBIAN_FRONTEND=noninteractive apt-get -y install \
php-fpm                \
\end{lstlisting}

\subsection{Activatie}
\begin{lstlisting}[language=bash]
my_php_fpm=`apt-cache showpkg php-fpm | grep -A 1 'Reverse Provides' | tail -1 | cut -d' ' -f1`
a2enmod proxy_fcgi setenvif
a2enconf ${my_php_fpm}
apache2ctl configtest
systemctl restart ${my_php_fpm}.service
systemctl restart apache2.service
\end{lstlisting}

\section{MariaDB}
MariaDB is een SQL-database. MariaDB is een onafhankelijke open source doorontwikkeling van MySQL dat in handen is van Oracle.

\subsection{Installatie}
\begin{lstlisting}[language=bash]
DEBIAN_FRONTEND=noninteractive apt-get -y install \
mariadb-client         \
mariadb-server         
\end{lstlisting}

\subsection{Configuratie}
\begin{lstlisting}[language=bash]
mysql_secure_installation
\end{lstlisting}

\section{Icinga}
Monitoring server gebaseerd op en compatible met Nagios.

\subsection{Installatie}
\begin{lstlisting}[language=bash]
DEBIAN_FRONTEND=noninteractive apt-get -y install \
icinga2                \
icingacli              \
icingaweb2             \
icinga2-ido-mysql      \
monitoring-plugins     \
nagios-nrpe-plugin     \
nagios-plugins-contrib
\end{lstlisting}

\subsection{Icinga2 database}
Achterhaal het gegenereerde wachtwoord voor db\_ido\_mysql en stop dat in een variabele
\begin{lstlisting}[language=bash]
ICINGA_IDO_PASSWORD="$(awk -F'"' '/password/ {print$2}' /etc/icinga2/features-available/ido-mysql.conf)"
\end{lstlisting}
Maak de icinga2 database aan:
\begin{lstlisting}[language=bash]
mysql <<< "
    CREATE DATABASE icinga2;
    GRANT SELECT, INSERT, UPDATE, DELETE, DROP, CREATE VIEW, INDEX, EXECUTE
    ON icinga2.*
    TO 'icinga2'@'localhost'
    IDENTIFIED BY '$ICINGA_IDO_PASSWORD';
    FLUSH PRIVILEGES;
"
\end{lstlisting}

\subsection{IDO-MySQL}
\begin{lstlisting}[language=bash]
icinga2 feature enable command ido-mysql
mysql icinga2 < /usr/share/icinga2-ido-mysql/schema/mysql.sql
\end{lstlisting}

\subsection{Icinga monitoring}
\begin{lstlisting}[language=bash]
icingacli module enable monitoring
\end{lstlisting}


\subsection{icingaweb2 database}
\begin{lstlisting}[language=bash]
mysql <<< "
    CREATE DATABASE icingaweb2;
    GRANT ALL
    ON icingaweb2.*
    TO 'icingaweb2'@'localhost'
    IDENTIFIED BY '$ICINGAWEB2_DB_PASSWORD';
    FLUSH PRIVILEGES;
"
mysql icingaweb2 < /usr/share/icingaweb2/etc/schema/mysql.schema.sql
\end{lstlisting}

\subsection{icingaweb2 configuratie}
\begin{lstlisting}[language=bash]
# roles.ini
cat << EOF > /etc/icingaweb2/roles.ini
[Administrators]
users = "$ICINGA_WEB_ADMIN_USER"
permissions = "*"
groups = "Administrators"
EOF

# groups.ini
cat << EOF > /etc/icingaweb2/groups.ini
[icingaweb2]
backend = "db"
resource = "icingaweb_db"
EOF

# config.ini
cat << EOF > /etc/icingaweb2/config.ini
[global]
show_stacktraces = "1"
config_backend = "db"
config_resource = "icingaweb_db"
#
[logging]
log = "syslog"
level = "ERROR"
application = "icingaweb2"
facility = "user"
EOF

# authentication.ini
cat << EOF > /etc/icingaweb2/authentication.ini
[icingaweb2]
backend = "db"
resource = "icingaweb_db"
EOF

# resources.ini
cat << EOF > /etc/icingaweb2/resources.ini
[icingaweb_db]
type = "db"
db = "mysql"
host = "localhost"
port = ""
dbname = "icingaweb2"
username = "icingaweb2"
password = "$ICINGAWEB2_DB_PASSWORD"
charset = "UTF8"
persistent = "0"
use_ssl = "0"
#
[icinga_ido]
type = "db"
db = "mysql"
host = "localhost"
port = ""
dbname = "icinga2"
username = "icinga2"
password = "$ICINGA_IDO_PASSWORD"
charset = "latin1"
persistent = "0"
use_ssl = "0"
EOF

mkdir /etc/icingaweb2/modules/monitoring/

# config.ini
cat << EOF > /etc/icingaweb2/modules/monitoring/config.ini
[security]
protected_customvars = "*pw*,*pass*,community"
EOF

# commandtransports.ini
cat << EOF > /etc/icingaweb2/modules/monitoring/commandtransports.ini
[icinga2]
transport = "local"
path = "/var/run/icinga2/cmd/icinga2.cmd"
EOF

# backends.ini
cat << EOF > /etc/icingaweb2/modules/monitoring/backends.ini
[icinga]
type = "ido"
resource = "icinga_ido"
EOF
\end{lstlisting}

\subsection{Admin wachtwoord voor icingaweb}
\begin{lstlisting}[language=bash]
# Create a hash from password
HASH_ICINGA_WEB_ADMIN_PASSWORD=$(openssl passwd -1 "$ICINGA_WEB_ADMIN_PWD")

# Create user in database
mysql icingaweb2 -Bse "
    INSERT INTO icingaweb_user
        (name, active, password_hash)
        VALUES ('$ICINGA_WEB_ADMIN_USER', 1, '$HASH_ICINGA_WEB_ADMIN_PASSWORD');
"
\end{lstlisting}
