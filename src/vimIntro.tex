De naam \texttt{vim}\index{vim} staat voor Vi IMproved\index{Vi IMproved}. Of wel een verbeterde versie van \texttt{vi}. Bram Molenaar, een Nederlandse software ontwikkelaar, schrijft al sinds 1991 aan de code van \texttt{vim} en zijn verbeteringen zijn zo populair dat op alle Linux systemen alleen nog \texttt{vim} ge\"installeerd wordt. Je kan op een Linux systeem \texttt{vim} opstarten als \texttt{vi} waarmee je zoveel mogelijk de functionaliteiten krijgt als het oude \texttt{vi} en je kan \texttt{vim} opstarten als \texttt{vim} waarmee je alle nieuwe toevoegingen van Bram Molenaar en zijn medeontwikkelaars krijgt.

\texttt{vim} is niet een van de makkelijkste editors maar heeft als grote voordeel, zoals eerder gezegd, dat het beschikbaar is op elk willekeurig Unix systeem.
