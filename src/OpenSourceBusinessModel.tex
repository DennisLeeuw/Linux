Een veel gehoord tegenargument bij (tegen) het gebruik van een open source licentie bij de ontwikkeling
van nieuwe software is dat er geen geld te verdienen zou zijn op deze manier en dat is gedeeltelijk waar. Het \'e\'en keer
ontwikkelen van software, het daarna eindeloos kopi\"eren en er geld voor vragen, dat werkt niet meer. Maar je mag nog
altijd geld vragen voor de distributie, je mag (installatie) hulp aanbieden (helpdesk), cursussen aanbieden, of adviesuren verkopen. Er blijven dus voldoende middelen over om geld te verdienen.
