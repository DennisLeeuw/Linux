Linux servers hangen meestal in een datacenter dat goed beveiligd en slecht toegankelijk is. Het is makkelijker om machines vanaf afstand te beheren. Voor het beheer op afstand zijn er verschillende oplossingen. Dit hoofdstuk behandelt er een aantal van. Niet alle oplossingen zullen behandeld worden, dat is ook de bedoeling niet, het is de bedoeling om een aantal oplossingen te laten zien zodat je zelf verder kan zoeken.

Remote management geeft de beheerder(s) de mogelijkheid om op afstand een systeem aan te passen, dat is een potentieel veiligheidsrisico. Het is verstandig om een management netwerk te maken dat niet toegankelijk is vanaf het Internet. Alle machines kunnen dan met een aparte netwerkkaart in dit management netwerk gehangen worden.

Er zijn verschillende manieren om systeembeheer op afstand te doen. Je kan inloggen op een systeem en dan handmatig de gewenste bestanden aanpassen. Er zijn ook applicaties waarmee je op afstand wijzigingen in een bestand kan aanbrengen.

