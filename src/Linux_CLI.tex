\documentclass[a4paper,12pt,twoside,openright,titlepage]{book}

%Additional packages
\usepackage[utf8]{inputenc}
\usepackage[T1]{fontenc}
\usepackage[dutch,english]{babel}
\usepackage{imakeidx}
\usepackage{syntonly}
\usepackage[official]{eurosym}
%\usepackage[graphicx]
\usepackage{graphicx}
\graphicspath{ {./images/} }
\usepackage{float}
\usepackage{xurl}
\usepackage{hyperref}
\hypersetup{colorlinks=true, linkcolor=blue, citecolor=blue, filecolor=blue, urlcolor=blue, pdftitle=, pdfauthor=, pdfsubject=, pdfkeywords=}
\usepackage{tabularx}
\usepackage{scrextend}
\addtokomafont{labelinglabel}{\sffamily}
\usepackage{listings}
\usepackage{adjustbox}
\usepackage{color}

% Define colors
\definecolor{ashgrey}{rgb}{0.7, 0.75, 0.71}

% Listing style
\lstset{
  backgroundcolor=\color{ashgrey}, % choose the background color; you must add \usepackage{color} or \usepackage{xcolor}; should come as last argument
  basicstyle=\footnotesize,        % the size of the fonts that are used for the code
  breakatwhitespace=true,          % sets if automatic breaks should only happen at whitespace
  breaklines=true,                 % sets automatic line breaking
  extendedchars=true,              % lets you use non-ASCII characters; for 8-bits encodings only, does not work with UTF-8
  frame=single,	                   % adds a frame around the code
  rulecolor=\color{black},         % if not set, the frame-color may be changed on line-breaks within not-black text (e.g. comments (green here))
  keepspaces=true,                 % keeps spaces in text, useful for keeping indentation of code (possibly needs columns=flexible)
  columns=fullflexible,		   % make copy and paste possible
  showstringspaces=false,          % if true show spaces in strings adding particular underscores
  showspaces=false,                % if true show spaces everywhere adding particular underscores; it does not override 'showstringspaces'
}

% Uncomment for production
% \syntaxonly

% Style
\pagestyle{headings}

% Turn on indexing
\makeindex[intoc]

% Define document
\author{D. Leeuw}
\title{Linux de Command Line Interface}
%\subtitle{Linux voor MBO niveau 4 en het LPI Linux Essentials examen}
%\subject{Een Praktische Gids}
\date{\today\\v.0.3.0}

\begin{document}
\selectlanguage{dutch}

\maketitle

\copyright\ 2020-2022 Dennis Leeuw\\

\begin{figure}
\includegraphics[width=0.3\textwidth]{CC-BY-SA-NC.png}
\end{figure}

\bigskip

\input{src/licentie}

%%%%%%%%%%%%%%%%%%%
%%% Introductie %%%
%%%%%%%%%%%%%%%%%%%

\frontmatter
\chapter{Over dit Document}
\input{src/OverDitDocument}
\input{src/DocChanges_CLI}

%%%%%%%%%%%%%%%%%
%%% De inhoud %%%
%%%%%%%%%%%%%%%%%
\tableofcontents

\mainmatter
\chapter{Inleiding}
Deze Linux cursus beoogt aan te sluiten bij het Linux Essentials examen van de LPI (Linux Professional Institute) en dient als voorbereiding op het MBO ICT Systems and Devices Expert examen. Voor het leren gebruiken van de grafische interface en de command line maken we gebruik van CentOS en om kennis te maken met het gebruik van Linux als server/command line installeren we Debian. De keuze om CentOS als werkstation te installeren en Debian als server is volledig willekeurig. Het doel is dat de studenten kennis maken met de zowel rpm/dnf en de apt package managers en leren dat het ene Linux systeem het andere niet is.

Alle Linux systemen zullen ge\"installeerd worden als virtuele machines. Door gebruik te maken van virtuele machines zijn we niet afhankelijk van de onderliggende hardware. De keuze van de virtuele omgeving is aan de gebruiker. Ons advies zou zijn om gebruik te maken van VirtualBox en VMware Workstation, beide zijn gratis en kunnen op de meest gangbare operating systems gebruikt worden.

Voor de CentOS machine is 15G vrije schijfruimte nodig en voor het Debian systeem 5G, wat een totaal aan 20G vrije schijfruimte vereist. Voor elke machine hebben we 2G RAM nodig, dus een totaal van 4G RAM moet vrij beschikbaar zijn.



% Requires: GUI, DASH
% Provides: ls, pwd, whoami, hostname, history, mkdir, cd, rmdir, ~
% PATH, HOME, USER
\chapter{Waarom de commandline interface?}
De eerste vraag is meestal waarom we de command line zouden gebruiken als we al een grafische interface hebben. Het meest simpele antwoord is dat Unix van oorsprong alleen maar een command line interface of CLI had. Maar een beter antwoord is dat een grafische interface veel resources (geheugen en processor) gebruikt en die resources kunnen we beter inzetten voor de taken die we het systeem geven. Veel Linux machines draaien als servers in een serverruimte en staan op de achtergrond hun ding te doen, bijvoorbeeld als webserver. Voor die functie is geen grafische interface noodzakelijk terwijl op een drukbezochte website elk stukje processor of geheugen nodig kan zijn om de website soepel te laten lopen. Wat we niet nodig hebben installeren we dan ook niet op de machine, dus geen grafische interface.

Daarnaast zal je hopelijk ervaren dat, omdat Linux op de schouders staat van de vele jaren ervaring uit de Unix-wereld, dat de command line een enorm krachtige interface is om mee te werken. Vaak kan je op de command line dingen sneller en makkelijker doen dan je in een grafische interface zou kunnen. Wees niet bevreest als dat in eerste instantie niet zo lijkt. De leercurve kan, zeker als je niet veel ervaring hebt met bijvoorbeeld de command prompt of powershell van Windows, soms erg stijl zijn.

De voorbeelden in dit document zijn gemaakt op een Debian machine, maar daarmee niet Debian specifiek. Ze zullen op een Red Hat, Centos, Mint of Ubuntu machine geen wezenlijk andere resultaten opleveren. De grote verschillen tussen de verschillende distributies zitten vooral in de grafische interface, de verschillen op de command line zijn minimaal voor de onderwerpen die in dit document worden behandeld. Het doel van dit document is je vetrouwt maken met de CLI, niet het configureren van van het systeem.


\section{Toegang tot de CLI}
Een Linux server die ge\"installeerd is zonder grafische interface heeft op zijn monitor de prompt Login: daar kan je
inloggen als gebruiker. Via de toets combinatie ALT en F1-F6 kan je schakelen naar verschillende consoles zodat je
meerdere keren ingelogd kan zijn.

Als je werkt vanuit een grafische interface kan je gebruik maken van de Terminal of Terminal Emulator applicatie. Deze brengt je bij de command line interface, zie figuur \ref{fig:DashTerminal}

\begin{figure}
\includegraphics{linuxreader-img021.png}
	\label{fig:DashTerminal}
	\caption{Terminal op de Dash}
\end{figure}


\section{De prompt}
\input{src/CLIPrompt}
\section{De shell}
\input{src/CLIShell}
\subsection{History}
\input{src/CLIHistory}
\section{De home-directory}
\input{src/HomeDirectory}
\section{Shell variabelen}
De shell houdt voor de gebruiker informatie bij over de omgeving waarin hij werkt. Met welke gebruikersnaam zijn we ingelogd, wat is onze home-directory en nog veel meer van dit soort informatie. Deze informtie is opgeslagen in variabelen. Met het \texttt{echo} commando kunnen we die informatie uit die variabelen opvragen:
\begin{lstlisting}[language=bash]
$ echo $USER
$ echo $SHELL
$ echo $HOME
$ echo $PWD
\end{lstlisting}
om een complete lijst te krijgen van alle variabelen die je in je huidige sessie tot je beschikking staan is er het commando \texttt{env}\index{env}\index{commando!env}.


\subsection{Eigen variabelen}
\input{src/VariabelenEigen}

% Requires: ls
% Provides: fhs, ps (short)
\chapter{Het bestandssysteem}
\section{Everything is a file}
In Unix en Unix-like operating systems zoals Linux is het basis principe dat alles een bestand (file) is. Dit betekent dat alles binnen het systeem; documenten, directories, harddisks, printers, toetsenborden, maar ook processen weergegeven worden als bestanden. Het voordeel hiervan is dat dezelfde commando's en API's gebruikt kunnen worden voor verschillende onderdelen van het besturingssysteem.

Het filesysteem is een enkele boom van bestanden en directories, zonder onderscheid tussen disks en partities. Zelfs verplaatsbare media zoals USB-sticks en DVDs zijn onderdeel van deze boom als ze 'gemount' zijn. Ook processen (/proc) en de kernel (/sys) is voor een groot deel benaderbaar via het bestandssysteem.

Met \texttt{ls} kan je in bijvoorbeeld \texttt{/proc} kijken en zien welke processen er zijn. Je ziet er nummer staan en die nummers komen overeen met de process nummers die ook \texttt{ps} heeft. Doe maar eens:
\begin{lstlisting}[language=bash]
$ ps aux
\end{lstlisting}
en je ziet in de tweede kolom dezelfde nummers staan. \texttt{ps} is dan ook het commando om te zien welke processen er op je systeem actief zijn. We zullen ps in een volgend hoofdstuk nog uitgebreid behandelen.

\subsection{Bestandstypen}
In de eerste kolom van \texttt{ls -l} vinden we de bestandsrechten en het geeft tevens aan met welk type bestand we te maken hebben. Naast normale bestanden (zoals tekstbestanden) hebben we op POSIX-compliant systemen ook speciale bestanden zoals directories.

\begin{tabularx}{5in}{ |c|c|X| }
\hline
	Mode veld & Bestandstype & Beschrijving \\
\hline
\hline
	- & normaal bestand & Documenten, etc. \\
\hline
	d & directory & Directories bevatten geen bestand, maar een overzicht van de bestandsnamen waaraan gekoppeld referenties naar iets wat inodes worden genoemd. De inodes bevatten de daadwerkelijke bestanden en meta-data (eigenaar, groep, permissies, time stamps, etc.). Door deze manier van werken kan een bestand (met meta-data) verschillende namen hebben (hard-link), mits binnen \'e\'en bestandssysteem (partitie of disk). Bij verschillende bestandsnamen kunnen dezelfde inodes vermeldt staan. \\
\hline
	l & Symbolic link & een link naar een bestand die over bestandssystemen heen kan gaan \\
\hline
	b & Block device & Een apparaat waar van of waar naar toe data in een random manier gestuurd kan worden. Het hoeft dus niet in de juiste volgorde te zijn. Denk aan harddisks waar eerst sector 2014 en dan sector 5678 geschreven kan worden. \\
\hline
	c & Character device & Een apparaat waar data in een stroom van characters naar of naar toe gestuurd kan worden. \\
\hline
	p & FIFO & Ook bekend als named pipes. Een pipe verbindt het ene proces met het andere proces zodat data van proces 1 naar proces 2 gestuurd kan worden. Dit kan maar \'e\'en kant op. \\
\hline
	s & Socket & Verbindt net als FIFO's processen, maar dan op een manier dat er twee weg communicatie mogelijk is. \\
\hline
\end{tabularx}


\section{FHS - Filesystem Hierarchy Standard}
Net als met de standaardisatie van Unix in een POSIX standaard werden er in het begin op Linux Distributies soms bestanden in verschillende directories neergezet. Dat is voor programma's die op die systemen moeten draaien niet handig. Als de ene distributie /var/db heeft voor het plaatsen van databases en de ander /var/databases dan schept dat verwarring. De oplossing die hiervoor gekomen is is de Filesystem Hierarchy Standard. Deze is beschikbaar op \url{https://refspecs.linuxfoundation.org/fhs.shtml}. Hier gaan we heel globaal in op een aantal belangrijke directories, mocht je alle ins en outs willen weten dan raden we je aan om het document een keer te lezen.

\begin{description}
\item [/] De basis van het bestandssysteem wordt bepaald door de root-directory, zo genoemd omdat de vertakkende directories op een boom structuur lijkt en het Engelse root betekend wortel.

Een ls van \texttt{/} laat ons een aantal verschillende directories zien. Waarvan we er een aantal zullen behandelen.

\item [/home/] bevat de directories waarin gebruikers hun bestanden kunnen zetten. Een uitzondering hierop is de directory waarin de root gebruiker (de baas of administrator van het systeem), zijn bestanden kan opslaan. Die directory is \texttt{/root/}.

\item [/etc/] Deze directory bevat de configuratiebestanden van het systeem. Als je een instelling systemwide wilt wijzigen is dit de plek om te gaan zoeken. De configuratie bestanden voor een gebruiker staan in zijn of haar home-directory.

\item [/boot/] Deze directory bevat bestanden die cruciaal zijn voor het opstarten maar die geen commando zijn. Hier vinden we de kernel en bestanden die behoren bij de bootloader.

\item [/dev/] Omdat op een Linus systeem alles een bestand is vind je in deze directory de bestanden die verwijzen aan devices. Devices worden verder besproken in het hoofdstuk over devices. Dus daar gaan we later nog op in.

\item [/var/] is de directory voor de systeem opslag van variabele data zoals bijvoorbeeld de logbestanden, databases, etc. De log bestanden kan je vinden in \texttt{/var/log/}.

\item [/srv/] bevat de data van de diensten die door het systeem worden aangeboden. Data van web- of ftp-servers kan hier gevonden worden.
\end{description}


% Requires: ls, fhs
% Provides: which
\chapter{Commando's}
\input{src/Commandos}
\section{Waar zijn de commando's?}
Een gemiddeld Linux systeem bevat heel veel commando's en dat is omdat er in de Unix-wereld twee filosofie\"en zijn die bij elkaar aansluiten de eerste is het KISS-principe. KISS staat voor Keep It Simple, Stupid en is oorspronkelijk afkomstig uit de US Navy. De tweede is Small is Beautiful en die is afkomstig uit de Economie.\footnote{Small Is Beautiful: A Study of Economics As If People Mattered door E. F. Schumacher}

Je komt op UNIX-achtige systemen dan ook vele kleine commando's tegen die \'e\'en ding goed doen. Dit heeft een aantal voordelen. Omdat ze maar \'e\'en ding doen is de code simpel en is dus code waarin de programma's geschreven zijn makkelijker te controleren op fouten. Een ander voordeel is dat niet iedereen in elk programma weer dezelfde code hoeft te herhalen maar gebruik kan maken van iemand anders zijn programma. De totale hoeveelheid code is daardoor klein, een compleet Linux systeem met grafische interface kan ge\"installeerd worden op een 15GB disk, zonder grafische interface past het zelfs op een 5GB disk. De laatste reden is dat wij als gebruikers vaak zonder te programmeren al heel complexe dingen met Linux kunnen doen omdat we al die kleine commando's aan elkaar kunnen knopen waarmee complexe dingen te doen zijn.

Het nadeel van heel veel kleine programma's is dat je het overzicht snel kwijt kan raken, zeker ook omdat vele commando's soms cryptische afkortingen hebben. Zo staat \texttt{ls} voor list. De oorspronkelijke ontwikkeling van Unix werd gedaan op systemen met toetsenborden die niet zo ergonomisch zijn als tegenwoordig. Ze hadden toetsen die je met enige kracht moest indrukken en na een dag programmeren hadden de programmeurs Ken Thompson en Dennis Ritchie en hun team vaak zere knokkels door overbelasting. Door commando's zo kort mogelijke namen te geven verminderden ze het aantal toetsaanslagen. Vandaar de vaak korte commando namen.

Type
\begin{lstlisting}[language=bash]
$ ls
\end{lstlisting}
en je zal een aantal blauwe directories op je scherm zien verschijnen. Als je nu
\begin{lstlisting}[language=bash]
$ ls /usr/bin/
\end{lstlisting}
typt dan verschijnen er allemaal groene commando's op je scherm, of beter ze scrollen van je scherm af in kolommen, omdat het er heel veel zijn. Het past niet op je scherm. Als we de hele lijst willen zien dan zullen we gebruik moeten maken van een programma de uitvoer van \texttt{ls} opdeelt in pagina's die zoveel regels bevatten dat ze het scherm vullen.
Een programma dat dat doet heet \texttt{more}. De kunst is nu om de uitvoer van \texttt{ls} te koppelen aan \texttt{more} en daarvoor is er de pipe, \texttt{|}, of de pijp. Het pipe-character koppelt twee commando's aan elkaar:
\begin{lstlisting}[language=bash]
$ ls /usr/bin/ | more
\end{lstlisting}
met de spatie-balk kan je nu pagina voor pagina bekijken en met de letter q verlaat je \texttt{more}. Nu is \texttt{more} wel heel simpel en kan het alleen dat wat je nu gezien hebt. Makkelijker zou het zijn als je omhoog en omlaag door de commando's kan gaan, en misschien zelfs wel zou kunnen zoeken in zo'n lange lijst. Dat kan ook, daarvoor hebben we de opvolger van \texttt{more} die meer kan en \texttt{less} heet want less is more.

\begin{lstlisting}[language=bash]
$ ls /usr/bin/ | less
\end{lstlisting}
Nu kan je met de spatie-balk door de pagina's gaan, met de pijltjes omhoog en omlaag per regel door de lijst gaan, met PgUp en PgDn per pagina omhoog en omlaag gaan en met / kan je zoeken. Type maar eens als je in \texttt{less} zit
\begin{lstlisting}[language=bash]
/less
\end{lstlisting}
Zo kom je bij het eerste commando uit dat less in de naam heeft. Met n kan je zoeken naar het volgende (next) commando dat ook less bevat, en zo vderer . Ook hier weer is de q-toets de manier om \texttt{less} te verlaten.

We hebben nu gezien dat heel veel commando's terug te vinden zijn in de \texttt{/usr/bin/} directory. Maar dit is maar \'e\'en plek waar commando's te vinden zijn. Commando's vind je terug in de \texttt{bin/} en \texttt{sbin/} directories. We gebruiken hier bewust een meervoud omdat we deze directories op verschillende plekken kunnen vinden. Je bent in de \texttt{/usr} directory al de \texttt{bin/} directory tegen gekomen. De \texttt{bin/} directories bevatten commando's die door iedereen gebruikt kunnen worden. De \texttt{sbin/} directories zijn voor de commando's die alleen toegankelijk zijn voor de systeembeheerder.

Naast in \texttt{/usr/} vind je ook \texttt{bin/} en \texttt{sbin/} directories in de \texttt{/} en de \texttt{/usr/local/} directory. Al deze locaties hebben een andere functie:
\begin{itemize}
	\item \texttt{/} is de root van het systeem en de \texttt{bin/} en \texttt{sbin/} directories bevatten daar de commando's die nodig zijn voor het opstarten van het systeem en zijn afkomstig van de distributie.
	\item \texttt{/usr/} bevat (s)bin directories die de commando's bevatten voor normaal gebruik van het systeem en deze commando's zijn afkomstig van de distributie.
	\item \texttt{/usr/local/} bevat (s)bin directories die commando's bevatten die door de systeembeheerder ge\"installeerd zijn (gecompileerd)
	\item \texttt{/opt/} bevat subdirectories met daarin (s)bin directories per ge\"installeerd softwarepakket dat niet door de distributie is meegeleverd, maar later vanaf een pakket is ge\"installeerd.
\end{itemize}

Om te bepalen welke directories gebruikt worden voor het zoeken naar een commando is er een variabele aanwezig in de shell en die variabele heeft de logische naam PATH. Type maar eens:
\begin{lstlisting}[language=bash]
$ echo $PATH
\end{lstlisting}
Dit levert een output op die er ongeveer zo uit ziet: \texttt{/usr/local/bin:/usr/bin:/bin} (dit kan per systeem verschillen). Voor een gebruiker met dit PATH wordt er voor een commando eerst gezocht in \texttt{/usr/local/bin/}, daarna in \texttt{/usr/bin/} en als laatste in \texttt{/bin/}.

Als we willen weten waar een commando vandaan komt, dan kunnen we \texttt{which} gebruiken:
\begin{lstlisting}[language=bash]
$ which ls
\end{lstlisting}

Met het commando su kan je als een andere gebruiker inloggen (switch user). Type eens:
\begin{lstlisting}[language=bash]
$ su - root
\end{lstlisting}
geef het root wachtwoord en type
\begin{lstlisting}[language=bash]
# echo $PATH
# exit
\end{lstlisting}
Na het su commando moet je het password van de root gebruiker geven dat je ingesteld hebt tijdens de installatie. Je zult nu zien dat het PATH van de root gebruiker ook de \texttt{sbin/} directories bevat. De root gebruiker heeft dus veel meer commando's tot zijn beschikking dan een normale gebruiker.


\section{Error codes}
\input{src/ErrorCodes}

% Requires: fhs
% Provides: man, locate, info, apropos
\chapter{Linux documentatie}
Bijna elk linux-systeem is standaard voorzien van een uitgebreide set van documentatie. De meeste uitgebreidde documentatie op Unix-achtige systemen is te vinden in de zogenaamde man-pages. \texttt{man} is een afkorting voor Manual ofwel handleiding. Daarnaast kent Linux nog het \texttt{info} documentatie systeem. Ook zijn er nog andere manieren om extra informatie over commando's of het systeem te vinden. De voornoemde zaken worden allemaal in dit document behandeld.

\section{man-pages}
\input{src/DocMan}
\section{Waar vind ik iets?}
%\begin{figure}
%\includegraphics[width=0.4\linewidth]{linuxreader-img028.png}
%	\caption{Het zoeken van informatie}
%\end{figure}

Om uit te vinden welk commando je kan gebruiken om iets op het systeem te bereiken kan je \texttt{apropos}\index{apropos}\index{Zoeken!apropos} gebruiken. Type eens (Let op de quotes!):
\begin{lstlisting}[language=bash]
$ apropos 'make directories'
\end{lstlisting}
je vindt dan het \texttt{mkdir} commando. Het nadeel van dit zoek systeem is dat het heel specifiek is.

\begin{lstlisting}[language=bash]
$ apropos 'make directory'
\end{lstlisting}
doet niets. Als je het dus niet meteen vindt probeer dan enkelvoud- en meervoudsvormen.

Heb je een commando gevonden waarvan je denkt dat het is wat je nodig hebt probeer dan \texttt{whatis}\index{whatis}:

\begin{lstlisting}[language=bash]
$ whatis mkdir
\end{lstlisting}

Dit geeft een korte beschrijving van een commando. Voor de volledige manual gebruiken we het natuurlijk \texttt{man}:
\begin{lstlisting}[language=bash]
$ man mkdir
\end{lstlisting}


\section{Info}
The GNU-project heeft een eigen documentatie systeem ontworpen dat info\index{info} genoemd wordt. Het is een hypertext, dus met links, gerelateerd systeem. Dit is dus documentatie die je naast \texttt{man} tegen komt op systemen. Mocht je in de manual-pages niet vinden wat je zoekt, misschien kan je dan eens \texttt{info} proberen. Ook in info werken de pijltjes voor het scrollen en is de q-toets er weer om het \texttt{info} te verlaten. Een extra functie is de enter-toets die je kan gebruiken om een link te volgen. Een link herken je aan een onderstreepte tekst.

Het \texttt{info} systeem is niet op iedere Linux-distributie standaard ge\"installeerd, het kan dus gebeuren dat \texttt{info} op jouw systeem niet werkt.


\section{/usr/share/doc}
Een andere plek waar we op ons systeem informatie kunnen vinden is in de /usr/share/doc directory. Met het commando \texttt{ls} kan je een listing van een directory opvragen. Type

\begin{lstlisting}[language=bash]
$ ls /usr/share/doc
\end{lstlisting}

en een enorme lijst van directories vliegt over je scherm. Dit is de documentatie van elk pakket dat op het systeem ge\"installeerd is. Een linux systeem is opgebouwd uit allerlei software pakketten die vanaf source code gecompileerd zijn (door je distributie maker). De informatie die met de source code meekwam, zoals de licentie-bestanden\index{Licenties}, eventuele FAQ-bestanden\index{FAQ}, READMEs\index{README} etc. vind je terug in de subdirectories van /usr/share/doc. Belangrijk zijn vaak de voorbeeld configuratie-bestanden.


\section{Internet}
Als laatste documentatie bron willen we graag het Internet vermelden. Omdat GNU/Linux een open source besturingssysteem is is er heel veel online te vinden. De kans dat jij als eerste tegen een probleem aanloopt is zeer klein. Het is dus naast de voornoemde bronnen \'e\'en van de eerste zaken om te raadplegen.

De broncode van alle software is openbaar en veel van de projecten hebben hun eigen website met vaak ook een eigen forum\index{forum} om contact te houden met gebruikers. Veel informatie is terug te vinden in de fora en de FAQs\index{FAQ}.



% Requires: fhs
% Provides: touch, cat, mkdir, rmdir, >, >>, <, <<, redirects
\chapter{Werken met bestanden}
Als alles een bestand is in Linux dan is het werken met \index{Bestanden}bestanden het belangrijkste wat er is. Dit hoofdstuk gaat je dan ook de basisbeginselen bijbrengen van het werken met bestanden.

\section{Directories}
Om data op een computer te structureren is het handig om bestanden te verdelen over directories\index{Directories}. Directories zijn ook bekend als mappen\index{Mappen} en folders\index{Folders}. Wij zullen alleen nog spreken van directories omdat dat binnen de Unix-wereld de meest gebruikte term is.

Een directory maak je aan met het commando \texttt{mkdir}\index{mkdir}\index{Directories!mkdir}:
\begin{lstlisting}[language=bash]
$ mkdir LinuxCursus
\end{lstlisting}
Met \texttt{ls} kan je controleren of de directory ook daadwerkelijk aangemaakt is.

Je kan ook meerdere directories tegelijk aanmaken door ze als een lijst op te geven, gescheiden door spaties:
\begin{lstlisting}[language=bash]
$ mkdir Aap Noot Mies
\end{lstlisting}

Soms wil je ook een heel pad gelijk aanmaken met:
\begin{lstlisting}[language=bash]
$ mkdir Boom/Roos/Vis/Vuur
\end{lstlisting}
gaat dat niet lukken, want de Boom directory bestaat niet. Gelukkig kan je aan \texttt{mkdir} en optie meegeven die vertelt dat \texttt{mkdir} ook alle onderliggende directories moet aanmaken:
\begin{lstlisting}[language=bash]
$ mkdir -p Boom/Roos/Vis/Vuur
\end{lstlisting}
Gebruik \texttt{ls} om te controleren dat alle directories aanwezig zijn.

Tot slot wil je ook instaat zijn om directories weg te gooien. Met \texttt{rmdir}\index{rmdir}\index{Directories!rmdir} kan dit als de directory leeg is.
\begin{lstlisting}[language=bash]
$ rmdir Aap Noot Mies
\end{lstlisting}
gooit keurig alle aangemaakte directies weg. Controleer dit met \texttt{ls}. Maar doen we:
\begin{lstlisting}[language=bash]
$ rmdir Boom
\end{lstlisting}
dan krijgen we een error melding, want de \texttt{Boom} directory is niet leeg. We zullen dus eerst alle andere subdirectories moeten weggooien, te beginnen met \texttt{Vuur}, dan \texttt{Vis}, dan \texttt{Roos} en tot slot kunnen we pas \texttt{Boom} weggooien. Gebruik \texttt{rmdir} om alle directories \textbf{behalve} Boom te verwijderen. De \texttt{Boom} directory moet dus blijven bestaan. 

\section{Bestanden maken en stdin, stdout en stderr}
Zorg dat je in de directory LinuxCursus staat en type:

\begin{lstlisting}[language=bash]
$ touch hello.txt
\end{lstlisting}

Na de Enter lijkt er helemaal niets te gebeuren. Dit is met de meeste Linux commando's het geval. Als het goed gegaan is dan laten ze niets weten, een beetje als \textquote{geen nieuws, is goed nieuws}. Doen we een \texttt{ls} dan zien we dat er een bestand is aangemaakt dat hello.txt heet.

Met \texttt{touch}\index{touch} kunnen we dus bestanden aanmaken\index{bestanden maken!leeg}\index{lege bestanden maken}, dit zijn lege bestanden. Type maar eens:

\begin{lstlisting}[language=bash]
$ cat hello.txt
\end{lstlisting}

dan zal je zien dat er weer niets op je scherm verschijnt. En dat is goed! Het \texttt{cat}\index{cat} commando plaatst de inhoud van een bestand op het scherm en daar we een leeg bestand hebben opgevraagd is wat er op het scherm komt dus niets en omdat dat succesvol is verlopen hoeft \texttt{cat} ook geen foutmelding te laten zien en met de wetenschap dat geen nieuws, goed nieuws is is \texttt{cat} klaar.

In een vorig hoofdstuk hebben we met \texttt{echo} tekst naar het scherm geschreven en nu hebben we met \texttt{cat} een bestand op het scherm afgebeeld. Vanuit Linux gezien is dat niet helemaal correct geformuleerd. Zowel \texttt{echo} als \texttt{cat} schrijven naar de \textquote{standaard output}\index{stanaard output} en in de terminal is het scherm de standaard output. De standaard output wordt vaak afgekort als stdout\index{stdout}.

We kunnen de standaard output ook omleiden\index{omleiden!stdout}\index{stdout!omleiden} (redirect\index{redirect!stdout}\index{stdout!redirect}) naar bijvoorbeeld een bestand:

\begin{lstlisting}[language=bash]
$ echo 'Ik werk met Linux' > hello.txt
\end{lstlisting}

We zien nu dat de zin die we met echo afbeelden niet meer op het scherm verschijnt. Hij is verdwenen en er lijkt weer helemaal niets gebeurd te zijn. Als we nu

\begin{lstlisting}[language=bash]
$ cat hello.txt
\end{lstlisting}

doen dan zien we waar onze zin is gebleven. Hij is in hello.txt terecht gekomen. We hebben de stdout van \texttt{echo} in hello.txt gestopt.

Laten we dat nog eens doen:

\begin{lstlisting}[language=bash]
$ echo 'Hello World!' > hello.txt
\end{lstlisting}

Doen we een \texttt{cat} van hello.txt dan zien we dat onze eerste zin verdwenen is en er alleen nog 'Hello World!' in hello.txt zit. We hebben kennelijk ons bestand overschreven met een nieuwe inhoud. We kunnen ook tekst toevoegen aan een bestand:

\begin{lstlisting}[language=bash]
$ cat 'Ik werk met Linux' >> hello.txt
\end{lstlisting}

door gebruik te maken van het dubbele groter dan teken voegen we een regel toe aan het eind van het bestand. De oude regel zie je met \texttt{cat} als eerste en daaronder komt onze nieuwe regel.

Zou er als we een stdout hebben ook een standaard input\index{standaard input} (stdin\index{stdin}) zijn en kunnen we daar dan van lezen? Ja, die is er! Als je
typt:

\begin{lstlisting}[language=bash]
$ cat < hello.txt
\end{lstlisting}

dan vertellen we eigenlijk dat \texttt{cat} de standaard invoer (stdin)\index{redirect!stdin}\index{stdin!redirect}\index{omleiden!stdin}\index{stdin!omleiden} op het scherm moet afbeelden. Dit is meer typen dan alleen \texttt{cat hello.txt} dus dit gaan we zo nooit gebruiken. Hoe we standaard input wel kunnen gebruiken is door bijvoorbeeld aan de shell te vertellen dat hij vanaf de standaard input moet lezen tot hij een markering tegen komt en daarna moet stoppen.
\begin{lstlisting}[language=bash]
$ cat <<EOF
> Hallo beste mensen
> dit is een stukje tekst
> dat uit meerdere regels bestaat
> EOF
\end{lstlisting}
Je ziet dat dan standaard invoer afgebeeld wordt op het scherm zodra deze de EOF tegen komt. We hebben met \texttt{<<EOF} tegen cat gezegd dat hij van standaard input moet blijven lezen tot hij de letters EOF\index{EOF} (End Of File\index{End Of File}) tegen komt. Daarna doet \texttt{cat} nog steeds waar het goed in is, namelijk het afbeelden op de standaard output. We kunnen natuurlijk ook de standaard output van \texttt{cat} omleiden naar een bestand:
\begin{lstlisting}[language=bash]
$ cat <<EOF >hello.txt
> Hallo beste studenten
> dit is een stukje tekst
> dat uit meerdere regels bestaat
> EOF
\end{lstlisting}
We vertellen \texttt{cat} dus dat hij moet lezen van standaard input totdat hij EOF tegen komt en dat zijn standaard output geredirect moet worden naar het bestand hello.txt. Nadat wij de EOF hebben ingetypt verschijnt de tekst niet op het scherm, maar zit in hello.txt wat we met \texttt{cat} kunnen controleren.

Naast de standaard input en standaard output is er ook nog standaard error\index{standaard error} (stderr\index{stderr}), waar de foutmeldingen naartoe gaan.
Laten we eens een fout maken door een niet bestaan bestand aan \texttt{cat} te geven:

\begin{lstlisting}[language=bash]
$ cat Hello.txt
\end{lstlisting}

Je krijgt nu een foutmelding dat Hello.txt niet bestaat. Linux is case-sensitive dus hello.txt is niet hetzelfde als Hello.txt. De standaard output, input en error zijn genummerd in Linux. Stdin is 0\index{stdin!0}\index{0!stdin}, stdout is 1\index{stdout!1}\index{1!stdout} en stderr is 2\index{stderr!2}\index{2!stderr}. Nu we dit weten zouden we het volgende kunnen doen:

\begin{lstlisting}[language=bash]
$ cat Hello.txt 2> Hello_error.txt
\end{lstlisting}

We zien nu geen foutmelding meer op ons scherm en hebben de stderr omgeleid\index{stderr!omleiden}\index{omleiden!stderr} (redirect\index{redirect!stderr}\index{stderr!redirect}) naar het bestand Hello\_error.txt. Doen we nu een
\begin{lstlisting}[language=bash]
$ cat Hello_error.txt
\end{lstlisting}
dan zien we dat de foutmelding daar is opgeslagen.

Deze vormen van het redirecten (omleiden) van data stromen gebeurt in Linux heel vaak. Programma's schrijven bijvoorbeeld de fouten die ze tegen komen naar een log bestand. Dit doen ze door de stderr te redirecten en de error regels toe te voegen aan het bestand, bijvoorbeeld 2>>error.log. Als ze dit doen met een datum- en tijdmelding dan kan je heel makkelijk problemen opzoeken.

Wij hebben nu geleerd om bestanden aan te maken en om invoer en uitvoer te redirecten.


\section{Bestanden kopie\"eren, verplaatsen, hernoemen, verwijderen}
Om bestanden te kopie\"eren gebruiken \texttt{cp}\index{cp}\index{Bestanden!cp} van het Engelse copy:
\begin{lstlisting}[language=bash]
$ cp hello.txt Boom/hello.txt
\end{lstlisting}
We kunnen ook gelijk de naam veranderen als we dat willen:
\begin{lstlisting}[language=bash]
$ cp hello.txt Boom/Hallo.txt
\end{lstlisting}

Om bestanden verplaatsen gebruiken \texttt{mv}\index{mv}\index{Bestanden!mv} van het Engelse move. Het verschil met copy is dat een bestand niet meer op de oorspronkelijke plek terug te vinden is. Bij move heb je dus maar 1 bestand na de handeling, na copy heb je 2 bestanden.
\begin{lstlisting}[language=bash]
$ mv Boom/Hallo.txt .
\end{lstlisting}
Het \texttt{mv} commando kunnen we ook gebruiken om bestanden van naam te veranderen:
\begin{lstlisting}[language=bash]
$ mv Hallo.txt hallo.txt
\end{lstlisting}
Veranderd de naam.

Voor het weggooien van bestanden gebruiken we \texttt{rm} van remove.
\begin{lstlisting}[language=bash]
$ rm hallo.txt
\end{lstlisting}

Omdat alles een bestand is op een Linux systeem zijn ook directories bestanden, speciale bestanden, maar toch bestanden. We hebben al gezien dat we met \texttt{rmdir} lege directories weg kunnen gooien. Zouden we nu \texttt{rm} kunnen gebruiken om ook directories weg te gooien. Ja, dat kan, maar ook hier geldt dat de directory leeg moet zijn.
\begin{lstlisting}[language=bash]
$ rm Boom
\end{lstlisting}
geeft weer een foutmelding. Het systeem zegt tegen ons dat \texttt{Boom} een directory is. Als we tegen \texttt{rm} vertellen dat hij de zaken recursive weg met gooien, dan zal de hele boomstructuur wegggegooid worden:
\begin{lstlisting}[language=bash]
$ rm -r Boom
\end{lstlisting}


% Requires: fhs
% Provides: vi, vim, nano, pico
\chapter{Het gebruik van een editor}
Op de commandline heb je geen menu's en vaak geen muis om door een applicatie te navigeren. Het maken van documenten is dan ook een stuk lastiger dan in een grafische interface. Toch zijn er oplossingen om op de commandline te werken met bestanden. Een tekstverwerker\index{tekstverwerker} op de commandline heet een editor\index{editor}. Er zijn verschillende editors bedacht en in gebruik. Een van de oudste voor Unix geschreven editors is \texttt{vi}.

Een van de grote namen achter Unix is Ken Thompson. De eerste drie commando's die hij schreef voor het jonge Unix systeem waren \texttt{as} (assembler), \texttt{ed}\index{ed} (editor) en \texttt{sh} (shell). Dennis M. Ritchie bracht verbeteringen aan op \texttt{ed} en vanaf 1969 tot 1976 bleef dit de editor op een Unix systeem. In 1976 kwamen Billy Joy en Chuck Haley met een nieuwe editor die \texttt{ex}\index{ex} werd genoemd. Voor \texttt{ex} schreef Billy Joy ook een soort interface om er makkelijker mee te kunnen werken en die wrapper om \texttt{ex} noemde hij \texttt{vi}\index{vi} (visual interface\index{visual interface}). Vanaf 1979 werd \texttt{ex} geintergreerd in \texttt{vi} en was er alleen nog \texttt{vi}. Later werd \texttt{vi} onderdeel van de Single Unix Specification en daarmee een editor die op bijna elk Unix systeem aanwezig is en dat is nogsteeds het geval. Op bijna alle beschikbare Unix systemen, van BSD tot Linux en Mac OS X is een vorm van \texttt{vi} aanwezig. Dat is dan ook het voordeel van het aanleren van het werken met \texttt{vi}, ja kan je kennis op verschillende platformen gebruiken.

\section{vi, pico, nano}
De eerste redelijk gebruiksvriendelijke editor op Unix was \texttt{vi}. De \texttt{vi} editor kent twee modi. De eerste modus is de \textquote{edit mode} en de tweede is de \textquote{command mode}. Standaard start \texttt{vi} op in de command mode waarin je commando's kunt geven om bestanden te laden of op te slaan en waarin je functies als knippen en plakken kan uitvoeren. De edit modus is die waarin je tekst invoert. Dit onderscheid maakt het voor beginnende gebruikers \texttt{vi} soms verwarrend.

Naast \texttt{vi} zijn er ook andere editors voor Unix-achtige systemen ontwikkeld. De meeste bekende zijn \texttt{pico}\index{pico} en \texttt{nano}\index{nano}. Pico was de oorspronkelijke editor. Nano is ontwikkled door het GNU-project en is een vervanging van pico omdat pico een licentie had die \textquote{problematisch} was. Dat probleem is inmiddels opgelost, maar nano biedt zoveel extra mogelijkheden dat velen de voorkeur geven aan nano.

Het grote voordeel van nano ten opzichte van vi is zijn gebruiksvriendelijke interface. Nano kent geen edit en command mode zoals vi. Nano gebruikt control codes om commando's te geven en is direct beschikbaar voor de invoer van tekst van de gebruiker.

\section{vim}
De naam \texttt{vim}\index{vim} staat voor Vi IMproved\index{Vi IMproved}. Of wel een verbeterde versie van \texttt{vi}. Bram Molenaar, een Nederlandse software ontwikkelaar, schrijft al sinds 1991 aan de code van \texttt{vim} en zijn verbeteringen zijn zo populair dat op alle Linux systemen alleen nog \texttt{vim} ge\"installeerd wordt. Je kan op een Linux systeem \texttt{vim} opstarten als \texttt{vi} waarmee je zoveel mogelijk de functionaliteiten krijgt als het oude \texttt{vi} en je kan \texttt{vim} opstarten als \texttt{vim} waarmee je alle nieuwe toevoegingen van Bram Molenaar en zijn medeontwikkelaars krijgt.

\texttt{vim} is niet een van de makkelijkste editors maar heeft als grote voordeel, zoals eerder gezegd, dat het beschikbaar is op elk willekeurig Unix systeem.

\subsection{vim opstarten}
Het kan zijn dat \texttt{vim} nog niet geinstalleerd is. Mocht dat het geval zijn, installeer dan \texttt{vim} via de packagemanager voor jouw systeem. Voor Debian systemen is dat:\index{vim!installeren}

\begin{lstlisting}[language=bash]
$ sudo apt-get install vim
\end{lstlisting}

De meeste gebruikelijke manier om \texttt{vim} op te starten is door aan \texttt{vim} meteen een bestandsnaam mee te geven:\index{vim!starten met bestandsnaam}
\begin{lstlisting}[language=bash]
$ vim bestand.txt
\end{lstlisting}
Als je klaar bent met het toevoegen van tekst kan je met \textbf{:wq} afsluiten\index{vim!opslaan en afsluiten}. Dit slaat het document op (w)\index{vim!opslaan} en sluit af (q). Wil je de editor verlaten zonder de gemaakte wijzigingen op te slaan, dan gebruikt je \textbf{:q!}\index{vim!afsluiten zonder opslaan}. Het doet een quit (q) zonder verdere vragen stellen.

Een andere manier om vim op te starten is door geen bestandsnaam mee te geven:\index{vim!starten zonder bestandsnaam}
\begin{lstlisting}[language=bash]
$ vim
\end{lstlisting}
de editor weet nu niet onder welke bestandsnaam een bestand opgeslagen moet worden. Bij de write (w) moet je nu dus de bestandsnaam meegeven: \textbf{:w bestand.txt}\index{vim!bestand schrijven} slaat het bestand dat je gemaakt hebt op als bestand.txt. Na deze opdracht kan je met \textbf{:q} vim afsluiten.

Om tekst toe te voegen of te wijzigigen in \texttt{vi} moet je vanuit de command mode\index{command mode} naar de edit mode\index{edit mode} gaan. Hiervoor zijn verschillende commando's beschikbaar. De meest gebruikte zijn \textbf{i} van insert\index{vim!insert} of \textbf{a} van add\index{vim!add}. Om de edit mode\index{edit mode!verlaten} te verlaten gebruik je de \textsc{ESC} toets.

Met het gebruik van het \textbf{i} commando voeg je tekst in v\'o\'or de plek van de cursor. Door gebruik te maken van \textbf{a} voeg je tekst in na de positie van de cursor.

\input{src/vimDelete}
Met het \textbf{x} commando kan je \'e\'en enkel character knippen.

De traditionele manier om een kopie van een stuk tekst te maken is het gebruik van het \textbf{y} commando\index{vim!copy}. Het \textbf{yl} commando kopieert een character, \textbf{yw} kopieert een woord en \textbf{yy} kopieert regels.

Het \textbf{p}\index{vim!paste} commando kan gebruikt worden om text te plakken. Het \textbf{p} commando is plakken achter de positie van de cursor en \textbf{P} is plakken voor de positie van de cursor.

Een speciale functie van vim en niet van vi is het gebruik van \textbf{v} om een visuele selectie\index{vim!visual selection} maken, met de pijltjes toetsen kan je nu bepalen hoe groot de selectie worden moet. Het commando \textbf{v} geeft je de mogelijkheid om de selectie op character niveau te maken. Met \textbf{V} maak je selecties per regel, hier gebruik je de omhoog en omlaag pijltjes toetsen om je selectie groter of kleiner te maken. Als je de selectie gemaakt hebt gebruik je \textbf{y} om te kopie\"eren of \textbf{d} om het stuk te verwijderen.

Met \textbf{u} kan je een wijziging ongedaan maken.


\input{src/vimBewegen}

% Requires: fhs, editor
% Provides: grep, find, globbing, regular expressions, sed
\chapter{Zoeken en vinden}\label{zoekenenvinden}
\input{src/ZoekenVindenIntro}
\section{Globbing}
\input{src/Globbing}
\section{Zoeken naar bestand}
\input{src/find}
Op een Linux systeem kunnen er vele directories zijn en is het van belang dat je op een simpele manier bestanden terug kan vinden. Om te zoeken naar bestanden of directories is er het commando \texttt{find}. Om een idee te krijgen hoe \texttt{find} werkt typen we:

\begin{lstlisting}[language=bash]
$ find ~ -name "Muziek" -print
\end{lstlisting}

Bij mij op het systeem was het antwoord
\begin{lstlisting}[language=bash]
/home/dennis/LinuxCursus/Muziek
\end{lstlisting}

Bij jou is dennis natuurlijk weer vervangen door je eigen gebruikersnaam. Print is de standaard functie van \texttt{find}, en daarom vonden de programmeurs van \texttt{find} dat als je geen opdracht meegeeft dat \texttt{find} dan ook print doet. Dus korter kan het zo:
\begin{lstlisting}[language=bash]
$ find ~ -name "Muziek"
\end{lstlisting}

Die \textasciitilde is makkelijk als je in je home-directory wilt zoeken, maar wat als je in bijvoorbeeld \texttt{/etc/} staat en in die directory wilt zoeken? Daar is ook over nagedacht. We hebben al '..' gezien als een aanduiding voor een lager gelegen directory, maar zo is er ook de '.' als we 'deze'-directory bedoel. En met deze bedoelen we de directory waarin we nu staan. Dus we kunnen ook het volgende doen:

\begin{lstlisting}[language=bash]
$ find . -name "Muziek"
\end{lstlisting}

En voor wie niets tegen typen heeft mag je natuurlijk ook de hele directory opgeven
\begin{lstlisting}[language=bash]
$ find /home/dennis/ -name "Muziek"
\end{lstlisting}

Met -name geven we op dat we naar een bestandsnaam zoeken en een bestandsnaam kan ook een directory zijn. Als we onderscheidt willen maken tussen bestanden en directories kan kunnen we een -type meegeven:
\begin{lstlisting}[language=bash]
$ find . -type d -name "Muziek"
\end{lstlisting}

Zal dezelfde output geven want type d is een directory, maar als we doen
\begin{lstlisting}[language=bash]
$ find . -type f -name "Muziek"
\end{lstlisting}

dan vinden we niets, want het type f is een regulier bestand en Muziek is een directory. Om dat te testen doen we het volgende
\begin{lstlisting}[language=bash]
$ touch ~/LinuxCursus/Documenten/leeg_bestand.txt
$ find . -type f -name "leeg_bestand.txt"
\end{lstlisting}

Met \texttt{touch} kan je een nieuw leeg bestand aanmaken en dat hebben we gedaan en daarna hebben we \texttt{find} naar dat nieuwe bestand laten zoeken. Nu lijkt dit een wat onzinnige actie omdat we al weten waar het bestand is, maar het geeft ons een optie om een andere functie van \texttt{find} te demonstreren, namelijk het zoeken naar lege bestanden op het systeem:
\begin{lstlisting}[language=bash]
$ find . -size 0
\end{lstlisting}
de -size optie geldt alleen voor bestanden, de -empty optie laat naast lege bestanden ook lege directories zien
\begin{lstlisting}[language=bash]
$ find . -empty
\end{lstlisting}

Het \texttt{find} commando kent nog veel meer opties zoals zoeken naar de datum en tijd waarop een bestand is aangemaakt of een moment later of eerder dan een bepaalde datum en tijd. De man-page van \texttt{find} documenteert al deze verschillende opties en op Internet is heel veel uitleg te vinden hoe je \texttt{find} met al deze opties kan gebruiken. Een laatste functie van \texttt{find} willen je nog meegeven. De kan behalve met -print \texttt{find} ook andere dingen laten doen dan de gevonden elementen printen, je kan \texttt{find} ook vertellen om een actie uit te voeren, zoals het deleten van de gevonden elementen:

\begin{lstlisting}[language=bash]
$ find . -size 0 -delete
\end{lstlisting}
een \texttt{ls} van \texttt{LinuxCursus/Documenten/} zal laten zien dat het bestand \texttt{leeg\_bestand.txt} niet meer bestaat.

Pas wel op, dit kan gevaarlijke situaties opleveren:
\begin{lstlisting}[language=bash]
$ find LinuxCursus/ -empty -delete
\end{lstlisting}

Dit commando zoekt in de LinuxCursus naar alle directories en bestanden die leeg zijn. Bestanden hadden we niet meer, maar er stonden nog wel twee directories in (Documenten en Muziek), beide directories waren leeg en werden door \texttt{find} dan ook keurig verwijderd van het systeem. Toen kwam \texttt{find} nog de directory LinuxCursus tegen, en ja die was inmiddels ook leeg, dus heeft \texttt{find} die ook weggegooid!


\section{Zoeken in bestanden}

Soms zou je willen dat je in een bestand kunt zoeken. Natuurlijk kan je met in een tekstverwerker of een editor zoeken in een bestand. Maar wat nu als je niet zeker meer weet in welk bestand het was dat je iets geschreven had. Dat kan zomaar gebeuren als je een boek zoals dit aan het schrijven bent. Dit boek is opgebouwd uit allemaal kleine bestanden die te samen het boek vormen. Op deze manier hou ik de onderwerpen gescheiden en hoef ik niet elke keer te scrollen om bij een ander deel te komen. Ik kan ook twee onderwerpen in twee verschillende terminals te gelijk open hebben staan en zo parallel aan elkaar werken. Dit deel gaat over hoe we in bestanden kunnen zoeken zonder dat we een tekstverwerker open hebben staan met ons document erin.

Er zijn verschillende commando's die we kunnen gebruiken, de meest gebruikte is denk ik \texttt{grep}. Met \texttt{grep} kan door regular expressions te gebruiken zoeken in bestanden. Meer over regular expressions vind je in de volgende sectie. Nu gaan we vooral kijken naar hoe het \texttt{grep} commando werkt.



\texttt{grep} is de Global Regular Expression Parser\index{grep}\index{commando!grep}\index{Global Regular Expression Parser}. De naam \texttt{grep} vindt zijn oorsprong in een zoekopdracht uit \texttt{ed}, een editor: \texttt{g/re/p} wat stond voor zoek door de hele tekst (\textbf{g}lobal) naar deze \textbf{r}egular \textbf{e}xpression en druk deze af (\textbf{p}rint). \texttt{re} werd dan vervangen door een regular expression.

Om te zien hoe \texttt{grep} werkt gaan we eerst een paar bestanden met inhoud aanmaken.

\lstinputlisting[language=bash]{bash/ZoekenGrep.sh}

\section{Regular Expressions}
Regular expressions\index{regular expressions} zijn speciale karakters waarmee je in complexe data-sets kan zoeken. De naam regular expression wordt vaak afgekort tot regexp\index{regexp} of regex\index{regex}. Regular expressions worden gebruikt om patronen te zoeken in data. Wordt dit patroon gevonden dan wordt de gehele regel waarin dit patroon voorkomt afgebeeld op het scherm. Je kan zoeken in bestanden of in de output (stdout) van een commando.

De basis van regular expressions is een aantal karakters met een speciale betekenis:

\begin{tabular}{| l | l |}
\hline
. & Is een willekeurig karakter \\
\hline
? & Is exact 1 karakter \\
\hline
* & Herhaal de voorgaande expressie 0 of meer keren \\
\hline
\textasciicircum & Begin van de regel \\
\hline
\$ & Einde van de regel \\
\hline
$\backslash$ & Escape; speciale karakters worden als echte karakters behandeld \\
\hline
\{\} & Zorg dat dat een regular expression een eenheid (groep) vormt \\
\hline
\end{tabular}

Deze speciale karakters kunnen we gebruiken om regular expressions te maken. We beginnen met de ., ? en *:
\begin{lstlisting}[language=bash]
$ mkdir regex
$ cd regex
$ touch aap.txt
$ touch ap.txt
$ touch pa
$ touch filetxt
$ ls | grep a?*p
$ ls | grep a.*p
\end{lstlisting}
merk op dat het verschil in output van de twee \texttt{ls} commando's zit in het aantal keren dat een letter voor moet komen. De combinatie ?* zegt dat er 1 of meer karakters moeten zijn, terwijl .* zegt dat het 0 of meer karakters moet zijn.

\begin{lstlisting}[language=bash]
$ ls | grep ^a
$ ls | grep a$
\end{lstlisting}
Met het \textasciicircum{} zoeken we vanaf het begin van de regel, dus de regel moet beginnen met een a. Met de \$ zoeken we vanaf het einde, dus de regel moet eindigen met een a.

Als we bestanden willen zien die op .txt eindigen dan willen we dat de punt onderdeel is van onze zoek opdracht en willen we niet dat de punt gezien wordt als een willekeurig karakter. Vergelijk de volgende twee commando's en hun uitkomst:
\begin{lstlisting}[language=bash]
$ ls | grep '.txt'
$ ls | grep '\.txt'
\end{lstlisting}

Let op de ticks om de regular expression. De ticks zorgen ervoor dat de regular expression bij \texttt{grep} terecht komt en niet al door de shell wordt uitgevoerd. De shell is natuurlijk de eerste die de complete commando regel krijgt. Het is de shell die moet beslissen wie welk deel uitvoert. Voor \texttt{ls} is dat niet zo moeilijk  daar wordt de opdracht aan \texttt{ls} overgelaten, maar voor de regular expression bij \texttt{grep} ontstaan er twee mogelijkheden. De shell lost zelf al een deel van de regular expression op en geeft wat er over blijft aan \texttt{grep} of de shell geeft de complete regular expression aan \texttt{grep}. Dat laatste is wat we in dit geval willen. Vergelijk je hiervoor gekregen uitkomsten eens met:
\begin{lstlisting}[language=bash]
$ ls | grep \.txt
\end{lstlisting}
Wat hopelijk opvalt is dat ondanks de $\backslash$ er niet gefilterd wordt op .txt maar op alles dat een willekeurig karakter heeft met daarachter txt. Dat komt omdat de shell de $\backslash$. al matched met de output van \texttt{ls}. Dit is dus wel iets om op te letten met regular expressions. Het is altijd veilig om de regex tussen ticks te zetten. Dan weet je zeker dat de regex bij \texttt{grep} uitkomt en niet bij de shell.

Soms willen we kunnen aangeven hoe vaak een bepaald karakter achter elkaar voorkomt. Daarvoor hebben we de curly braces (accolades).
\begin{lstlisting}[language=bash]
$ ls | grep -E 'a{2}'
\end{lstlisting}
Zoek zelf in de man-page van grep op wat -E betekent.


\section{Zoek en vervang}
We kunnen ook regels zoeken in een bestand die aan een bepaald patroon (regex) voldoen en dan het patroon vervangen door iets anders. Een van de meest gebruikte tools daarvoor heet \texttt{sed}\index{sed}.

\begin{lstlisting}[language=bash]
$ ls -1
$ ls | sed -e 's/aap/noot/'
\end{lstlisting}
overal waar aap stond staat nu noot. Op de disk is er niets gewijzigd, we hebben alleen de output van ls aangepast. We hebben de \textquote{taak} voor \texttt{sed} tussen ticks gezet omdat we zeker willen weten dat deze opdracht bij \texttt{sed} terecht komt en dat niet de shell zich ermee gaat bemoeien.

Om iets meer van \texttt{sed} te leren gaan we eerst een bestand aanmaken. Gebruik hiervoor \texttt{vi}, \texttt{vim} of een andere editor.
\lstinputlisting[language=bash]{data/stageverslag.txt}

In deze tekst zitten wat taalfouten. Die gaan we zoeken en vervangen. We beginnen met het gebruik van Me aan het begin van de eerste twee alinea's. Dat moet natuurlijk Mijn zijn.
\begin{lstlisting}[language=bash]
$ sed -e 's/^Me/Mijn/'
\end{lstlisting}
Op het scherm zien we de verbeterde tekst, op disk staat echter nog de oude tekst. We kunnen de verbeterde tekst natuurlijk met het groter dan teken wegschrijven naar disk met een nieuwe bestandsnaam. Makkelijker is het om de verbeterde tekst in het al bestaande bestand aan te passen:
\begin{lstlisting}[language=bash]
$ sed -ie 's/^Me/Mijn/'
\end{lstlisting}
We zien nu geen output, omdat de tekst op de disk gewijzigd is. Gebruik \texttt{cat} om te zien dat de tekst inderdaad gewijzigd is. Zoek in de man-page op wat de betekenis is van de -i en de -e opties.

Een ander 'me' die fout is is de 'me' voor 'me school' maar de 'me' voor 'me ontwikkelen' is bijvoorbeeld goed. We moeten dus een regular expression schrijven ide alleen matched op 'me school':
\begin{lstlisting}[language=bash]
$ sed -ie 's/me\ school.$/mijn\ school/'
\end{lstlisting}
Let op de $\backslash$ voor de spatie. Een spatie is in de shell een scheidingsteken en dat willen we nu niet. We bedoelen een echte spatie, dus moeten we hem escapen.


% Requires: fhs, ls, grep
% Provides: users, groups, id, adduser, deluser, gpasswd, usermod
%           /etc/passwd, /etc/shadow, /etc/group, /etc/gshadow
% FIXME /etc/gshadow
\chapter{Gebruikers, groepen en rechten}
\input{src/gebruikersgroepenrechten}
\section{Gebruikers en groepen}
Bij het inloggen heb je een gebruikersnaam en wachtwoord opgegeven en bij de installatie heb je ook een wachtwoord moeten opgeven voor de gebruiker root. Op het systeem zijn dus minimaal al twee gebruikers aanwezig. Op een Linux systeem kunnen ook processen een gebruiker hebben. Dus een proces kan onder een bepaalde gebruiker werken zodat andere gebruikers niet bij dit proces kunnen. Processen zijn taken die op de achtergrond draaien zoals bijvoorbeeld een webserver.

De database met gebruikersnamen is een bestand dat staat in de \texttt{/etc} directory. Het bestand heet \texttt{passwd}\index{passwd} en dat kan je bekijken met \texttt{less}.

\begin{lstlisting}[language=bash]
$ less /etc/passwd
\end{lstlisting}

De wachtwoorden staan in een ander bestand, dat heet \texttt{shadow}\index{shadow}. Dit bestand kan je met \texttt{less} niet bekijken, omdat alleen de beheerder (root) hier rechten voor heeft. De wachtwoorden zijn niet leesbaar, maar ge-encrypt, opgeslagen. Met het \texttt{passwd}-commando\index{passwd}\index{commando!passwd} kan je je wachtwoord wijzigen.

Gebruik \texttt{grep}\index{grep} om je eigen gegevens uit \texttt{/etc/passwd}\index{/etc/passwd} te halen:
\begin{lstlisting}[language=bash]
$ grep dennis /etc/passwd
\end{lstlisting}
vervang hierbij \textsl{dennis} door je eigen gebruikersnaam.

De output van de vorige commando zal er ongeveer zo uit zien:
\begin{lstlisting}[language=bash]
dennis:x:1000:1000:Dennis Leeuw,,,:/home/dennis:/bin/bash
\end{lstlisting}
Het \texttt{passwd} bestand is een soort database waarin de verschillende elementen (velden) gescheiden zijn door een :. Computers werken met getallen, het zijn rekenaars, dus elke gebruiker en elke groep heeft een nummer. De gebruikersnaam en groepsnaam zijn er voor ons mensen, de nummers voor de computer. Het nummer voor de gebruiker noemen we het UID of User ID en het getal voor de groep het GID of Group ID.

De verschillende velden van het \texttt{/etc/passwd} bestand bevatten, van voor naar achter, deze informatie:
\begin{enumerate}
	\item gebruikersnaam (login-naam)
	\item dit veld werd vroeger gebruikt voor het wachtwoord, nu staat er altijd een x. Wachtwoorden staan nu in het \texttt{shadow} bestand.
	\item nummerieke ID van de gebruiker (UID: User ID). De UID's 0-999 zijn gereserveerd voor het systeem en 1000 en hoger zijn vrij te gebruiken voor gebruikers. Het root-account heeft altijd UID 0.
	\item nummerieke ID van de primaire groep van de gebruiker (GID: Group ID). De GID's 0-999 zijn gereserveerd voor het systeem.
	\item extra informatie over de gebruiker, met komma's gescheiden. Heet ook wel het GECOS-field en kan dan de volgende informatie bevatten:
		\begin{enumerate}
			\item Volledige naam van de gebruiker
			\item Adres gegevens van de gebruiker (gebouw en kamernummer)
			\item Werk telefoonnummer
			\item Thuis telefoonnummer
			\item Overige contact informatie (fax, prive e-mail adres, pager, social media)
		\end{enumerate}
	\item de home-directory van de gebruiker
	\item de shell die wordt opgestart als de gebruiker inlogt
\end{enumerate}


Elke gebruiker is ook lid van minimaal 1 groep, de primaire groep zoals opgegeven in \texttt{/etct/passwd}. Op sommige systemen is dat de groep \textbf{users} op andere systemen is dat een andere groep. Om te zien van welke groepen je lid bent kan je \texttt{id}\index{id}\index{commandos!id} gebruiken.
\begin{lstlisting}[language=bash]
$ id
\end{lstlisting}
De output geeft weer dat je maar \'e\'en UID hebt en \'e\'en of meer GID's. Je kan dus lid zijn van meer groepen. De eerste groep is de standaard (default) groep waarvan je lid bent.

In de output zie je ook dat elk UID en elke GID eigenlijk een nummer is. Computers kunnen alleen met getallen werken, terwijl wij mensen beter met namen om kunnen gaan. Vandaar dat het besturingssysteem steeds een vertaling maakt van naam naar getal.

Zoekt uit hoe je \texttt{id} alleen het nummerieke ID terug kunt laten geven en hoe alleen de naam van de primaire groep waarvan je lid bent.

De 'database' met groep informatie vind je in \texttt{/etc/group}. Je kan de naam van bijvoorbeeld je primaire groep terug vinden door een \texttt{grep} te doen op :GID:, dus in het voorgaande voorbeeld zou dat betekenen:
\begin{lstlisting}[language=bash]
$ grep :1000: /etc/group
\end{lstlisting}
de dubbele punten om het id is om te voorkomen dat je bijvoorbeeld ook groep ID 10000 of 10001 terug krijgt.

User Personal Group\index{User Personal Group} of UPG\index{UPG} is een gecombineerd gebruik van de user-ID en de group-ID voor het opslaan van data. Er is ook een groepsnaam met de gebruikersnaam.

In de voorgaande voorbeelden zag het gebruik van UPG. De gebruiker dennis heeft een UID van 1000 een GID van 1000 en beide zijn gekoppeld aan de naam dennis. Dus de groepsnaam is gelijk aan de gebruikers naam. De gebruiker is dus niet lid van de groep \textbf{users} waar op andere systemen elke gebruiker lid van is.

Door gebruik te maken van een groep \textbf{users} waar elke gebruiker lid van is maak je het delen van informatie met andere gebruikers een stuk makkelijker omdat iedereen een gedeelde groep heeft. Data die toegankelijk is voor de groep \textbf{users} is dus toegankelijk voor iedereen.

UPG is meer gericht op veiligheid. Omdat elke gebruiker alleen in zijn eigen groep zit kan data alleen gedeeld worden als er een groep aangemaakt wordt waarin bepaalde gebruikers worden toegevoegd. Die groep kan dan data met elkaar delen. Verder is alle data die je maakt alleen toegankelijk voor jezelf.

\section{Werken als root}
De root-gebruiker\index{root-gebruiker} is op een Linux-systeem almachtig. Deze gebruiker mag alles inclusief het systeem stuk maken en dat is helemaal niet zo moeilijk om te doen. Juist omdat root alles mag is het niet verstandig om als de root gebruiker op een systeem te werken. Doe zoveel mogelijk als een normale gebruiker. Pas als het echt niet anders kan doe je het als root.

Inloggen als root zou je eigenlijk nooit moeten doen. Als je iets als root wilt doen gebruik je \texttt{sudo} (Super User Do), zie daarvoor de volgende sectie.

Om te werken als een andere gebruiker is er \texttt{su} (switch user), dit kan je natuurlijk ook gebruiken om root te worden en die verleiding is waarschijnlijk groot. Maar wen jezelf aan om dat niet te doen en \texttt{sudo} te gebruiken om handelingen als root uit te voeren.


\subsection{sudo}
Met het \texttt{sudo}\index{sudo} (super user do\index{super user do}) commando kan je commando's uitvoeren alsof ze van root zijn, je moet daar dan natuurlijk wel de rechten voor hebben niet iedereen mag dat doen. Wie die rechten heeft wordt bepaald door het \texttt{/etc/sudoers} bestand of door bestanden in de \texttt{/etc/sudoers.d} directory of, zoals we eerder gezien hebben, door lid te zijn van de sudo-groep.

Om te zien wat er in de home-directory van root staat kan je het volgende commando geven:
\begin{lstlisting}[language=bash]
$ sudo ls /root
\end{lstlisting}
Hiermee hebben we aangetoond dat we commando's als root kunnen uitvoeren.

\subsection{su}
Met het commando \texttt{su}\index{su} (switch user\index{switch user}) kan je een werken als een andere gebruiker, mits je het wachtwoord weet van die gebruiker.

Om ook alle omgevingsvariabelen van die gebruiker mee te krijgen moet je aan \texttt{su} het min-teken(-) meegeven. Dat ziet er dan zo uit:
\begin{lstlisting}[language=bash]
$ su - mies
\end{lstlisting}

De kans dat mies bestaat op ons systeem is niet zo groot. De gebruiker waarvan we zeker weten dat deze wel bestaat is onze root gebruiker:
\begin{lstlisting}[language=bash]
$ su -
\end{lstlisting}
We hebben bij de switch naar root niet opgegeven dat we de root gebruiker willen worden. Als er bij \texttt{su} geen naam wordt opgeven gaat \texttt{su} ervan uit dat je root wil worden. Met het \texttt{exit} commando kom je weer terug naar je eigen omgeving.

Voor het installeren van software moet je root zijn. Een gewone gebruiker mag dat niet. We gaan nu eenmalig het \texttt{sudo} packet installeren als root gebruiker, zodat we hierna \texttt{sodu} kunnen gebruiken.

\begin{lstlisting}[language=bash]
$ su -
# apt-get install sudo
# usermod -a -G sudo dennis
\end{lstlisting}
We hebben het sudo-packet ge\"installeerd en de gebruiker (dennis, die je natuurlijk weer vervangen hebt door je eigen gebruikersnaam) toegevoegd (-a is append) aan de groep sudo (-G sudo) die rechten heeft om \texttt{sudo} te gebruiken.

Als we met \texttt{exit} de omgeving van root verlaten en weer terug zijn in ons eigen account dan typen we nog een keer \texttt{exit} zodat we uitgelogd zijn. Als we daarna weer inloggen kunnen we met \texttt{id} zien dat we nu een extra (sudo) groep hebben waarvan we lid zijn.



\section{Gebruikersbeheer}
Om gebruikers toe te voegen aan het systeem zijn er twee tools. De eerste is \texttt{useradd}\index{useradd}\index{commandos!useradd} en de andere is \texttt{adduser}\index{adduser}\index{commandos!adduser}. Het programma \texttt{useradd} is een low-level tool, de standaard manier om gebruikers aan te maken is via \texttt{adduser} en dat is dan ook wat we gaan gebruiken.

\begin{lstlisting}[language=bash]
$ sudo adduser eengebruiker
\end{lstlisting}
Om gebruikers te kunnen aanmaken hebben we root-rechten nodig, dus we gebruiken het \texttt{sudo} commando.

Het \texttt{adduser} commando maakt bijna alles automatisch aan. Het enige dat je hoeft te doen is te vertellen wat het wachtwoord van de gebruiker is en welke informatie er in het GECOS-veld terecht moet komen.

Je mag aan \texttt{adduser} ook meegeven dat zaken anders moeten zijn. Bijvoorbeeld dat een gebruiker een andere shell gebruikt dan bash. Dat kan door de optie \texttt{--shell /bin/chsh} mee te geven. Lees de man-page van \texttt{adduser} eens door met wat er nog meer mogelijk is.

Als op het systeem UPG gebruikt wordt dan zal je zien dat er ook gelijk een groep voor de gebruiker aangemaakt is. Gebruik \texttt{id} om te zien wat de ID is van de nieuwe gebruiker.


We gaan nu een extra groep aanmaken\index{addgroup}\index{commandos!addgroup}:
\begin{lstlisting}[language=bash]
$ sudo addgroep specialgroup
\end{lstlisting}
deze groep is nu toegevoegd aan het \texttt{/etc/group} bestand.


Om een gebruiker toe te voegen aan een groep, gebruiken we \texttt{gpasswd}\index{gpasswd}\index{commandos!gpasswd}.  \texttt{gpasswd} is een tool om groepen te beheren.

Om een gebruiker aan een groep toe te voegen gebruiken we:
\begin{lstlisting}[language=bash]
$ sudo gpasswd -a eengebruiker specialgroup
\end{lstlisting}
om diezelfde gebruiker weer uit de groep te halen gebruiken we:
\begin{lstlisting}[language=bash]
$ sudo gpasswd -d eengebruiker specialgroup
\end{lstlisting}


Om de instellingen van een gebruiker te wijzigen gebruiken we \texttt{usermod}\index{usermod}\index{commandos!usermod}.

Er zijn vele zaken die we kunnen instellen voor een gebruiker. De man-page van \texttt{usermod} vermeldt ze allemaal.


Om een groep weer te verwijderen is er \texttt{delgroup}\index{delgroup}. Het verwijderen van een groep betekent dat gebruikers uit de groep verdwijnen, maar eventuele bestanden op disk die van deze groep zijn veranderen niet. Je houdt dus bestanden over die nummeriek nog van de groep zijn, maar waar geen groepsnaam voor is. Dus alleen de owner kan er nog bij (en root natuurlijk).

Het verwijderen van gebruikers kan heel simpel gedaan worden met \texttt{deluser}\index{deluser}. Het heeft echter wel wat gevolgen. Als een gebruiker niet meer bestaat op het systeem dan kunnen er nog wel bestanden zijn die van deze gebruiker waren (zoals in zijn of haar home-directory). Nummeriek zijn deze bestanden dan ook nog van het UID van de voormalige gebruiker. Alleen root kan nog bij deze bestanden, maar wat ook kan gebeuren is dat het UID opnieuw toegewezen wordt aan een nieuwe gebruiker die aangemaakt wordt. Deze gebruiker wordt dan opeens de eigenaar van de nog aanwezige bestanden. Zorg er dus voor dat de bestanden van eigenaar veranderd zijn voordat je een gebruiker verwijderd.


% Requires: fhs, ls, grep, users, groups
% Provides: chown, chmod, access rights
\chapter{Toegangsrechten op bestanden en directories}
Elk besturingssysteem dat meer dan \'e\'en gebruiker kent heeft een manier nodig om ervoor te zorgen dat de twee gebruikers niet bij elkaars bestanden kunnen als ze dat niet willen. Ook moeten gebruikers niet bij de bestanden van de beheerder (root) kunnen komen. Er moet dus door het systeem bijgehouden worden wie welke rechten heeft op een bestand.

Bij het ontwerp van Unix was de centrale gedachte dat alles binnen het systeem gerepresenteerd werd door het idee van een bestand: 'Everything is a file'. Dus niet alleen documenten zijn bestanden, maar ook aangesloten printers, harddisks, etc. Omdat alles een bestand is kan je met rechten op bestanden ook bepalen wie toegang heeft tot bepaalde stukken hardware.

Dit hoofdstuk gaat over de rechten op Unix-achtige systemen zoals Linux.

Met \texttt{ls -l} kan je een lijst van bestanden opvragen die meer weergeeft dan alleen de bestandsnaam. De output van \texttt{ls -l} zou er zo uit kunnen zien:
\begin{lstlisting}[language=bash]
total 31657256
drwxr-xr-x  3 dennis dennis        4096 Jan  9 08:38  Apps
drwxr-xr-x  2 dennis dennis        4096 Jan  8  2020  Desktop
drwxr-xr-x 10 dennis dennis        4096 Jul 13  2022  Documents
drwxr-xr-x  4 dennis dennis       16384 Jan 19 13:21  Downloads
-rw-r--r--  1 dennis dennis          13 Jan 22  2020  hello.txt
-rw-r--r--  1 dennis dennis         131 Jan 22  2020  HELLO.txt
drwx------  2 root   root         16384 Sep 16  2019  lost+found
drwxr-xr-x  5 dennis dennis        4096 Mar 15  2022  Nextcloud
-rw-r--r--  1 dennis dennis      137665 Mar  3  2022  output.pdf
drwxr-xr-x  3 dennis dennis        4096 Apr  5  2022  Pictures
drwxr-xr-x  4 dennis dennis        4096 Sep 30  2019  Projects
drwxr-xr-x  3 dennis dennis        4096 Feb  1  2022  src
-rw-r--r--  1 dennis dennis         420 Mar 31  2020  Teams.txt
drwxr-xr-x  2 dennis dennis        4096 Sep 25  2019  Templates
-rw-r--r--  1 dennis dennis          65 Apr  6  2022  test.py
-rwxr--r--  1 dennis dennis         345 Apr 16  2020  test.sh
drwxr-xr-x  2 dennis dennis        4096 Sep 23 15:32  tmp
\end{lstlisting}

De kolommen van links naar rechts geven weer:
\begin{itemize}
\item De rechten op een bestand,
\item Het aantal links naar dit bestand
\item De eigenaar van het bestand
\item De groepseigenaar van het bestand
\item De grootte van het bestand
\item De maand van de laatste wijziging
\item De dag van de laatste wijziging
\item Het jaar van de laatste wijziging (of het tijdstip van de laatste wijziging als het minder dan een jaar geleden is)
\item De naam van het bestand
\end{itemize}

\section{Bestandstypen}
\section{src/bestandstypen}
\section{De eigenaar}
Met \texttt{chown}\index{chown}\index{commando!chown} kan de eigenaarschap van een bestand wijzigen. Om dit te kunnen testen moeten we een aantal zaken regelen:
\begin{itemize}
\item Maak een groep aan met de naam \textbf{samen}
\item Voeg jezelf en de gebruiker eengebruiker toe aan deze groep
\item Maak een directory aan \texttt{/home/samen}
\end{itemize}

Nu gaan we ervoor zorgen dat de groep samen toegang heeft tot de directory samen en dat jezelf de hoofd eigenaar wordt. We beginnen met het laatste:
\begin{lstlisting}[language=bash]
$ sudo chown $(id -un) /home/samen
\end{lstlisting}
We hebben in dit commando een extraatje toegevoegd. We kunnen een variabele gebruiken om de gebruikers naam in te zetten en die gebruiken bij \texttt{chown}. Er bestaat echter ook een mogelijkheid om in de shell direct een commando aan te roepen en deze als variabele te gebruiken en dat is wat we hier gedaan hebben. We hebben \texttt{id -un} aangeroepen (wat onze gebruikersnaam terug geeft) en deze hebben we gebruikt als optie aan \texttt{chown}. Dus eigenlijk staat er \texttt{sudo chown username /home/samen}. Bekijk met \texttt{ls -l} het resultaat.

Nu gaan we zorgen dat de beide gebruikers gebruik kunnen maken van deze directory. Eerst moeten we zorgen dat de groep \textbf{samen} de groeps-eigenaar wordt van de directory:
\begin{lstlisting}[language=bash]
$ sudo chown .samen /home/samen
\end{lstlisting}
Door een punt voor \textbf{samen} te zetten gevevn we aan dat we de groeps-eigenaarschap willen wijzigen. Bekijk met \texttt{ls -l} het resultaat.

Beide commando's hadden we ook in \'e\'en keer kunnen doen:
\begin{lstlisting}[language=bash]
$ sudo chown dennis.samen /home/samen
\end{lstlisting}


\section{read, write and execute}
De rechten op een bestand zijn opgedeeld in drie blokken. Elk blok kan de waarden r, w, en x bevatten. Er zijn dus in totaal 9 posities (rwxrwxrwx). De mogelijke rechten zijn:
\begin{description}
\item[r] Read
\item[w] Write
\item[x] Execute
\end{description}
Elk blokje heeft zijn eigen betekenis:
\begin{tabular}{ | c | c | c | c | }
\hline
Type & Owner & Group & Other \\
\hline
d & rwx & rwx & rwx \\
\hline
\end{tabular}
Er zijn dus rechten te vergeven voor de eigenaar, voor de groep en voor de rest van de wereld.

Voor normale bestanden geldt dat je met read rechten een bestand mag openen en dus het kunt lezen, met write rechten mag je bestand ook schrijven en dus wijzigen en met execute rechten mag je een bestand opstarten, dat is natuurlijk alleen handig als je een bestand ook daadwerkelijk op mag starten zoals een script of programma.

Voor directories zijn de regels even anders. Met leesrechten mag je zien welke bestanden er in een directory staan en mag je deze lezen, je kan dus \texttt{ls} gebruiken en een bestand openen in de directory, met schrijfrechten mag je nieuwe bestanden aanmaken en met execute het je daadwerkelijk toegang tot de directory kortom je kunt \texttt{cd} gebruiken om in de directory te komen.

Computers zijn slecht in namen, maar goed in nummers, dus de rechten r, w en x moeten omgezet worden naar een getal waarmee de computer kan werken. Omdat computers heel goed zijn in binair zijn de rechten omgezet naar binaire getallen. Een 1 betekent dat het recht gegeven is, een 0 zegt dat het recht niet gegeven is. 101 betekent dus leesrechten en execute-rechten. Deze binaire manier van tellen kan ook decimaal geschreven worden 101 is dan 5. Een blok van rechten voor eigenaar, groep en de wereld zou er zo uit kunnen zien: rwxr-x-r--. Omgerekend naar binair is dat 111 101 100 en dat per stukje omgezet naar decimaal is 754.


Om de rechten op een bestand te wijzigen is er het commando \texttt{chmod}\index{chmod}\index{commando!chmod}. Je kan \texttt{chmod} gebruiken om de rechten op bestanden te wijzigen door gebruik te maken van read, write en execute of door gebruik te maken van de decimale waarden van de rechten. Een voorbeeld van het gebruik van de decimale waarden zou voor de directory \textbf{samen} er zo uit kunnen zien:
\begin{lstlisting}[language=bash]
$ sudo chmod 777 /home/samen
\end{lstlisting}
We hebben nu de eigenaar, de groep en de wereld alle rechten gegeven, dus de beide gebruikers in de groep \textbf{samen} kunnen nu bij alle documenten die ze in deze directory aanmaken.

Helaas hebben we ook alle rechten gegeven aan Other. Dat betekent dat de hele wereld bij alle documenten kan. We kunnen met chmod ook rechten afnemen. Gebruik eens:
\begin{lstlisting}[language=bash]
$ sudo chmod o-x /home/samen
\end{lstlisting}
Na een \texttt{ls -l} zal je zien dat van other (o) de execute-rechten (x) verdwenen zijn. Dat kunnen we ook met meerdere rechten doen:
\begin{lstlisting}[language=bash]
$ sudo chmod g-w,o-rw /home/samen
\end{lstlisting}
We ontnemen hier van other de read en write rechten en van de group rechten verwijderen we de schrijfrechten. Het plus-teken kunnen we gebruiken om rechten toe te kennen:
\begin{lstlisting}[language=bash]
$ sudo chmod g+w /home/samen
\end{lstlisting}
zorgt ervoor dat de groep weer schrijfrechten heeft.

\section{Het 4de-bit}
\index{4th-bit} De rechten r, w en x zijn elke keer blokjes van 3-bits. Het totaal is dus 3x3 is 9 bits lang. Dat is een raar getal in de computerwereld en dan ook niet helemaal correct, eigenlijk is het blok 12-bits lang en bestaat het uit 4x3 bits. De triplet 777 is dus eigenlijk een quadlet 7777. De eerste 7 wordt gebruikt om extra zaken in te coderen. De rechten op een bestand zonder extra functionaliteit is dus eigenlijke 0777, maar daar laten we de 0 meestal weg.

Met de extra rechten kunnen we de volgend functionaliteit weergeven:
\begin{itemize}
\item SUID bit
\item SGID bit
\item Sticky bit
\end{itemize}


\subsection{SUID-bit}
Het SUID-bit\index{SUID-bit} is het meest linkse bit uit de reeks en heeft dus een decimale waarde van 4. Het is de bit die hoort bij de user dus kan je het SUID-bit ook setten door \texttt{u+s} te gebruiken. Voorbeelden:
\begin{lstlisting}[language=bash]
$ mkdir SUID.d
$ touch SUID.txt
$ chmod u+s SUID.d
$ chmod 4777 SUID.txt
$ ls -ld SUID*
drwsr-xr-x 1 dennis dennis  4096 Feb  7  2022 SUID.d
-rwsrwxrwx 1 dennis dennis 63960 Feb  7  2022 SUID.txt
\end{lstlisting}
Het eerste rwx blokje is nu veranderd in rws om aan te geven dat het SUID-bit gezet is.

Het bit op een bestand zorgt ervoor dat, als je het bestand kunt opstarten, de applicatie draait onder de username van de eigenaar. Dus als een bestand als eigenaar heeft root.admin en het SUID-bit is gezet dan zal bij opstarten het programma draaien met root-rechten. Het voordeel is dat gewone gebruikers zo programma's kunnen opstarten met rechten die ze normaal niet hebben. Het nadeel is dat er een security-lek zou kunnen ontstaan, dus je moet heel voorzichtig zijn met deze rechten.

Een voorbeeld op een Linux systeem waar het SUID-bit gebruikt wordt is op het \texttt{passwd} programma. Een gebruiker moet instaat zijn om zijn wachtwoord te wijzigen, terwijl het \texttt{/etc/shadow} bestand alleen lees en schrijfbaar is door root.

Het bit op een directory heeft geen betekenis in GNU/Linux.


\subsection{SGID-bit}
Het SGID-bit\index{SGID-bit} is het tweede bit uit de reeks en is dus decimaal: 2. Het behoort bij de groepsrechten en kan dus gezet worden met \texttt{g+s}. Voorbeelden
\begin{lstlisting}[language=bash]
$ mkdir SGID.d
$ touch SGID.txt
$ chmod g+s SGID.d
$ chmod 2777 SGID.txt
$ ls -ld SGID*
drwxr-sr-x 2 dennis dennis 4096 Jan 24 09:28 SGID.d
-rwxrwsrwx 1 dennis dennis    0 Jan 24 09:28 SGID.txt
\end{lstlisting}

Het SGID-bit op een bestand zet de effectieve groep waaronder een applicatie draait. Dus net als wat de SUID-bit doet voor de gebruiker doet het SGID-bit voor de groep,

Het SGID-bit op directories betekent dat de groep eigenaarschap wordt doorgegeven aan nieuwe bestanden of directories die aangemaakt worden in de directory. Dus als we een directory hebben met het SGID-bit en de groepseigenaar van de directory is \textbf{samen} en we maken in die directory een nieuwe bestand aan dan wordt de groepseigenaar weer \textbf{samen} ongeacht de groep waarin de gebruiker zit die het bestand aanmaakt. Natuurlijk moet een van de groepen van die gebruiker wel \textbf{samen} zijn. 


\subsection{Sticky-bit}
Het Sticky-bit\index{Sticky-bit} kan gezet worden met de decimale waarde 1 of met \texttt{o+t}. Voorbeelden:
\begin{lstlisting}[language=bash]
$ mkdir sticky.d
$ touch sticky.txt
$ chmod o+t sticky.d
$ chmod 1777 sticky.txt
$ ls -dl sticky*
drwxr-xr-t 2 dennis dennis 4096 Jan 24 10:04 sticky.d
-rwxrwxrwt 1 dennis dennis    0 Jan 24 10:04 sticky.txt
\end{lstlisting}
Bij het rechtenblok van other is de x vervangen door een t.

Voor bestanden heeft het Sticky-bit in Linux geen betekenis.

Als het Sticky-bit gezet is op een directory dan betekent dat alleen de eigenaar, de eigenaar van de directory en root het bestand kunnen weggooien en hernoemen. Zelfs als iemand in de juiste groep zit en de groep heeft schrijfrechten dan nog heeft die persoon niet de rechten. Een voorbeeld van het gebruik van het Sticky-bit is het \texttt{/tmp} directory.



% Requires: sudo, cat
% Provides: lscpu, free, dmidecode, uname
% lspci, lsusb, lsscsi, lsblk, du, df, fdisk
\chapter{Systeem inventarisatie}
\input{src/inventarisatie}
\section{System versienummers}
\input{src/versienummers}
\subsection{Distributie versie}
\input{src/distro_version}
\subsection{Kernel versie}
\input{src/kernel_version}
\section{Het moederbord}
\subsection{CPU}\index{CPU}\index{Hardware informatie!CPU}
\input{src/cpu_info}
\subsection{RAM}\index{RAM}\index{Hardware informatie!RAM}
\input{src/ram_info}
\subsection{BIOS}\index{BIOS}\index{Hardware informatie!BIOS}
\input{src/bios_info}
\section{Extensie bussen}
\subsection{PCI}\index{PCI}\index{Hardware informatie!PCI}
\input{src/pci_info}
\subsection{USB}\index{USB}\index{Hardware informatie!USB}
\input{src/usb_info}
\subsection{SCSI}\index{SCSI}\index{Hardware informatie!SCSI}
\input{src/scsi_info}
\section{Harddisks}
\input{src/disks}
\input{src/disks-fs}
\input{src/disks-partition}
\input{src/disk-free}
\input{src/disk-usage}
\section{Netwerk}
\input{src/netwerk_hardware}
\section{Inventarisatie opdracht}
\input{src/inventarisatie_opdracht}

% Provides: journalctl, ps, dmesg, top, free, last
% uptime, syslog, kill
\chapter{Systeembeheer}
\input{src/systeembeheer}
\section{CPU gebruik}
\input{src/uptime}
\section{Geheugen gebruik}
\input{src/geheugengebruik}
\section{Aanwezige gebruikers}
\input{src/usermonitoring}
\section{Processen}
\input{src/processen_ps}
\input{src/processen_kill}
\input{src/processen_top}
\section{Logging}
\input{src/logging}
\subsection{syslog}
\input{src/syslog}
\subsection{kernel berichten}
\input{src/logging_kernel}

% Requires: lsblk
% Provides: mount, umount
\chapter{File systems}
Om bestanden naar een disk te kunnen schrijven moet er een soort van database bijgehouden worden met op welke track en sector bij het bestand horen, maar ook gegevens als van wie is het bestand en wie heeft er toegang tot het bestand. Al deze gegevens worden bijgehouden door het file system\index{file system}.

In de loop der tijd zijn er verschillende bestandssystemen bedacht voorbeelden zijn FAT, ExtFAT, FAT32, NTFS, ext3, ext4, HFS+ en nog vele anderen. Sommige besturingssystemen ondersteunen alleen hun eigen bestandssysteem, Linux ondersteunt er heel veel. Je kan in Linux vaak dus schijven van andere OSen lezen en schrijven.

\section{Formatting file systems}
Om een disk voor te bereiden op het ontvangen van data moet de disk eerst voorzien worden van de nodige structuren die uniek zijn voor het specifieke bestandssysteem. Het aanbrengen van deze structuren heet het formateren\index{formateren} van de disk. Voor elke bestandssysteem heb je dan ook een eigen format tool.

De format tool op een Linux systeem heet \texttt{mkfs}\index{mkfs}\index{commando!mkfs} (make filesystem). Voor de verschillende file systems is er een eigen tool, die begint met mkfs, dan een punt en dan de naam van het bestandssysteem zoals Linux het kent. Voorbeelden: \texttt{mkfs.exfat}, \texttt{mkfs.ext3}, \texttt{mkfs.fat}, \texttt{mkfs.msdos}, \texttt{mkfs.vfat}.


\section{Mounten van lokale bestandssystemen}
Wat je zal zijn opgevallen als je een Windows-gebruiker bent is dat Linux geen drive letters heeft. In Linux is het hele systeem \'e\'en groot file system waarbij de verschillende disks of partities gemount worden op een directory in het file system.
\begin{lstlisting}[language=bash]
$ lsblk
NAME   MAJ:MIN RM   SIZE RO TYPE MOUNTPOINT
sda      8:0    0 931.5G  0 disk 
|-sda1   8:1    0 931.5G  0 part /home/dennis
sdb      8:16   1  59.8G  0 disk 
|-sdb1   8:17   1  58.8G  0 part /
|-sdb2   8:18   1     1K  0 part 
|_sdb5   8:21   1   976M  0 part [SWAP]
\end{lstlisting}


\section{Mounten van disks}
\input{src/MountenDisks}

\chapter{Netwerk configuratie}

\chapter{Software installeren}

\chapter{Programmeertalen}
Van oudsher werden Unix systemen veel gebruikt door programmeurs. Er zijn dan ook veel talen ontwikkeld en overgezet naar Linux. Natuur is er een C-compiler. De meest gebruikte is die uit het GNU project die de GNU Compiler Collection (GCC) heet omdat hij naast \index{C}C ook compilers bevat voor \index{C++}C++, \index{Objective-C}Objective-C, \index{Fortran}Fortran, \index{Ada}Ada, \index{Go}Go en \index{D}D.

Ook voor scripting talen zijn er veel \foreignlanguage{english}{interpreters} aanwezig zoals voor PHP, Perl, Python en Java.

Daarnaast is er via verschillende kanalen nog veel meer te installeren.

In het document Linux Introductie (ook beschikbaar via https://github.com/DennisLeeuw/Linux) staat beschreven hoe je van C-source code naar een vorm komt die de CPU begrijpt. Dat proces heet compileren. Je gebruikt een compiler om bron-code om te zetten naar binaire code.

Er is nog een andere manier om een programma te draaien op je computer. Je kan ook de broncode omzetten naar binaire code via een runtime engine. Programmeertalen die hiervan gebruik maken heten scripting talen.

Het belangrijkste verschil tussen de twee vormen is de manier waarop een applicatie verspreid wordt. Als je de broncode compileert en er een binairbestand van maakt dan kan je alleen het binaire bestand delen en iedereen met dezelfde OS-versie en hardware kan jouw software dan gebruiken. Dit is wat bijvoorbeeld Apple en Microsoft doen en hoe veel van ook de Linux software verspreid wordt via de repositories. Als je software een script is dan is de verspreiding per definitie als broncode. Een shell-script, JavaScript, PHP of Python wordt bijna altijd als broncode verspreid en een runtime engine wordt gebruikt om het script te starten op de computer.

Scripting had de naam om traag te zijn, maar dat is bij moderne scripting talen zoals Python inmiddels bijna niet meer het geval.


We kunnen programeertalen grofweg opdelen in twee soorten:
\begin{itemize}
\item Procedural Programming
\item Object Oriented Programming (OOP)\index{OOP}
\end{itemize}

Een Procedural\index{Procedural Programming} programming language is gebaseerd op het gebruik van procedure aanroepen (procedure calls). Een Procedure is een routine of subroutine, misschien beter bekent als functies. Een functie of routine is een blok met commando's die bij elkaar horen en die vanuit het hoofdprogramma \'e\'en of meer keren aangeroepen kan worden.

Een Object Oriented\index{Object Oriented Programming} language is een taal die is gebaseerd op Objecten. Een object is programma-code met data. Waar een functie alleen de programma code bevat bevat een object ook de data. De code bestaat net als bij Procedural languages uit procedures, maar heten dan methods en de data kan aangesproken worden als attributes of fields (velden). Er is gebleken dat veel vertalingen van functies in de wereld die we willen automatiseren zich makkelijker laten vertalen in objecten.

\section{C}
C\index{C} is een procedurele programmeertaal. Hij is ontworpen bijna gelijk met het ontstaan van Unix. De ontwerpers van C zijn Dennis Ritchie en Brian Kernighan, de taal stamt al uit 1969 en zijn voorlopers waren daadwerkelijk de talen 'A' en 'B'.

Een voorbeeld van een programma geschreven in C is Hello World uit het C programmeer boek. Dit is het eerste programma dat je als voorbeeld kreeg. Dit heeft veel navolging gehad in andere talen. Wij zullen bij elke taal die we hier noemen een voorbeeld geven van Hello World.
\lstinputlisting[language=C]{c/helloworld.c}

Voor C heb je een compiler nodig om de broncode om te zetten in iets dat de computer begrijpt:
\begin{lstlisting}[language=bash]
$ gcc hello.c
$ ./a.out
\end{lstlisting}


\section{C++}
C++\index{C++} is een Object Oriented language. C++ heeft een compiler nodig om van de broncode iets te maken dat de computer begrijpt.

Een voorbeeld van het Hello World programma in C++:
\lstinputlisting[language=C++]{c++/helloworld.cpp}

Voor de vertaling maar machinetaal gebruiken we \texttt{gcc}:
\begin{lstlisting}[language=bash]
$ gcc helloworld.cpp
$ ./a.out
\end{lstlisting}

\section{Perl}
Perl\index{perl}\index{scripting!perl} is begonnen als procedurele taal die voornamelijk gebruikt werd voor het verwerken van data. Later is er ook de mogelijkheid toegevoegd om object georienteerd te kunnen werken. Perl is een scripting taal die ook hoofdzakelijk zo gebruikt wordt.

\lstinputlisting[language=Perl]{perl/helloworld.pl}

We gebruiken \texttt{perl} als runtime engine om het perl-script te draaien:
\begin{lstlisting}[language=bash]
$ perl helloworld.pl
\end{lstlisting}

\section{PHP}
PHP\index{PHP}\index{scripting!PHP} is \textbf{de} programmeer taal van het Internet (geworden). PHP draait op de webserver en genereert de output die door webservers aan de aanvrager gegeven wordt. Je kan PHP echter ook op de commandline gebruiken zonder web output.

Een PHP-script dat je op de commandline kan gebruiken:
\lstinputlisting[language=PHP]{php/helloworld.php}

Om het op te starten hebben we de PHP runtime engine nodig:
\begin{lstlisting}[language=bash]
$ php helloworld.php
\end{lstlisting}

Maar PHP is eigenlijk bedoelt voor op het web, dat betekent dat er een webpagina moet komen met daarop Hello, World!:
\lstinputlisting[language=PHP]{php/helloworld_web.php}

Je kan dat ook op de commandline testen:
\begin{lstlisting}[language=bash]
$ php helloworld_web.php
\end{lstlisting}

\section{Python}
Python\index{python}\index{scripting!python} is een Object Oriented scripting taal.

Voorbeeld van een Python:
\lstinputlisting[language=Python]{python/helloworld.py}

We gebruiken de \texttt{python} of \texttt{python3} runtime engine om het script te draaien:
\begin{lstlisting}[language=bash]
$ python3 helloworld.py
\end{lstlisting}

\section{Java}
Java is een object oriented taal\index{java} die gebruikt kan worden als gecompileerde applicatie of als scripting taal. Op de meeste linux distributies is Java niet standaard ge\"installeerd. Installeer de Oracle versie of OpenJDK als je de volgende oefeningen wil kunnen doen.

\lstinputlisting[language=Java]{java/helloworld.java}

Je kan direct de java runtime engine gebruiken:
\begin{lstlisting}[language=bash]
$ java helloworld.java
\end{lstlisting}

Om java te kunnen compileren heb je de \texttt{javac} compiler nodig:
\begin{lstlisting}[language=bash]
$ javac helloworld.java
\end{lstlisting}



\chapter{Shell scripting}\index{Shell scripting}
\input{src/ScriptingIntro}
\section{Waarom scripting?}
\input{src/ScriptingGebruik}
\section{Hello World - een eerste script}
\input{src/ScriptingHelloWorld}
\section{Het starten van scripts}\index{Shell scripting!Starten}
\input{src/ScriptingOpstarten}
\section{Commentaar}\index{Shell scripting!commentaar}
\input{src/ScriptingCommentaar}
\section{Variabelen}\index{Variabelen, shell scripting}\index{Shell scripting!Variabelen}
\input{src/ScriptingVariabelen}
\section{if}\index{if}\index{Shell scripting!if}
\input{src/ScriptingIf}
\subsection{Opdracht: Werken met if}
\input{src/ScriptingIfOpdracht}
\section{for}\index{for}\index{Shell scripting!for}
\input{src/ScriptingFor}
\subsection{Opdracht: Werken met for}
\input{src/ScriptingForOpdracht}
\section{while}\index{while}\index{Shell scripting!while}
\input{src/ScriptingWhile}
\subsection{Opdracht: Werken met While}
\input{src/ScriptingWhileOpdracht}
\section{Opdracht: Gemiddelde cijfer van een student}
\input{src/ScriptingKlasOpdracht}

%%%%%%%%%%%%%%%%%%%%%
%%% Index and End %%%
%%%%%%%%%%%%%%%%%%%%%
\backmatter
\printindex
\end{document}

%%% Last line %%%
