Het kan zijn dat \texttt{vim} nog niet geinstalleerd is. Mocht dat het geval zijn, installeer dan \texttt{vim} via de packagemanager voor jouw systeem. 

De meeste gebruikelijke manier om vim op te starten is door aan vim meteen een bestandsnaam mee te geven:
\begin{lstlisting}[language=bash]
$ vim bestand.txt
\end{lstlisting}
Als je klaar bent met het toevoegen van tekst kan je met \textbf{:wq} afsluiten. Dit slaat het document op (w) en sluit af(q). Wil je de editor verlaten zonder de gemaakte wijzigingen op te slaan, dan gebruikt je \textbf{:q!} om dat te doen. Het doet een quit (q) zonder verdere vragen stellen.

Een andere manier om vim op te starten is door geen bestandsnaam mee te geven:
\begin{lstlisting}[language=bash]
$ vim
\end{lstlisting}
de editor weet nu niet onder welke bestandsnaam een bestand opgeslagen moet worden. Bij de write (w) moet je nu dus de bestandsnaam meegeven: \textbf{:w bestand.txt} slaat het bestand dat je gemaakt hebt op als bestand.txt. Na deze opdracht kan je met \textbf{:q} vim afsluiten.
