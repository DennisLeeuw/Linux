%\begin{figure}
%\includegraphics[width=0.4\linewidth]{linuxreader-img028.png}
%	\caption{Het zoeken van informatie}
%\end{figure}

Om uit te vinden welk commando je kan gebruiken om iets op het systeem te bereiken kan je \texttt{apropos}\index{apropos}\index{Zoeken!apropos} gebruiken. Type eens (Let op de quotes!):
\begin{lstlisting}[language=bash]
$ apropos 'make directories'
\end{lstlisting}
je vindt dan het \texttt{mkdir} commando. Het nadeel van dit zoek systeem is dat het heel specifiek is.

\begin{lstlisting}[language=bash]
$ apropos 'make directory'
\end{lstlisting}
doet niets. Als je het dus niet meteen vindt probeer dan enkelvoud- en meervoudsvormen.

Heb je een commando gevonden waarvan je denkt dat het is wat je nodig hebt probeer dan \texttt{whatis}\index{whatis}:

\begin{lstlisting}[language=bash]
$ whatis mkdir
\end{lstlisting}

Dit geeft een korte beschrijving van een commando. Voor de volledige manual gebruiken we het natuurlijk \texttt{man}:
\begin{lstlisting}[language=bash]
$ man mkdir
\end{lstlisting}

