Het kan voorkomen dat een stukje software dat jij wilt gebruiken niet beschikbaar is vanuit een repository, of dat je een nieuwere versie wilt hebben dan er vanuit een repository beschikbaar is. Als je de software niet zelf wilt compileren, dan zal je opzoek moeten naar iemand die een binary voor je heeft. Daar zijn natuurlijk risico's aan verbonden, want je weet niet zeker wat diegene ermee gedaan heeft, dus download alleen binaire software van vertrouwde bronnen (websites).

Er zijn verschillende manieren waarop een binary kan worden aangeboden. Hij kan gecompiled zijn voor jouw distributie en dus aangeboden worden als .deb of .rpm package. Het kan echter ook op een andere manier waardoor de software in een bundel geleverd wordt die zelfstandig kan draaien, zoals AppImage of Flatpak. Het kan ook geleverd worden als een container zoals Docker.
