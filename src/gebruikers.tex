Bij het inloggen heb je een gebruikersnaam en wachtwoord opgegeven en bij de installatie heb je ook een wachtwoord moeten opgeven voor de gebruiker root. Op het systeem zijn dus minimaal al twee gebruikers aanwezig. Op een Linux systeem kunnen ook processen een gebruiker hebben. Dus een proces kan onder een bepaalde gebruiker werken zodat andere gebruikers niet bij dit proces kunnen. Processen zijn taken die op de achtergrond draaien zoals bijvoorbeeld een webserver.

De database met gebruikersnamen is een bestand dat staat in de \texttt{/etc} directory. Het bestand heet \texttt{passwd}\index{passwd} en dat kan je bekijken met less.

\begin{lstlisting}[language=bash]
$ less /etc/passwd
\end{lstlisting}

De wachtwoorden staan in een ander bestand, dat heet \texttt{shadow}\index{shadow}. Dit bestand kan je met \texttt{less} niet bekijken, omdat alleen de beheerder (root) hier rechten voor heeft. De wachtwoorden zijn niet leesbaar, maar geencrypt, opgeslagen. Met het \texttt{passwd}-commando\index{passwd}\index{commando!passwd} kan je je wachtwoord wijzigen.

Gebruik \texttt{grep}\index{grep} om je eigen gegevens uit \texttt{/etc/passwd}\index{/etc/passwd} te halen:
\begin{lstlisting}[language=bash]
$ grep dennis /etc/passwd
\end{lstlisting}
vervang hierbij \textsl{dennis} door je eigen gebruikersnaam.

De output van de vorige commando zal er ongeveer zo uit zien:
\begin{lstlisting}[language=bash]
dennis:x:1000:1000:Dennis Leeuw,,,:/home/dennis:/bin/bash
\end{lstlisting}
Het is een soort database waarin de verschillende elementen gescheiden zijn door een :.
\begin{enumerate}
	\item gebruikersnaam (login-naam)
	\item werd vroeger gebruikt voor het wachtwoord, nu altijd een x. Wachtwoorden staan nu in het \texttt{shadow} bestand.
	\item nummerieke ID van de gebruiker (UID: User ID). De UID's 0-999 zijn gereserveerd voor het systeem en 1000 en hoger zijn vrij te gebruiken voor gebruikers. Het root-account heet altijd UID 0.
	\item nummerieke ID van de primaire groep van de gebruiker (GID: Group ID). De GID's 0-999 zijn gereserveerd voor het systeem.
	\item extra informatie over de gebruiker, met komma's gescheiden. Heet ook wel het GECOS-field en kan dan de volgende informatie bevatten:
		\begin{enumerate}
			\item Volledige naam van de gebruiker
			\item Adres gegevens van de gebruiker (gebouw en kamernummer)
			\item Werk telefoonnummer
			\item Thuis telefoonnummer
			\item Overige contact informatie (fax, prive e-mail adres, pager, social media)
		\end{enumerate}
	\item de home-directory van de gebruiker
	\item de shell die wordt opgestart als de gebruiker inlogt
\end{enumerate}
