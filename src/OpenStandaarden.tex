Een verwarring die weleens wil ontstaan is het verschil tussen open source en open standaarden en toch is daar een
wezenlijk verschil.\par

Open standaarden beschrijven hoe bijvoorbeeld data uitgewisseld kan worden. Protocollen als SMTP, POP3, IMAP, Telnet en
FTP zijn allemaal beschreven in documenten die vrij op Internet toegankelijk zijn. Iedereen, dus ook de open source
wereld, kan deze standaarden implementeren en er dus voor zorgen dat verschillende systemen, open source en
commercieel, met elkaar kunnen communiceren. Gesloten protocollen die door een bedrijf zijn bedacht kunnen alleen door
dat bedrijf gebruikt worden, hoewel door luisteren op het netwerk er natuurlijk ook gekeken kan worden hoe het protocol
werkt.

Ook voor het uitwisselen van data is het van belang dat er open standaarden zijn. Het feit dat je een document dat je in Microsoft Word maakt alleen in Word kan lezen is natuurlijk een enorme beperking van je vrijheid. Je document zou in elke willekeurige tekstverwerker te openen en wijzigen moeten zijn, het is tenslotte jouw tekst. Toch is er pas sinds 2005 een open standaard voor office documenten. In 2005 werd door Organization for the Advancement of Structured Information Standards (OASIS) de OpenDocument standaard goedgekeurd. Deze standaard beschrijft hoe office documenten eruit moeten zien en omdat het een open standaard is kan iedereen de standaard implementeren, daarmee zou het mogelijk moeten zijn dat elke tekstverwerker een OpenDocument tekstdocument moet kunnen lezen en schrijven.
