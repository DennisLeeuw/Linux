Zorg dat je in de directory LinuxCursus staat en type:

\begin{lstlisting}[language=bash]
$ touch hello.txt
\end{lstlisting}

Na de Enter lijkt er helemaal niets te gebeuren. Dit is met de meeste Linux commando's het geval. Als het goed gegaan is
dan laten ze niets weten, een beetje als ``geen nieuws, is goed nieuws''. Doen we een `ls' dan zien we dat er een
bestand is aangemaakt dat hello.txt heet.

Met touch kunnen we dus bestanden aanmaken, dit zijn lege bestanden. Type maar eens:

\begin{lstlisting}[language=bash]
$ cat hello.txt
\end{lstlisting}

dan zal je zien dat er weer niets op je scherm verschijnt. En dat is goed! Het `cat' commando plaatst de inhoud van een
bestand op het scherm en daar we een leeg bestand hebben opgevraagd is wat er op het scherm komt dus niets en omdat dat
succesvol is verlopen hoeft cat ook geen foutmelding te laten zien en met de wetenschap dat geen nieuws, goed nieuws is
is cat klaar.

In de acties die we nu hebben doorlopen staat er dat echo tekst naar het scherm schrijft en cat
bestanden op het scherm afbeeldt. Dat is inderdaad wat er gebeurt, maar vanuit het Linux gezien niet helemaal correct.
Zowel echo als cat schrijven naar de standaard output en in de terminal is het scherm de standaard output. De standaard
output wordt vaak afgekort als stdout.

We kunnen de standaard output ook omleiden (redirect) naar bijvoorbeeld een bestand:

\begin{lstlisting}[language=bash]
$ echo 'Ik werk met Linux' > hello.txt
\end{lstlisting}

We zien nu dat de zin die we met echo afbeelden niet meer op het scherm verschijnt. Hij is verdwenen en er lijkt weer
helemaal niets gebeurd te zijn. Als we nu

\begin{lstlisting}[language=bash]
$ cat hello.txt
\end{lstlisting}

doen dan zien we waar onze zin is gebleven. Hij is hello.txt terecht gekomen. We hebben de stdout van echo in hello.txt
gestopt.

Laten we dat nog eens doen:

\begin{lstlisting}[language=bash]
$ echo 'Hello World!' > hello.txt
\end{lstlisting}

Doen we een cat van hello.txt dan zien we dat onze eerste zin verdwenen is en er alleen nog 'Hello World!' in hello.txt
zit. We hebben kennelijk ons bestand overschreven met nieuwe inhoud. We kunnen ook tekst toevoegen aan een bestand:

\begin{lstlisting}[language=bash]
$ cat 'Ik werk met Linux' >> hello.txt
\end{lstlisting}

door gebruik te maken van het dubbele groter dan teken voegen we een regel toe aan het eind van het bestand. De oude
regel zie je met cat als eerste en daaronder komt onze nieuwe regel.

Zou er als we een stdout hebben ook een standaard input zijn en kunnen we daar dan van lezen? Ja, die is er. Als je
typt:

\begin{lstlisting}[language=bash]
$ cat < hello.txt
\end{lstlisting}

dan vertellen we eigenlijk dat cat de invoer (stdin) op het scherm moet afbeelden. Maar ja, dan moeten we een spatie en
een <-teken extra typen en dat doen we liever niet.

Naast de standaard input en standaard output is er ook nog standaard error (stderr), waar de foutmeldingen naartoe gaan.
Laten we eens een fout maken door een niet bestaan bestand aan cat te geven:

\begin{lstlisting}[language=bash]
$ cat Hello.txt
\end{lstlisting}

Je krijgt nu een foutmelding dat Hello.txt niet bestaat. Linux is case-sensitive dus hello.txt is niet hetzelfde als
Hello.txt. De standaard output, input en error zijn genummerd in Linux. Stdin is 0, stdout is 1 en stderr is 2. Nu we
dit weten zouden we het volgende kunnen doen:

\begin{lstlisting}[language=bash]
$ cat Hello.txt 2> Hello_error.txt
\end{lstlisting}

We zien nu geen foutmelding meer op ons scherm en hebben de stderr omgeleid (redirect) naar het bestand
Hello\_error.txt. Doen we nu een
\begin{lstlisting}[language=bash]
$ cat Hello_error.txt
\end{lstlisting}
dan zien we dat de foutmelding daar is opgeslagen.

Deze vormen van het redirecten (omleiden) van data stromen gebeurt in Linux heel vaak. Programma's schrijven
bijvoorbeeld de fouten die ze tegen komen naar een log bestand. Dit doen ze door de stderr te redirecten en de error
regels toe te voegen aan het bestand, bijvoorbeeld 2>>error.log. Als ze dit doen met een
datum- en tijdmelding dan kan je heel makkelijk problemen opzoeken.

Wij hebben nu geleerd om bestanden aan te maken en om invoer en uitvoer te redirecten.

