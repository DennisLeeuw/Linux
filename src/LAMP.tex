LAMP is een afkorting voor Linux, Apache, MySQL, PHP. Het is een veel gebruikte 'stack' voor het opzetten van webservers op het Internet. Linux is hierbij het operating systeem, Apache de webserver, MySQL de database en PHP de taal die ervoor zorgt dat de browser HTML pagina's met content krijgt, ofwel de server-side scripting language.

In de loop van de tijd zijn er veel variaties op deze afkorting gekomen door vervanging van bijvoorbeeld Linux door Windows (WAMP), of door MacOS X (MAMP), maar er kan ook een andere databases gebruikt worden zoals MariaDB dat een vervanging is voor MySQL, maar we zouden bijvoorbeeld ook PostgreSQL kunnen gebruiken (LAPP) of een andere webserver zoals bijvoorbeeld NGINX (waarbij we de E van engine gebruiken) LEMP, en de P kan ook Perl of Phyton zijn. Wij houden het bij de traditionele stack van LAMP maar vervangen MySQL door MariaDB omdat dat op de meeste systemen aanwezig is en bijna volledig compatible is met MySQL.

Apache, MariaDB en PHP vormen samen de Middleware in onze PAAS (Platform As A Service) omgeving.

