LAMP is een afkorting voor Linux, Apache, MySQL, PHP. Het is een veel gebruikte 'stack' voor het opzetten van webservers op het Internet. In de loop van de tijd zijn er veel variaties op deze afkoring gekomen door vervanging van bijvoorbeeld Linux door Windows (WAMP), of door MacOS X (MAMP), maar er kan ook een andere databases gebruikt worden in plaats van MariaDB (was MySQL). We zouden bijvoorbeeld ook PostgreSQL kunnen gebruiken (LAPP) of andere webservers bijvoorbeeld NginX (waarbij we de E van engine gebruiken) LEMP, en de P kan ook Perl of Phyton zijn. Wij houden het bij de traditionele stack van LAMP, namenlijk Linux, Apache, MariaDB en PHP.

Apache, MariaDB en PHP vormen samen de Middleware in onze PAAS omgeving. 
