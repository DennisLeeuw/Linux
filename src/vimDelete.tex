Vanuit de command modus kan je het character waarop je staat verwijderen door gebruik te maken van het \textbf{x}\index{vim!cut character} commando, eigenlijk is dit het knippen commando, maar kan gebruikt worden om snel wat characters te verwijderen. Het \textbf{D}\index{vim!delete tot einde regel} commando verwijdert de text vanaf de plek waarop je staat tot het einde van de regel. Het einde van de regel is tot de plek waar je \textsc{ENTER} gegeven hebt, dus niet tot het einde van de regel op het scherm.

Het verwijderen van een complete regel van begin tot eind doe je met \textbf{dd}\index{vim!delete een regel}. Dit commando kan je uitbreiden met aantallen regels door tussen de d's een getal op te geven. Bijvoorbeeld \textbf{d2d} verwijdert 2 regels vanaf de plek waar je staat. En zo kan je ook met \textbf{dl}\index{vim!delete character}letters verwijderen. De combinatie \textbf{d5l} verwijdert vanaf de positie van de cursor 5 characters. Om hele woorden weg te gooien gebruik je \textbf{dw}\index{vim!delete een word} en ook daar geldt de mogelijkheid om meerdere woorden in \'e\'en keer weg te gooiden met \textbf{d3w} gooi je 3 woorden achter elkaar weg.
