Van oudsher werden Unix systemen veel gebruikt door programmeurs. Er zijn dan ook veel talen ontwikkeld en overgezet naar Linux. Natuur is er een C-compiler. De meest gebruikte is die uit het GNU project die de GNU Compiler Collection (GCC) heet omdat hij naast \index{C}C ook compilers bevat voor \index{C++}C++, \index{Objective-C}Objective-C, \index{Fortran}Fortran, \index{Ada}Ada, \index{Go}Go en \index{D}D.

Ook voor scripting talen zijn er veel \foreignlanguage{english}{interpreters} aanwezig zoals voor PHP, Perl, Python en Java.

Daarnaast is er via verschillende kanalen nog veel meer te installeren.

In het document Linux Introductie (ook beschikbaar via https://github.com/DennisLeeuw/Linux) staat beschreven hoe je van C-source code naar een vorm komt die de CPU begrijpt. Dat proces heet compileren. Je gebruikt een compiler om bron-code om te zetten naar binaire code.

Er is nog een andere manier om een programma te draaien op je computer. Je kan ook de broncode omzetten naar binaire code via een runtime engine. Programmeertalen die hiervan gebruik maken heten scripting talen.

Het belangrijkste verschil tussen de twee vormen is de manier waarop een applicatie verspreid wordt. Als je de broncode compileert en er een binairbestand van maakt dan kan je alleen het binaire bestand delen en iedereen met dezelfde OS-versie en hardware kan jouw software dan gebruiken. Dit is wat bijvoorbeeld Apple en Microsoft doen en hoe veel van ook de Linux software verspreid wordt via de repositories. Als je software een script is dan is de verspreiding per definitie als broncode. Een shell-script, JavaScript, PHP of Python wordt bijna altijd als broncode verspreid en een runtime engine wordt gebruikt om het script te starten op de computer.

Scripting had de naam om traag te zijn, maar dat is bij moderne scripting talen zoals Python inmiddels bijna niet meer het geval.

