Een van de allereerste distributies was het Softlanding Linux System (SLS) door Peter MacDonald in 1992, deze Linux distributie bevatte, als eerste, een grafische interface. Het stond bekent om zijn buggy character en er ontstonden dan ook al snel opvolgers zoals Yggdrasil en Slackware van Patrick Volkerding. Slackware kwam in 1993 uit en is de oudste nog steeds bestaande distributie. Ook Debian is een afgeleide van SLS.

Een Linux distributie is een collectie van software samen met de Linux-kernel. Veel van de software is afkomstig van het GNU-project. Makers van een distributie maken hun eigen keuzes welke software zij belangrijk vinden. Daarom zijn er ook veel verschillende distributies omdat er zoveel mogelijk is met open source in iedereen iets anders belangrijk vindt. We kunnen je in dit document niet kennis laten maken met alle bestaande distributies, maar we kunnen wel een paar van de belangrijkste distributies voor je beschrijven.

De software die meegeleverd wordt met een distributie is allemaal voor gecompileerde software, je krijgt dus binairies net als bij Windows en Mac OS. Je hoeft de software niet meer zelf te compileren. Om die voorgecompileerde software te kunnen installeren is er een package manager nodig. Een stukje software dat de binaries allemaal op de juiste plek op de harddisk zet en er eventueel voor zorgt dat benodigde extra software, zoals libraries, ook ge\"installeerd worden. Verschillende distributies gebruiken verschillende package managers.
