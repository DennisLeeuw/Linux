\documentclass[a4paper,12pt,twoside,openright,titlepage]{book}

%Additional packages
\usepackage[utf8]{inputenc}
\usepackage[T1]{fontenc}
\usepackage[dutch,english]{babel}
\usepackage{syntonly}
\usepackage[official]{eurosym}

% Handle images
%\usepackage[graphicx]
\usepackage{graphicx}
\graphicspath{ {./images/}{./styles/} }
\usepackage{float}
\usepackage{wrapfig}

% Handle URLs
\usepackage{xurl}
\usepackage{hyperref}
\hypersetup{colorlinks=true, linkcolor=blue, citecolor=blue, filecolor=blue, urlcolor=blue, pdftitle=, pdfauthor=, pdfsubject=, pdfkeywords=}

% Tables and listings
\usepackage{tabularx}
\usepackage{scrextend}
\addtokomafont{labelinglabel}{\sffamily}
\usepackage{listings}
\usepackage{adjustbox}

% Turn on indexing
\usepackage{imakeidx}
\makeindex[intoc]

% Define colors
\usepackage{color}
\definecolor{ashgrey}{rgb}{0.7, 0.75, 0.71}



% Listing style
\input{styles/lstset}

% Uncomment for production
% \syntaxonly

% Style
\pagestyle{headings}

% Turn on indexing
\makeindex[intoc]

% Define document
\author{D. Leeuw}
\title{Linux de grafische interface}
%\subtitle{Linux voor MBO niveau 4 en het LPI Linux Essentials examen}
%\subject{Een Praktische Gids}
\date{\today\\v.1.0.0}

\begin{document}
\selectlanguage{dutch}

\maketitle

\copyright\ 2020-2022 Dennis Leeuw\\

\input{styles/licentie}

%%%%%%%%%%%%%%%%%%%
%%% Introductie %%%
%%%%%%%%%%%%%%%%%%%

\frontmatter
\chapter{Over dit Document}
\input{src/OverDitDocument}
\begin{flushleft}
\begin{table}[h!]
\centering
\begin{tabularx}{\textwidth}{ |c|c|c|X| }
\hline
	Versienummer &
	Auteurs &
	Verspreiding &
	Wijzigingen\\
\hline
	1.0.0 &
	Dennis Leeuw &
	 &
	Opdrachten naar eigen opdrachten document. Veel taal- en stijlfouten uit de teksten gehaald\\
\hline
	0.9.0 &
	Dennis Leeuw &
	 &
	Na splitsing GUI document in Intro en GUI\\
\hline
\end{tabularx}
\caption{Document wijzigingen}
\label{table:1}
\end{table}
\end{flushleft}



%%%%%%%%%%%%%%%%%
%%% De inhoud %%%
%%%%%%%%%%%%%%%%%
\tableofcontents

\mainmatter
\chapter{Inleiding}
Deze Linux cursus beoogt aan te sluiten bij het Linux Essentials examen van de LPI (Linux Professional Institute) en dient als voorbereiding op het MBO ICT Systems and Devices Expert examen. Voor het leren gebruiken van de grafische interface en de command line maken we gebruik van CentOS en om kennis te maken met het gebruik van Linux als server/command line installeren we Debian. De keuze om CentOS als werkstation te installeren en Debian als server is volledig willekeurig. Het doel is dat de studenten kennis maken met de zowel rpm/dnf en de apt package managers en leren dat het ene Linux systeem het andere niet is.

Alle Linux systemen zullen ge\"installeerd worden als virtuele machines. Door gebruik te maken van virtuele machines zijn we niet afhankelijk van de onderliggende hardware. De keuze van de virtuele omgeving is aan de gebruiker. Ons advies zou zijn om gebruik te maken van VirtualBox en VMware Workstation, beide zijn gratis en kunnen op de meest gangbare operating systems gebruikt worden.

Voor de CentOS machine is 15G vrije schijfruimte nodig en voor het Debian systeem 5G, wat een totaal aan 20G vrije schijfruimte vereist. Voor elke machine hebben we 2G RAM nodig, dus een totaal van 4G RAM moet vrij beschikbaar zijn.


Dit document behandelt de GUI ofwel de Graphical User Interface van Linux. Heel specifiek behandelen we de GNOME desktop omgeving zoals deze meegeleverd wordt met CentOS.


\chapter{Werken met de Desktop}
De Unix-wereld houdt erg van het paradigma ``Small is beautiful''. Daarmee bedoelen ze dat ze
graag kleine tools maken die \'e\'en ding goed doen. Dat zien we ook terug bij de grafische interface\index{Grafische Interface}. Allereerst is er
een display server, dit is een stuk software dat ervoor zorgt dat er een grafische interface is. Het luistert naar de
muis, bestuurt de cursor en toont een grafisch scherm en dat is het wel zo'n beetje. Op deze grafische server\index{Grafische server}
draait een window-manager\index{Window-manager}. De window-manager vangt een applicatie in een frame (een window) en zorgt ervoor dat er naar
wens scrol-knoppen zijn en knopjes om het scherm te minimaliseren en/of te sluiten. Ook het achtergrondscherm is een
taak van de window-manager. Als laatste is er de desktop\index{Desktop} omgeving die zorgt voor de taakbalk, het configuratiescherm en
alle andere zaken die nodig zijn om van een desktop te kunnen spreken.

{\selectlanguage{dutch}
Als we dit allemaal hebben hebben we een desktop omgeving waarbinnen applicaties kunnen draaien.}

{\selectlanguage{dutch}
Er zijn twee dominante desktop omgevingen beschikbaar op de verschillende Linux distributies en dat zijn KDE en GNOME.
Naast deze twee zijn er nog vele verschillende anderen, maar die zullen we hier niet bespreken.}

KDE\index{KDE} was, van de twee genoemde desktopomgevingen, de eerste. Het is gebaseerd op de Qt-library\index{Qt}. In
het begin was de Qt-library geen open source vandaar dat er een concurrerent project is ontstaan. Later is het met Qt
helemaal goed gekomen en nu behoort ze tot de open source gemeenschap.

{\selectlanguage{dutch}
GNOME\index{GNOME} was het concurrerende project dat gestart werd omdat Qt niet open source was. Voor GNOME tot stand kwam was er een
open source fotomanipulatie applicatie dat The \index{GIMP}GIMP heet, zie later in dit hoofdstuk. Om het pakket te kunnen
maken hadden de ontwikkelaars een grafische library ontwikkeld die GTK\index{GTK} werd genoemd. Veel van wat er nodig is voor een
desktop zat daar al in en dus gebruikte het GNOME project de GTK-library als basis.}

De grafische interface kan enorm verschillen per distributie. Het maakt al enorm veel verschil
of je KDE of GNOME gebruikt als desktop omgeving. Laat je hierdoor niet imponeren, het wijst zich vaak vanzelf. KDE
ligt qua interface het dichtst tegen Windows aan, en zal dus het makkelijkst zijn om naar over te stappen. CentOS
gebruikt GNOME en vergt iets meer doorzettingsvermogen om te doorgronden.

Mocht het scherm in zijn screensaver\index{Screensaver!Login} vallen dan kan je door klikken met de muis een scherm krijgen waarop
de tijd te zien is, daarna kom je met de {\textless}Enter{\textgreater} toets op een login scherm en kan je in loggen met je
gebruikersnaam en wachtwoord.

Eenmaal ingelogd kan je door op Activities\index{Activities} te klikken extra scherm elementen te zien krijgen (zie figuur \ref{fig:de_activities}). Het nog een keer
aanklikken van Activities verbergt de elementen weer waardoor je meer ruimte op je desktop hebt voor applicaties (zie figuur \ref{fig:de_deactivities}).

\begin{figure}[H]
\includegraphics[width=0.9\textwidth]{linuxreader-img013.png}
	\caption{De desktop met alle scherm elementen}
	\label{fig:de_activities}
\end{figure}
\begin{figure}[H]
\includegraphics[width=0.9\textwidth]{linuxreader-img014.png}
	\caption{De desktop zonder alle scherm elementen}
	\label{fig:de_deactivities}
\end{figure}

\section{Zoeken van bestanden of applicaties}
Met alle elementen op het scherm kan je de searchbar\index{Searchbar}\index{Zoeken}, midden boven, gebruiken om te zoeken naar applicaties\index{Applicaties!Zoeken}\index{Zoeken!Applicaties} en bestanden\index{Bestanden!Zoeken}\index{Zoeken!Bestanden}. Als je zoekt op
Word, een Microsoft applicatie die niet op Linux beschikbaar is, dan vind je LibreOffice Writer een gratis en open
source alternatief.

Meer over LibreOffice en de verschillende onderdelen van dit office pakket komt later aan de
orde als we Office Pakketten gaan bespreken. Nu concentreren we ons eerst op de beschikbare scherm elementen.

\section{Systeem configuratie}
Op de donkere balk waarop ook Activities staat vind je aan de rechterkant een naar beneden wijzend driehoekje. Het
aanklikken van het driehoekje geeft een menu met daarop een overzicht van de helderheid\index{Scherm!Helderheid} van het scherm, aan welk
netwerk\index{Desktop!Netwerk}\index{Netwerk} je gekoppeld bent, als je een laptop gebruikt wat de batterij status\index{Batterij status}\index{Desktop!Batterij status} is, je loginnaam\index{Loginnaam}\index{Desktop!Loginnaam} en drie knopjes die je
van links naar rechts toegang geven tot de systeemsettings\index{Systeem configuratie}\index{Desktop!Systeem configuratie}, het locken van je scherm\index{Scherm!Locken} en het uitzetten\index{Uitzetten}\index{Shutdown}\index{Desktop!Shutdown}\index{Desktop!Uitzetten} of herstarten\index{Herstarten}\index{Reboot}\index{Desktop!Herstarten}\index{Desktop!Reboot} van
je machine (zie figuur \ref{fig:de_status}).



\begin{figure}[H]
\includegraphics[width=0.9\textwidth]{linuxreader-img016.png}
	\caption{Status overzicht}
	\label{fig:de_status}
\end{figure}
Selecteer Settings\index{Settings}\index{Desktop!Settings}, scroll naar beneden naar Devices en selecteer deze, klik dan op Displays. Trek het scherm los van
de topbar en schuif hem naar links. \index{Resolutie}\index{Scherm!Resolutie}Klik op de 800x600 resolutie en zet deze naar 1024x768, zoals weergegeven in figuur \ref{fig:de_resolutie}.

Klik op de Apply knop rechtsboven aan het scherm en daarna op Keep Settings. Natuurlijk mag je
de resolutie ook hoger zetten, maar de minimale resolutie waarmee GNOME op CentOS 8 op een virtual machine prettig werkt
zonder dat je steeds met windows moet slepen is 1024x768. Selecteer {\textless} in de balk van Devices om terug te
komen in het hoofdmenu voor Settings. Loop door de verschillende opties om te ervaren waar je welke configuratie items
kan vinden en wijzigen.

\begin{figure}[H]
\includegraphics[width=0.9\textwidth]{linuxreader-img017.png}
	\caption{Scherm resolutie}
	\label{fig:de_resolutie}
\end{figure}


\section{The Dash}
\begin{wrapfigure}{l}{0.2\textwidth}
	\begin{center}
	\includegraphics[width=30px]{linuxreader-img018.png}
	\end{center}
	\caption{The Dash}
	\label{fig:de_dash}
\end{wrapfigure}

Aan de linkerkant van je scherm heb je de Dash, ook bekend als de Dock, applicatie bar of taskbar. Als je met je muis over de iconen van de taskbar gaat dan zie je per icoon wat deze betekent. Van boven naar beneden kom je het volgende tegen.

\begin{itemize}
\item Firefox -- een webbrowser
\item Evolution -- Een e-mail client
\item Rhythmbox -- een muziekspeler
\item Files -- Bestandsbrowser
\item Software -- Softwarebeheer
\item Help -- Documentatie
\item Terminal -- Toegang tot de console
\item Show applications -- een beperkt overzicht van beschikbare applicaties.
\end{itemize}

In de volgende hoofdstukken zullen we deze elementen doorlopen maar in een bredere context. We zullen bijvoorbeeld niet alleen Firefox behandelen, maar webbrowsers in zijn algemeenheid.


\section{Bestandsbrowser}
De filebrowser\index{Filebrowser}\index{Desktop!Filebrowser} kan gevonden worden op de Dash met het archief icon.
\begin{figure}[H]
	\centering
\includegraphics{filebrowser-dash.png}
	\caption{Filebrowser Icoon}
	\label{fig:de_filebrowser_icon}
\end{figure}

De filebrowser geeft je de mogelijkheid om op een grafische manier door de mappen en bestanden van het systeem te bladeren.
\begin{center}
\begin{figure}[H]
\includegraphics[width=\linewidth]{filebrowser.png}
	\caption{Filebrowser}
	\label{fig:de_filebrowser}
\end{figure}
\end{center}




\chapter{Installeren en updaten van software}
Via het Software\index{Software}\index{Desktop!Software} icoon, zie figuur \ref{fig:de_software_icon}, op de dash\index{Dash!Software} kan je de applicatie opstarten om de software op je systeem te beheren.

\begin{figure}[H]
\includegraphics{software-dash.png}
	\caption{Software icoon}
	\label{fig:de_software_icon}
\end{figure}

Je kan er applicaties mee toevoegen aan je systeem, verwijderen van je systeem of de bestaande applicaties updaten naar de laatste versie. Met een dubbel klik op het icoon start je de Software applicatie op.

\begin{figure}[H]
\includegraphics[width=0.9\textwidth]{linuxreader-img019.png}
	\caption{Software start scherm}
	\label{fig:de_software_start}
\end{figure}



\section{Installeren van GIMP}
Als we de Software\index{Software!Installatie} applicatie gestart hebben dan kunnen we
rechtsboven zoeken op een applicatie, we kunnen echter ook kiezen voor applicaties uit een Categorie. Klikken
we op Graphics \& Photography. Dan vinden we tussen de opties de GIMP\index{GIMP!Installatie}. Selecteer de GIMP en klik Install. Er zal
gevraagd worden om het root-wachtwoord, na dit ingevoerd te hebben begint de installatie.

Hierna kan je Software afsluiten of direct de GIMP opstarten.

\begin{figure}
\includegraphics[width=0.9\textwidth]{linuxreader-img020.png}
	\caption{Ge\"installeerde GIMP}
	\label{fig:de_software_gimp}
\end{figure}

\section{Show applications}
\input{src/GI_CentOS_ShowApplications}

\chapter{Internet}
\section{Webbrowsers}
Een populaire browser is Mozilla \index{Firefox}Firefox welke dan ook door veel distributies standaard meegeleverd wordt. Firefox is de doorontwikkelde browser van Netscape toen die in 1998 open source werd. Eerst hete de gehele suite Mozilla\index{Mozilla}. De library die voor alle HTML/CSS afhandeling zorgt is Gecko. Het Mozilla project heeft inmiddels vele software producten opgeleverd waarvan de belangrijkste de webbrowser Firefox en de e-mail client Thunderbird.

Het KDE-project heeft zijn eigen webbrowser: Konquerer. De KDE browser bestaat uit een engine en een interface. De engine is de library die alle benodigde functies voor het afhandelen van webpagina's heeft. Die engine is ooit begonnen als KHTML, maar hij heet nu \index{WebKit}WebKit en wordt inmiddels ook door Apple gebruikt voor zijn Safari browser.

Toen Google zijn eigen webbrowser ontwikkelde werd de basis hiervan vrij gegeven als open source
browser met de naam \index{Chromium}Chromium. Google gebruikt chromium als basis voor zijn Chrome browser en Microsoft gebruikt het als basis voor Edge. De open source versie is op de meeste Linux systemen te installeren via de standaard package managers zoals het eerder beschreven Software.

\section{E-mail clients}
Mozilla levert naast de browser Firefox ook een open source \index{e-mail!e-mail client}e-mail client met de naam
\index{Thunderbird}Thunderbird dit is een volwaardige e-mail client inclusief kalenderfunctionaliteit.

Een e-mail client die erg lijkt op Microsoft Outlook is \index{Evolution}Evolution. Sinds versie 2.8 is het onderdeel
van het GNOME project en Evolution is dan ook standaard ge\"installeerd op CentOS.

Evolution en Thunderbird draaien ook op Windows en Mac OS X.

Het KDE project heeft daarnaast ook zijn eigen e-mail client en die heet \index{KMail}KMail.

Standaard zijn er dus voor Linux al vele e-mail clients om uit te kiezen. Als je Op Internet gaat zoeken zijn er nog veel meer smaken beschikbaar. Dat is een van de
vele voordelen van open source, anderen zeggen een nadeel, er zijn enorm veel keuzes.


\chapter{Office Applicaties}
\section{Office pakketten}\index{Office pakketten}
\input{src/GIOffice}
\section{Grafische applicaties}
\input{src/GIGrafischeApps}
\subsection{The GIMP}\index{GIMP}\index{The GIMP}
\input{src/GIGIMP}
\subsection{Inkscape}\index{Inkscape}
\input{src/GIInkscape}
\section{Scribus}\index{Scribus}
\input{src/GIScribus}

\chapter{Multimedia}
\input{src/multimedia}
\section{Muziekspelers}\index{muziekspelers}
\input{src/muziekspelers}
\section{VLC}\index{VLC}\index{Video LAN Client}
VLC is uitgegroeid tot een cross-platform multimedia speler die veel gebruikt wordt om films mee te kijken. VLC draait op Linux, Mac OS X en Windows. VLC kan films ook spelen vanaf CD, DVD en BluRay. Het ondersteunt vele codecs en bestandsformaten. VLC kan je ook als server gebruiken om films te streamen over het netwerk.

\begin{center}
\begin{figure}[H]
\includegraphics[width=0.9\textwidth]{VLC-working.png}
\caption{VLC}
\end{figure}
\end{center}

\section{KODI}\index{KODI}
\input{src/kodi}

\chapter{Naar de command line}
\section{Terminal}
De terminal applicatie wordt gebruikt om op de command line terecht te komen. Deze applicatie komt in het CLI document uitgebreid aan
de orde.

\begin{figure}
\includegraphics{linuxreader-img021}
\caption{Terminal op de Dash}
\end{figure}


%%%%%%%%%%%%%%%%%%%%%
%%% Index and End %%%
%%%%%%%%%%%%%%%%%%%%%
\backmatter
\printindex
\end{document}

%%% Last line %%%
