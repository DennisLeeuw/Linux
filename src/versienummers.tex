Een Linux systeem is opgebouwd uit een kernel, extra software van bijvoorbeeld het GNU-project en nog veel meer software van verschillende plekken. En toch levert een distributie bouwer een systeem af met \'e\'en enkel versienummer. Voor de verschillende software pakketten zijn er een aantal standarden in omloop om het versienummer te achterhalen. Bij de meeste commando's kan je \texttt{-v} of \texttt{--version} gebruiken als optie het versienummer te achterhalen. Bij veel commando's werken beide opties, bij sommige alleen \'e\'en van de twee. De lange versie met \texttt{--version} is een GNU-optie, op het oorsponkelijke UNIX systeem bestonden geen lange opties. Dus de meeste GNU-commando's ondersteunen deze lange optie. Een heel enkele keer wordt \texttt{-v} voor iets anders gebruikt, dus het is verstandig om eerst de manual-pagina van het commando te raadplegen om te zien of de optie inderdaad het versienummer weergeeft.

In de rest van deze paragraaf gaan we het hebben over hoe je het versienummer van de distributie en Linux achterhaalt.
