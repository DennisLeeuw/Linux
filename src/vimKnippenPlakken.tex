Met het \textbf{x} commando kan je \'e\'en enkel character knippen.

De traditionele manier om een kopie van een stuk tekst te maken is het gebruik van het \textbf{y} commando\index{vim!copy}. Het \textbf{yl} commando kopieert een character, \textbf{yw} kopieert een woord en \textbf{yy} kopieert regels.

Het \textbf{p}\index{vim!paste} commando kan gebruikt worden om text te plakken. Het \textbf{p} commando is plakken achter de positie van de cursor en \textbf{P} is plakken voor de positie van de cursor.

Een speciale functie van vim en niet van vi is het gebruik van \textbf{v} om een visuele selectie\index{vim!visual selection} maken, met de pijltjes toetsen kan je nu bepalen hoe groot de selectie worden moet. Het commando \textbf{v} geeft je de mogelijkheid om de selectie op character niveau te maken. Met \textbf{V} maak je selecties per regel, hier gebruik je de omhoog en omlaag pijltjes toetsen om je selectie groter of kleiner te maken. Als je de selectie gemaakt hebt gebruik je \textbf{y} om te kopie\"eren of \textbf{d} om het stuk te verwijderen.

Met \textbf{u} kan je een wijziging ongedaan maken.

