Om een service te starten of te stoppen gebruikt \texttt{systemd} bestanden waarin beschreven wordt hoe een service heet en welke commando's er nodig zijn om te starten en stoppen. Deze beschrijving van een service wordt binnen \texttt{systemd} een \textbf{unit} genoemd. Units kan je onder andere terug vinden in de \texttt{/etc/systemd/system} directory. Een voorbeeld van een unit-file in Debian 11 is de \texttt{syslog.service}. Deze ziet er zo uit:

\begin{lstlisting}[language=bash]
[Unit]
Description=System Logging Service
Requires=syslog.socket
Documentation=man:rsyslogd(8)
Documentation=man:rsyslog.conf(5)
Documentation=https://www.rsyslog.com/doc/

[Service]
Type=notify
ExecStart=/usr/sbin/rsyslogd -n -iNONE
StandardOutput=null
Restart=on-failure

# Increase the default a bit in order to allow many simultaneous
# files to be monitored, we might need a lot of fds.
LimitNOFILE=16384

[Install]
WantedBy=multi-user.target
Alias=syslog.service
\end{lstlisting}

Als je een service hebt gewijzigd of toegevoegd moet je systemctl vertellen dat er iets nieuws is. Dat doe je met het volgende commando:
\begin{lstlisting}[language=bash]
$ sudo systemctl daemon-reload
\end{lstlisting}

