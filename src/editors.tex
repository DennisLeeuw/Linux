De eerste redelijk gebruiksvriendelijke editor op Unix was \texttt{vi}. De \texttt{vi} editor kent twee modi. De eerste modus is de \textquote{edit mode} en de tweede is de \textquote{command mode}. Standaard start \texttt{vi} op in de command mode waarin je commando's kunt geven om bestanden te laden of op te slaan en waarin je functies als knippen en plakken kan uitvoeren. De edit modus is die waarin je tekst invoert. Dit onderscheid maakt het voor beginnende gebruikers \texttt{vi} soms verwarrend.

Naast \texttt{vi} zijn er ook andere editors voor Unix-achtige systemen ontwikkeld. De meeste bekende zijn \texttt{pico}\index{pico} en \texttt{nano}\index{nano}. Pico was de oorspronkelijke editor. Nano is ontwikkled door het GNU-project en is een vervanging van pico omdat pico een licentie had die \textquote{problematisch} was. Dat probleem is inmiddels opgelost, maar nano biedt zoveel extra mogelijkheden dat velen de voorkeur geven aan nano.

Het grote voordeel van nano ten opzichte van vi is zijn gebruiksvriendelijke interface. Nano kent geen edit en command mode zoals vi. Nano gebruikt control codes om commando's te geven en is direct beschikbaar voor de invoer van tekst van de gebruiker.
