
Soms zou je willen dat je in een bestand kunt zoeken. Natuurlijk kan je met in een tekstverwerker of een editor zoeken in een bestand. Maar wat nu als je niet zeker meer weet in welk bestand het was dat je iets geschreven had. Dat kan zomaar gebeuren als je een boek zoals dit aan het schrijven bent. Dit boek is opgebouwd uit allemaal kleine bestanden die te samen het boek vormen. Op deze manier hou ik de onderwerpen gescheiden en hoef ik niet elke keer te scrollen om bij een ander deel te komen. Ik kan ook twee onderwerpen in twee verschillende terminals te gelijk open hebben staan en zo parallel aan elkaar werken. Dit deel gaat over hoe we in bestanden kunnen zoeken zonder dat we een tekstverwerker open hebben staan met ons document erin.

Er zijn verschillende commando's die we kunnen gebruiken, de meest gebruikte is denk ik \texttt{grep}. Met \texttt{grep} kan door regular expressions te gebruiken zoeken in bestanden. Meer over regular expressions vind je in de volgende sectie. Nu gaan we vooral kijken naar hoe het \texttt{grep} commando werkt.

