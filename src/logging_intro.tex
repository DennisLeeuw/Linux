Omdat daemons op de achtergrond draaien en geen direct contact met een gebruiker of met root hebben zullen ze op een andere manier moeten laten weten of het goed met ze gaat, wat ze aan het doen zijn en of de beheerder misschien moet ingrijpen omdat er iets fout gaat. Daemons laten ons weten wat er aan de hand is door hun berichten te loggen\index{logging}\index{loggen}. Loggen is het wegschrijven van de berichten naar een logbestand.

Linux systemen verzamelen alle log bestanden in een directory \texttt{/var/log}. Een daemon kan zelfstandig zijn eigen log wegschrijven naar een bestand, of de daemon kan gebruik maken van een log-server\index{log-server}. Het voordeel van een log-server is dat de daemon alleen de API van de log-server hoeft te kennen. Alle andere logica rond het loggen wordt afgehandel door de log-server. Voor meer informatie over de log-server zie de sectie over syslog (\ref{sec:syslog}).

Om direct te kunnen volgen wat er op een systeem gebeurt wordt veel het \texttt{tail} commando gebruikt. \texttt{tail} heeft een optie \texttt{-f} die follow betekent. Het blijft volgen welke nieuwe regels er aan een bestand worden toegevoegd. Een manier om bijvoorbeeld de \texttt{messages} log te blijven volgen is:
\begin{lstlisting}[language=bash]
$ sudo tail -f /var/log/messages
\end{lstlisting}

Omdat alle log-bestanden tekst bestanden zijn kan je ook bijvoorbeeld \texttt{grep} erop los laten en zo zoeken naar bepaalde patronen in de log-bestanden.

