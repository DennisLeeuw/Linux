De eerste vraag is meestal waarom we de command line zouden gebruiken als we al een grafische interface hebben. Het meest simpele antwoord is dat Unix van oorsprong alleen maar een command line interface of CLI had. Maar een beter antwoord is dat een grafische interface veel resources (geheugen en processor) gebruikt en die resources kunnen we beter inzetten voor de taken die we het systeem geven. Veel Linux machines draaien als servers in een serverruimte en staan op de achtergrond hun ding te doen, bijvoorbeeld als webserver. Voor die functie is geen grafische interface noodzakelijk terwijl op een drukbezochte website elk stukje processor of geheugen nodig kan zijn om de website soepel te laten lopen. Wat we niet nodig hebben installeren we dan ook niet op de machine, dus geen grafische interface.

Daarnaast zal je hopelijk ervaren dat, omdat Linux op de schouders staat van de vele jaren ervaring uit de Unix-wereld, dat de command line een enorm krachtige interface is om mee te werken. Vaak kan je op de command line dingen sneller en makkelijker doen dan je in een grafische interface zou kunnen. Wees niet bevreest als dat in eerste instantie niet zo lijkt. De leercurve kan, zeker als je niet veel ervaring hebt met bijvoorbeeld de command prompt of powershell van Windows, soms erg stijl zijn.

De voorbeelden in dit document zijn gemaakt op een Debian machine, maar daarmee niet Debian specifiek. Ze zullen op een Red Hat, Centos, Mint of Ubuntu machine geen wezenlijk andere resultaten opleveren. De grote verschillen tussen de verschillende distributies zitten vooral in de grafische interface, de verschillen op de command line zijn minimaal voor de onderwerpen die in dit document worden behandeld. Het doel van dit document is je vetrouwt maken met de CLI, niet het configureren van van het systeem.

