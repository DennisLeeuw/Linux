Om bestanden te kopie\"eren gebruiken \texttt{cp}\index{cp}\index{Bestanden!cp} van het Engelse copy:
\begin{lstlisting}[language=bash]
$ cp hello.txt Boom/hello.txt
\end{lstlisting}
We kunnen ook gelijk de naam veranderen als we dat willen:
\begin{lstlisting}[language=bash]
$ cp hello.txt Boom/Hallo.txt
\end{lstlisting}

Om bestanden verplaatsen gebruiken \texttt{mv}\index{mv}\index{Bestanden!mv} van het Engelse move. Het verschil met copy is dat een bestand niet meer op de oorspronkelijke plek terug te vinden is. Bij move heb je dus maar 1 bestand na de handeling, na copy heb je 2 bestanden.
\begin{lstlisting}[language=bash]
$ mv Boom/Hallo.txt .
\end{lstlisting}
Het \texttt{mv} commando kunnen we ook gebruiken om bestanden van naam te veranderen:
\begin{lstlisting}[language=bash]
$ mv Hallo.txt hallo.txt
\end{lstlisting}
Veranderd de naam.

Voor het weggooien van bestanden gebruiken we \texttt{rm} van remove.
\begin{lstlisting}[language=bash]
$ rm hallo.txt
\end{lstlisting}

Omdat alles een bestand is op een Linux systeem zijn ook directories bestanden, speciale bestanden, maar toch bestanden. We hebben al gezien dat we met \texttt{rmdir} lege directories weg kunnen gooien. Zouden we nu \texttt{rm} kunnen gebruiken om ook directories weg te gooien. Ja, dat kan, maar ook hier geldt dat de directory leeg moet zijn.
\begin{lstlisting}[language=bash]
$ rm Boom
\end{lstlisting}
geeft weer een foutmelding. Het systeem zegt tegen ons dat \texttt{Boom} een directory is. Als we tegen \texttt{rm} vertellen dat hij de zaken recursive weg met gooien, dan zal de hele boomstructuur wegggegooid worden:
\begin{lstlisting}[language=bash]
$ rm -r Boom
\end{lstlisting}
