\documentclass[a4paper,12pt,twoside,openright,titlepage]{book}

%Additional packages
\usepackage[ascii]{inputenc}
\usepackage[T1]{fontenc}
\usepackage[dutch,english]{babel}
\usepackage{syntonly}
\usepackage[official]{eurosym}
%\usepackage[graphicx]
\usepackage{graphicx}
\graphicspath{ {./images/} }
\usepackage{float}
\usepackage{xurl}
\usepackage{hyperref}
\hypersetup{colorlinks=true, linkcolor=blue, citecolor=blue, filecolor=blue, urlcolor=blue, pdftitle=, pdfauthor=, pdfsubject=, pdfkeywords=}
\usepackage{tabularx}
\usepackage{scrextend}
\addtokomafont{labelinglabel}{\sffamily}
\usepackage{listings}
\usepackage{adjustbox}

% Turn on indexing
\usepackage{imakeidx}
\makeindex[intoc]

% Define colors
\usepackage{color}
\definecolor{ashgrey}{rgb}{0.7, 0.75, 0.71}

% Listing style
\lstset{
  backgroundcolor=\color{ashgrey},   % choose the background color; you must add \usepackage{color} or \usepackage{xcolor}; should come as last argument
  basicstyle=\footnotesize,        % the size of the fonts that are used for the code
  breakatwhitespace=false,         % sets if automatic breaks should only happen at whitespace
  breaklines=true,                 % sets automatic line breaking
  extendedchars=true,              % lets you use non-ASCII characters; for 8-bits encodings only, does not work with UTF-8
  frame=single,	                   % adds a frame around the code
  keepspaces=true,                 % keeps spaces in text, useful for keeping indentation of code (possibly needs columns=flexible)
  rulecolor=\color{black},         % if not set, the frame-color may be changed on line-breaks within not-black text (e.g. comments (green here))
  showspaces=false,                % show spaces everywhere adding particular underscores; it overrides 'showstringspaces'
}

% Uncomment for production
% \syntaxonly

% Style
\pagestyle{headings}

% Define document
\author{D. Leeuw}
\title{Linux Server}
%\subtitle{Linux voor MBO niveau 4 en het LPI Linux Essentials examen}
%\subject{Een Praktische Gids}
\date{\today\\v.0.1.0}

\begin{document}
\selectlanguage{dutch}

\maketitle

\copyright\ 2021-2022 Dennis Leeuw\\

\begin{figure}
\includegraphics[width=0.3\textwidth]{CC-BY-SA-NC.png}
\end{figure}

\bigskip

\input{src/licentie}

%%%%%%%%%%%%%%%%%%%
%%% Introductie %%%
%%%%%%%%%%%%%%%%%%%

\frontmatter
\chapter{Over dit Document}
\input{src/OverDitDocument}
\input{src/DocChanges_Server}

%%%%%%%%%%%%%%%%%
%%% De inhoud %%%
%%%%%%%%%%%%%%%%%
\tableofcontents

\mainmatter
\chapter{Inleiding}
Deze Linux cursus beoogt aan te sluiten bij het Linux Essentials examen van de LPI (Linux Professional Institute) en dient als voorbereiding op het MBO ICT Systems and Devices Expert examen. Voor het leren gebruiken van de grafische interface en de command line maken we gebruik van CentOS en om kennis te maken met het gebruik van Linux als server/command line installeren we Debian. De keuze om CentOS als werkstation te installeren en Debian als server is volledig willekeurig. Het doel is dat de studenten kennis maken met de zowel rpm/dnf en de apt package managers en leren dat het ene Linux systeem het andere niet is.

Alle Linux systemen zullen ge\"installeerd worden als virtuele machines. Door gebruik te maken van virtuele machines zijn we niet afhankelijk van de onderliggende hardware. De keuze van de virtuele omgeving is aan de gebruiker. Ons advies zou zijn om gebruik te maken van VirtualBox en VMware Workstation, beide zijn gratis en kunnen op de meest gangbare operating systems gebruikt worden.

Voor de CentOS machine is 15G vrije schijfruimte nodig en voor het Debian systeem 5G, wat een totaal aan 20G vrije schijfruimte vereist. Voor elke machine hebben we 2G RAM nodig, dus een totaal van 4G RAM moet vrij beschikbaar zijn.


%Management netwerk

\chapter{Debian installatie}

\begin{enumerate}
\item Selecteer de Grafische installatie interface in de Debian Installer Menu (\ref{DebInstallMenu}).

\begin{figure}[H]
	\centering
	\includegraphics[width=\linewidth]{debian_install_01.png}
	\caption{Debian Installer Menu}
	\label{DebInstallMenu}
\end{figure}

\item Selecteer daarna de Engelse taal. Bij problemen is de kans dat er iemand in het Engels het probleem al eens is tegengekomen vele malen groter dan in het Nederlands, dus wen je aan om in het Engels te werken met Linux.

\item Bij Location kiezen we eerst other en daarna Europe en dan Netherlands.

\item Voor de Locales laten we staan en\_US.UTF-8.

\item Voor de keyboard houden we American English aan, tenzij je een specifiek toetsenbord hebt.

\item De volgende keuze die je moet maken is die voor een hostname, kortom hoe gaat je computer heten. Kies hiervoor je achternaam met 01 eraan vast. Voor mij zou dat dus leeuw01 zijn. Let op! allemaal kleine letters.

\item De keuze voor het domein waarin de machine zit is lastig omdat we geen domein hebben. Het makkelijkst is om hier voorlopig localdomain in te vullen. We kunnen dat later nog wijzigen.

\item Nu moeten we een root wachtwoord kiezen. Het is verstandig om een sterk wachtwoord te kiezen, omdat de root gebruiker alles mag.

\item Ook voor de gebruiker is dat een goed plan. Gebruik een password manager om een wachtwoord te generen of om je gekozen wachtwoorden erin op te slaan.

\item We houden de harddisk indeling simpel, dus kiezen we Guided als onze optie voor de harddisk indeling zoals aangegeven in \ref{DebDiskPart}.

\begin{figure}[H]
	\centering
	\includegraphics[width=\linewidth]{debian_install_partition.png}
	\caption{Debian Disk Partitioning}
	\label{DebDiskPart}
\end{figure}

\item We kiezen de enige harddisk die er is en kiezen voor All files in one partition. In het overzicht dat we te zien krijgen zien we dat er naast een bestandssysteem ook een swap-partitie aangemaakt is zoals weergeven in figuur \ref{DebDiskParts}.

\begin{figure}[H]
	\centering
	\includegraphics[width=\linewidth]{Debian_Install_Partitioned.png}
	\caption{Debian Disk Overview}
	\label{DebDiskParts}
\end{figure}

\item Nadat we hebben gekozen voor Yes bij de vraag of we het zeker weten zal Debian beginnen met de installatie van het systeem.

\item Na de installatie van het base system (een minimaal systeem), krijgen we de vraag of we nog meer software van andere CD's willen installeren. Dat willen we niet.

\item Debian gaat nu opzoek naar updates en additionele software die ge\"installeerd kan worden. De repo's van Debian staan over de wereld verspreid, om geen data op te halen van bijvoorbeeld servers in de Verenigde Staten, maar alleen van servers in Nederland, zodat de belasting van het Internet alleen lokaal is, moeten we een keuze maken vanaf welke servers wij onze software willen ophalen. Kies Netherlands en deb.debian.org. We gebruiken geen proxy dus bij die vraag mag je continue geven.

\item Debian gaat nu de databases ophalen van alle software die er voor Debian beschikbaar is en gaat daarna meteen je systeem updaten met de laatst beschikbare versies. Afhankelijk van hoe nieuw de ISO was waarvan je hebt geboot kan dit even duren. Je kan af en toe een vraag krijgen van de installer, lees de tekst dan goed en bepaal door en bepaal of je het wel of niet wilt wat er gevraagd wordt, of vraag je docent.

\item Als de update klaar is krijg je de keuze om nog extra software te installeren (zie \ref{DebSoft}).

\begin{figure}[H]
	\centering
	\includegraphics[width=\linewidth]{Debian_Install_Software_uit.png}
	\caption{Debian Additional Software Choice}
	\label{DebSoft}
\end{figure}

We willen alleen de Standard System Utilities, dus die moet aan. De rest moet uit staan.

\item Nadat Debian klaar is met de installatie kiezen we ervoor om GRUB te installeren in de master boot record

\item We selecteren \textbf{/dev/sda} (zie \ref{DebGrubDev}
\begin{figure}[H]
	\centering
	\includegraphics[width=\linewidth]{Debian_Install_Grub-device.png}
	\caption{Debian Install Grub Device}
	\label{DebGrubDev}
\end{figure}

\item Als alles klaar is rebooten we het systeem.
\end{enumerate}

Login als \texttt{root} met het root-wachtwoord dat je hebt aangemaakt tijdens de installatie. Installeer het sudo pakket:
\begin{lstlisting}[language=bash]
# apt-get install sudo
\end{lstlisting}

Voeg daarna jezelf als gebruiker toe aan de sudo groep:
\begin{lstlisting}[language=bash]
# usermod -a -G sudo dennis
\end{lstlisting}
Vervang \texttt{dennis} door je eigen gebruikersnaam; degene die je aangemaakt hebt bij de installatie.

We loggen uit:
\begin{lstlisting}[language=bash]
# exit
\end{lstlisting}

We loggen nu in als een gewone gebruiker. Nadat je ingelogd bent type je op de prompt:
\begin{lstlisting}[language=bash]
$ sudo shutdown -h now
\end{lstlisting}
Je VM zal afsluiten en uit gaan. Voortaan gebruik je altijd dit commando om je VM uit te zetten. Zet je VM nooit uit met de power-knop, het kan je Linux-systeem beschadigen en dan moet je deze hele installatie opnieuw doen.




\chapter{Werken met daemons}
In de Unix wereld zeggen we altijd 'Everything is a file and if it is not a file it is a process'.

Processen die op de achtergrond draaien heten daemons in de Unix-wereld. Ze zijn ook bekend als services, zeker in de Windows wereld. Daemons zijn dus processen waar je weinig van merkt en die rustig hun werk staan te doen. Deze processen draaien zonder dat ze input van de gebruiker nodig hebben.

Bekende daemons zijn bijvoorbeeld een web- of mailservers, maar het kan ook een printserver zijn.


\section{init en systemd}
Nadat de kernel in het geheugen geladen is zal deze gestart worden. De Linux kernel zorgt ervoor dat de beschikbare hardware klaar is voor gebruik en dat er processen gestart kunnen worden. Het eerste proces dat de kernel start is \texttt{systemd}\index{systemd}. Vroeger was dit \texttt{init}\index{init} en dat kan je op vele ander Unix-achtige besturingssystemen nog tegen komen, maar de meeste Linux distributies zijn over naar \texttt{systemd}.

\texttt{systemd} is het proces dat ervoor zorgt dat alle ander processen gestart worden. De \texttt{systemd} daemon heeft proces nummer 1. Het is de eerste daemon die start bij het opstarten van het systeem en de laatste die afgesloten wordt bij het afsluiten van het systeem.

Naast de \texttt{systemd} daemon die de moeder is van alle processen, zijn er ook commando's waarmee je (als root) daemons kunt starten, stoppen en herstarten. Het commando dat doorvoor beschikbaar is heet \texttt{systemctl}\index{systemctl}. Een standaard met de installatie meegekomen daemon in de systemd-timesyncd\index{timesyncd}\index{systemd!timesyncd} daemon. We gaan deze daemon gebruiken om een beetje vertrouwd te raken met \texttt{systemctl}.

\begin{lstlisting}[language=bash]
sudo systemctl status systemd-timesyncd
\end{lstlisting}

\begin{figure}[h]
\includegraphics[width=8cm]{systemd-timesyncd-status}
\centering
	\caption{Status output van systemctl}
	\label{scrn:systemd-timesyncd-status}
\end{figure}

Als je net als in het voorbeeld (Figuur \ref{scrn:systemd-timesyncd-status}) een lijn hebt met END dan kan je gebruik maken van 'q' om weer op de command-prompt terecht te komen.

Door aan \texttt{systemctl} de commando's stop of start mee te geven kunnen we daemons op ons systeem stoppen en starten. De tijd synchronisatie daemon is op dit moment gestart, dus het eerste wat we kunnen doen is hem stoppen:

\begin{lstlisting}[language=bash]
sudo systemctl stop systemd-timesyncd
\end{lstlisting}

Met \texttt{systemctl status} kunnen nu de status van de daemon zien en dan zien we dat deze gestopt is. Het opnieuw opstarten doen we met start:

\begin{lstlisting}[language=bash]
sudo systemctl start systemd-timesyncd
\end{lstlisting}


\section{Logging}
\subsection{journalctl}
\texttt{systemd}\index{systemd} is het eerste proces (daemon) dat opgestart wordt op een Linux systeem. Het is daarmee de moeder van alle processen. Als er een daemon opgestart moet worden dan wordt dat gedaan via \texttt{systemd}. \texttt{systemd} is dus de plek waar ook in de gaten gehouden kan worden wat er opgestart is en wat niet en daarmee kan \texttt{systemd} een bijzondere inkijk geven in de status van een systeem. Om meer inzicht te krijgen in wat \texttt{systemd} allemaal voorbij heeft zien komen is er een tool met de naam \texttt{journalctl}\index{journalctl}.

Het opstarten van \texttt{journalctl} zonder enige opties of argumenten laat zien wat \texttt{systemd} heeft verzameld sinds het opstarten van het systeem, of sinds de laatste logrotate.

Om alleen de berichten te zien van een specifiek proces kunnen we het volgende commando gebruiken:
\begin{lstlisting}[language=bash]
$ sudo journalctl -f -u systemd-journald
\end{lstlisting}
De \texttt{-f} optie doet hetzelfde als bij \texttt{tail}, het zegt dat de log gevolgd (follow) moet worden. Met \texttt{-u} geef je op welke 'unit' er gemonitord moet worden.


\subsection{syslog}
\subsection{Log rotation}
\section{Je eigen startup script}
Om een service te starten of te stoppen gebruikt \texttt{systemd} bestanden waarin beschreven wordt hoe een service heet en welke commando's er nodig zijn om te starten en stoppen. Deze beschrijving van een service wordt binnen \texttt{systemd} een \textbf{unit} genoemd. Units kan je onder andere terug vinden in de \texttt{/etc/systemd/system} directory. Een voorbeeld van een unit-file in Debian 11 is de \texttt{syslog.service}. Deze ziet er zo uit:

\begin{lstlisting}[language=bash]
[Unit]
Description=System Logging Service
Requires=syslog.socket
Documentation=man:rsyslogd(8)
Documentation=man:rsyslog.conf(5)
Documentation=https://www.rsyslog.com/doc/

[Service]
Type=notify
ExecStart=/usr/sbin/rsyslogd -n -iNONE
StandardOutput=null
Restart=on-failure

# Increase the default a bit in order to allow many simultaneous
# files to be monitored, we might need a lot of fds.
LimitNOFILE=16384

[Install]
WantedBy=multi-user.target
Alias=syslog.service
\end{lstlisting}

Als je een service hebt gewijzigd of toegevoegd moet je systemctl vertellen dat er iets nieuws is. Dat doe je met het volgende commando:
\begin{lstlisting}[language=bash]
$ sudo systemctl daemon-reload
\end{lstlisting}



\chapter{Linux in een Windows Domain}
\section{SaMBa}

\chapter{Remote access}
%alleen vanaf het management netwerk
\section{SSH}

\chapter{Mailserver}
\section{postfix}

\chapter{LAMP}
LAMP is een afkorting voor Linux, Apache, MySQL, PHP. Het is een veel gebruikte 'stack' voor het opzetten van webservers op het Internet. Linux is hierbij het operating systeem, Apache de webserver, MySQL de database en PHP de taal die ervoor zorgt dat de browser HTML pagina's met content krijgt, ofwel de server-side scripting language.

In de loop van de tijd zijn er veel variaties op deze afkorting gekomen door vervanging van bijvoorbeeld Linux door Windows (WAMP), of door MacOS X (MAMP), maar er kan ook een andere databases gebruikt worden zoals MariaDB dat een vervanging is voor MySQL, maar we zouden bijvoorbeeld ook PostgreSQL kunnen gebruiken (LAPP) of een andere webserver zoals bijvoorbeeld NGINX (waarbij we de E van engine gebruiken) LEMP, en de P kan ook Perl of Phyton zijn. Wij houden het bij de traditionele stack van LAMP maar vervangen MySQL door MariaDB omdat dat op de meeste systemen aanwezig is en bijna volledig compatible is met MySQL.

Apache, MariaDB en PHP vormen samen de Middleware in onze PAAS (Platform As A Service) omgeving.


\section{Apache}
De meest bekende webservers in gebruik op het Internet zijn de Apache en NGINX webservers. Sinds enige tijd is de NGINX webserver populairder omdat de performance van deze webserver beter is dan die van apache. De configuratie van Apache is iets eenvoudiger dan die van NGINX vandaar dat we in dit document Apache behandelen om eerst een goed begrip te krijgen van de werking van een webserver.


\input{src/ApacheInstall}
\section{MariaDB}
% Extra disk voor DB
\input{src/MariaDBInstall}
\section{PHP}
\input{src/PHPInstall}

\chapter{NextCloud}
% Extra disk voor Data
\input{src/NextCloudIntro}
\section{NextCloud installatie}
\input{src/NextCloudInstallatie}
\input{src/NextCloudPHP}

%%%%%%%%%%%%%%%%%%%%%
%%% Index and End %%%
%%%%%%%%%%%%%%%%%%%%%
%\backmatter
\printindex
\end{document}

%%% Last line %%%
