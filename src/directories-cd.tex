Het \texttt{cd} commando gebruiken we om van directory te wisselen (cd staat voor \textbf{c}hange \textbf{d}irectory). We kunnen aan \texttt{cd} een relatief of een absoluut pad meegeven. Een absoluut pad is het complete pad vanaf de root-directory:
\begin{lstlisting}[language=bash]
$ cd /home/dennis
\end{lstlisting}
Waarbij je \textquote{dennis} vervangt door je eigen gebruikersnaam zorgt ervoor dat je in je eigen home-directory komt te staan. Als je een relatief pad gebruikt betekent dit dat je niet het hele pad meegeeft, maar een stukje. Bijvoorbeeld:
\begin{lstlisting}[language=bash]
$ cd LinuxCursus
\end{lstlisting}
Je geeft niet het complete pad \texttt{/home/dennis/LinuxCursus} op maar slechts het deel \texttt{LinuxCursus}, wat dus relatief is ten opzichte van de home-directory waarin je al staat.


