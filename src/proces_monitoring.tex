In het document 'Linux CLI' hebben we \texttt{ps}\index{ps} behandeld. Met \texttt{ps} kan je zien welke processen er draaien op een systeem. In deze sectie gaan we \texttt{ps} met andere opties gebruiken dan we eerder hebben gedaan. De Linux variant van \texttt{ps} kent de mogelijkheid om ook opties te gebruiken zonder gebruik van het min teken voor de optie, dit is de zogenaamde BSD-style:
\begin{lstlisting}[language=bash]
$ ps aux
\end{lstlisting}
doet bijna hetzelfde als \texttt{ps -ef}. De output ziet er ook bijna hetzelfde uit:
\begin{lstlisting}[language=bash]
$ ps aux
USER         PID %CPU %MEM    VSZ   RSS TTY      STAT START   TIME COMMAND
root           1  0.0  0.0 167316 11924 ?        Ss   Feb02   0:48 /sbin/init
root           2  0.0  0.0      0     0 ?        S    Feb02   0:00 [kthreadd]
root           3  0.0  0.0      0     0 ?        I<   Feb02   0:00 [rcu_gp]
root           4  0.0  0.0      0     0 ?        I<   Feb02   0:00 [rcu_par_gp]
root           9  0.0  0.0      0     0 ?        I<   Feb02   0:00 [mm_percpu_wq]
root          10  0.0  0.0      0     0 ?        S    Feb02   0:00 [rcu_tasks_rude_]
root          11  0.0  0.0      0     0 ?        S    Feb02   0:00 [rcu_tasks_trace]
root          12  0.0  0.0      0     0 ?        S    Feb02   1:25 [ksoftirqd/0]
root          13  0.0  0.0      0     0 ?        I    Feb02   6:07 [rcu_sched]
root          14  0.0  0.0      0     0 ?        S    Feb02   0:03 [migration/0]
root          16  0.0  0.0      0     0 ?        S    Feb02   0:00 [cpuhp/0]
root          17  0.0  0.0      0     0 ?        S    Feb02   0:00 [cpuhp/1]
root          18  0.0  0.0      0     0 ?        S    Feb02   0:03 [migration/1]
root          19  0.0  0.0      0     0 ?        S    Feb02   0:10 [ksoftirqd/1]
\end{lstlisting}
De extra informatie is voornamelijk de extra data over het gebruik van de CPU en van het geheugen (MEM). Deze informatie kennen we ook uit de \texttt{top} applicatie. Ook dit commando is behandeld in de 'Linux CLI' handleiding.

